\documentclass[a4paper,10pt]{report}
\usepackage[utf8x]{inputenc}
\usepackage{url}

%------------ My commands
\newcommand{\footurl}[1]{\footnote{\url{http://#1}}}
\newcommand{\notes}[1]{[NOTE: \textit{#1}]}
\newcommand{\code}[1]{\texttt{\url{#1}}}
\newcommand{\RESTURL}[1]{\\ \texttt{\url{#1}} \\}
\newcommand{\todo}[1]{\\ \textbf{TODO:#1} \\}
%-----------

\title{PhD Outline: Integrating the Real-World with the World Wide Web - A Service Platform to Support Application Development}
\author{Dominique Guinard}

% PhD Structure
% 
% Intro: sensor, web, mashups, etc.:
% 	explain the thesis and problems and thesis very early.
% 	Bringing web and physical together
% Thesis: demonstrate it's possible in a number of different domains
% 	Validate it's easier.
% 	Explain the design patterns.
% 
% WS-* should be a part of related work.
% REST: these concepts have been developed by others, we apply them to the real-world
%  

\begin{document}

\maketitle

\chapter{Introduction}

  \section{Real-World Service Integration Paradigms}
    \subsection{WS-*}
    \subsection{REST}
    \subsection{UpnP}
    \subsection{A Web-Service Integration Infrastructure for the Real-World}
    SOCRADES Integration Architecture~\cite{souza_socrades:web_2008,guinard_tsc}.  In~\cite{dominique_guinard_are_2009} we implemented Web proxies for Smart Meters.

  \section{Understanding Differences between Integration Models}
    \subsection{Qualitative}
    Show ease of use.
    \todo{Still to SHOW that it does foster and make it easier to build apps upon, for that use: Distributed Sys. lecture, OSS eval.}
    \subsection{Quantitative}
    Studies of~\cite{guinard_resource_2010} and~\cite{guinard_giving_2010}
    \subsection{Federative Architecture}
    We proposed a system federating both RESTful and WS-* services for things in~\cite{guinard_tsc}.

  \section{Empowering Users: Physical Mashups}
    \subsection{Web 2.0 Web Mashups}
    Explain Web Mashups.
    \subsection{Physical Mashups}
    Explain the concept introduced in~\cite{guinard_resource_2010}.

\chapter{A Web Oriented Architecture for a Composable Ecosystem of Smart Things}
  In~\cite{Guinard09b}, we introduced the Web of Things architecture which we further described in~\cite{guinard_resource_2010}.

    \section{From the Internet of Things to the Web of Things}
    \section{Designing RESTful Smart Things}
      \subsection{Smart Gateways}
      From this we derived the concept of Smart Gateways in~\cite{guinard_tsc} and~\cite{trifa_design_2009}.
      \subsection{Modeling Functionality as Linked Resources}
      \subsection{Representing Resources}
      \subsection{Servicing Through a Uniform Interface}
      \subsection{Syndicating Things}
      \subsection{Real Time Web of Things}
      \subsection{Finding and Describing Smart Things}
      Infra-WOT, Microformats, etc.
      \subsection{Sharing Smart Things}
      We developed a sharing and service advertising platform making use of social networks~\cite{guinard_sharing_2010} (SAC and FAT). 
      \todo{Evaluation of SAC/FAT?}
      \subsection{Design Automatization}
      Auto-WOT
      \todo{Include discovery and real-time}

\chapter{Wireless Sensors and Actuator Networks}
  RESTful Sun SPOTs~\cite{Guinard09}, RESTful Ploggs~\cite{dominique_guinard_are_2009,Guinard09b,weiss_emeter:interactive_2009}
  \subsection{Design}
  \subsection{Prototyping}
  \subsection{Home Mashups}
  We evaluated different approaches and proposed a Web-oriented prototype in~\cite{kovatsch_embedding_2010} and~\cite{guinard_mashing_2010}.
  Clickscript project, Energy Mobile Mashup, Physical Mashup Framework, etc.
  \todo{Better publication/eval?}
  \subsection{Evaluation}

 \chapter{Resource-Oriented Auto-ID Networks}
 RESTful EPCIS~\cite{guinard_giving_2010}
 \todo{More to come thanks to the project at Auto-ID}
  \subsection{Design}
  \subsection{Prototyping}
  \subsection{Evaluation}
  \subsection{RFID Mashups}
  Make it easy to build simple RFID apps.
  \todo{``Clickscript'' for RFID}

\chapter{Conclusion}


%\nocite{*} % Cite all uncited papers in the bib.
% References
% Max 1 page
\small
\bibliographystyle{splncs} %acm
\bibliography{../../phd}

\end{document}          
