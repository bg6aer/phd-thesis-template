\chapter{Introduction}
\minitoc

\section{Motivation}
%What is the problem.
\begin{figure}
\centering
\includegraphics[width=\linewidth]{figures/wot/smart-things}
\caption{Smart things are digitally enhanced objects and devices that have communication capabilities. They range from machines and home appliances to wireless sensor and actuator networks as well as tagged every-day objects.}
\label{fig:smartThings}
\end{figure}
Pervasive and ubiquitous computing have a long-lasting tradition of looking into integration of physical objects with the digital world. Recent developments in the field of embedded devices have led to \newterm{\sts{}} increasingly populating our daily life, slowly but steadily forming interconnected networks of physical objects: Sensor nodes are networked together to create environmental monitoring applications, making cities smarter and dynamically adapting to their context~\cite{Vasseur2010}. Home appliances such as TVs, alarm clocks, digital picture frames and Hi-Fi systems can communicate with each other to offer integrated services such as cross-devices multimedia experiences, smarter HVAC (Heating, Ventilating, and Air Conditioning) systems or more energy aware and efficient homes~\cite{Kovatsch2010,Helal2005,Vasseur2010,Mattern2005,Mattern2010a}. RFID-tagged objects in stores and along supply-chains allow manufacturers, suppliers and service providers to optimizes their operations~\cite{Bolic2010,Floerkemeier2010}. Products get digital identities through barcodes or RFID tags and offer unprecedented business opportunities~\cite{Floerkemeier2007,Floerkemeier2007a,Bolic2010,Mattern2005}. 

To facilitate these connections, research and industry have come up with a number of low-power physical and transport network protocols such as Zigbee, Bluetooth, IEEE 802.15.4 or, more recently, low-power WiFi and 6LoWPAN~\cite{Hui2008a,Hui2008}~\citeweb{ipso}. However, while these developments help towards integrating \sts{} at the network layer, at the application layer, embedded devices still form multiple, small, incompatible islands: Developing applications using them remains a challenging task that requires expert knowledge of each \sts{} platform~\cite{DeSouza2008,Riedel2010}. As a consequence, \sts{} remain hard to integrate into composite applications. 

Several service platforms propose a standardized integrated architecture to facilitate the cross-integration of \sts{}. Standards such as UPnP and DNLA, Jini or OSGi successfully address concerns such as service discovery and registration. However, these systems are not fully compatible with one another and their complexity and lack of well-known tools let them only reach a relatively small community of expert developers~\cite{DeDeugd2006}. Hence their direct usage for innovative applications (e.g., mobile or Web-based applications) has been rather limited to date.
 
In the world of enterprise computing, interoperability and loose-coupling at the business application layer is achieved using WS-* Web services~\cite{Alonso2010}. WS-* Web services, sometimes called \quote{Big Web services}~\cite{Pautasso2008}, are based on two main XML formalisms: WSDL and SOAP as well as a set of additional standards (WS-Addressing, WS-Security, WS-Discovery, etc.). With the goal of facilitating the integration of \sts{} with applications, several research initiatives look at adapting these services to the real-world~\cite{Priyantha2008,DeSouza2008,Jammes2005,Guinard2009-INSS,Riedel2010}. This research led to lighter forms of WS-* services targeted towards real-world applications such as DPWS (Device Profile for Web Services)~\cite{Jammes2005} or DNLA, both direct evolutions of UPnP.

The ultimate goal of these initiatives can be summarized as trying to \important{create a loosely-coupled ecosystem of services for \sts{}}. That is, a widely distributed platform in which the services provided by \sts{} can be easily composed to create new applications and use-cases. As shown in several research projects, the WS-* approach is an improvement over the proprietary protocols traditionally used in this field~\cite{DeSouza2008,Jammes2005,DeDeugd2006}, however the approach also has important shortcomings. First, WS-* standards are rather verbose and heavy in terms of required bandwidth, memory and CPU. This makes them challenging to implement on devices with limited resources~\cite{Yazar2009,Pautasso2008}. More importantly, despite their name, Web services actually use the Web as a transport layer and not as an application architecture~\cite{Pautasso2009} making them harder to use and integrate with the World Wide Web.

The Internet is a compelling example of a scalable global network of computers that interoperate across heterogeneous hardware and software platforms. On top of the Internet, the Web illustrates well how a set of relatively simple and open standards (e.g., HTTP, HTML, XML, JSON, etc.) can be used to build very flexible systems while preserving efficiency and scalability. The cross-integration and developments of composite applications on the Web, alongside with its ubiquitous availability across a broad range of devices (e.g., desktop computers, laptops, mobile phones, set-top boxes, gaming devices, etc.), make the Web an outstanding candidate for a universal integration platform~\cite{Guinard2010-WoT}.

Hence, as more and more devices are getting connected to the Internet, the next logical step is to use the World Wide Web and its associated technologies as a platform for \sts{}. In the \important{\WoTLong{} (\WoT{})}, we are considering \sts{} as first-class citizens of the Web and position the \WoT{} as a refinement of the Internet of Things (IoT)~\cite{Mattern2005,Mattern2010a,Sarma2001a} by integrating \sts{} not only into the Internet (the network), but into the Web (the application layer). 

To achieve this integration, we propose to reuse and adapt patterns commonly used for the Web. We embed Web servers~\cite{Hui2008,Duquennoy2009,Guinard2009} on \sts{} and apply the REST (Representational State Transfer) architectural style~\cite{Fielding2000,Richardson2007} to the physical world. The essence of REST is to focus on creating loosely coupled services on the Web so that they can be easily reused~\cite{Pautasso2009}. REST is actually core to the Web and uses URIs for encapsulating and identifying services on the Web. In its Web implementation it also uses HTTP as a true application protocol. It finally decouples services from their presentation and provides mechanisms for clients to select the best possible formats. This makes REST an ideal candidate architecture to build a \newterm{universal API} for \sts{}. As the \quote{client-pull} interaction model of HTTP does not fully match the needs of event-driven IoT applications, we further suggest the use of syndication techniques such as Atom and some of the recent real-time Web technologies to enable \sts{} push interactions.


In this thesis, we propose a \WoTA{}: a Web-based distributed application platform for \sts{}. As a consequence of the proposed architecture, \sts{} and their functionality get transportable URIs that one can exchange, reference on Web sites and bookmark. \stsBeg{} are also linked together enabling discovery simply by browsing. The interaction with \sts{} can also almost entirely happen in a browser, a tool that is ubiquitously available and that most users are familiar with~\cite{Kindberg2002}. Applications can be built upon them using well-known Web languages and technologies. Furthermore, \sts{} can benefit from the mechanisms that made the Web scalable and successful such as caching, load-balancing, indexing and searching.

In this thesis, rather than looking at one particular challenge, we take a holistic view, looking also at the bigger picture. We propose a number of building-blocks towards creating a distributed deployment of \sts{} which fosters serendipitous re-use of \sts{}. Our goal is to create a participatory \WoT{} where communities and users can create opportunistic applications, i.e., composite applications easily created by re-using existing services or devices. Just as people create Web mashups~\cite{Hartmann2008} involving Web 2.0 services, they should be able to create \important{\pMashups{}}~\cite{Guinard2010-WoT} mixing services from the real and virtual worlds together. 


\section{Contributions}
%what do we do to solve the problem
In this section we outline the main contributions of the thesis towards an Internet of Things supporting opportunistic applications or physical mashups. First, we present the Web of Things architecture. We then apply it systematically to two domains: Wireless Sensor and Actuator Networks and Auto-ID (Automatic Identification) networks.

\subsection{The Web of Things: A Web-Oriented Service Platform for \stsBig{}}
Our first contribution is the \WoTA{}. For this, we build on top of network connectivity and focus on the application layer. We take a systematic approach required to achieve the \newterm{mashability} of \sts{} with the Web. We identify four layers: \important{accessibility, findability, sharing and composition}. We study how each layer can be designed and implemented as a service on the Web using Web languages and patterns.

We begin by addressing \important{device accessibility}. We propose to use the REST  architectural style~\cite{Fielding2000} and study its applicability to \sts{}~\cite{Guinard2010-WoT}. For this part, we build upon several projects in the field of ubiquitous computing~\cite{Kindberg2002,Drytkiewicz2004-pREST,Luckenbach2005} and look at a systematic application of the REST principles and their current Web implementation in HTTP to adapt the architecture to the needs of \sts{}. We then discuss two ways of integrating \sts{} to the Internet and the Web. Either directly or through the use of enhanced reverse proxies that we call \sgs{}~\cite{Trifa2009,Guinard2010-Search}. As a result, \sts{} become easier to build upon. Popular Web languages (e.g.,~HTML, Python, JavaScript, PHP) can be used to easily build applications involving \sts{} and users can leverage well-known Web mechanisms (e.g.,~browsing, searching, bookmarking, caching, linking) to interact and share these devices. We illustrate this by means of a user study.

We then study the \important{findability}~\cite{Morville2005} of \sts{}: once they are connected to Web, how does one find the services they offer to integrate them into composite applications? In particular, we propose a discovery and lookup infrastructure for the \WoTLong{} and describe \sts{} according to a metadata model that we implement using semantic annotations based on Microformats~\citeweb{microformats}. Furthermore, we propose, implement and evaluate extending user search queries for \sts{} services based on a new process~\cite{GuinardPatent2010} using related keywords extracted from services on the Web (e.g., Wikipedia, Yahoo Web Search, etc.)~\cite{Guinard2010-Search}.

Using \sts{} in composite applications requires a scalable \important{sharing} mechanism that lets owners of \sts{} manage access control in a convenient and straightforward way. We introduce the \sacLong{}~\cite{Guinard2010-sharing}, a platform that relies on social networks (e.g., Facebook, LinkedIn, Twitter, etc.) and their open APIs (e.g., OpenSocial) to enable owners to leverage the social structures in place for sharing \sts{} with others.

Finally, we discuss the \important{composition} of \sts{} on the Web. We introduce the notion of \newterm{\pMashups{}} where services from the Web are serendipitously composed with services offered by \sts{}. We discuss a number of requirements towards a \pMashupsFw{} upon which \pMashup{} editors can be built and propose an implementation as an open service on the Web~\cite{Guinard2010-mashup-home}.

These four layers form the basis of our \WoTA{}. Not every Internet of Things application that is to be ported to the Web will require the four of them, but they present a model that can be used towards a looser-coupling, better and easier integration of the physical and the virtual worlds thanks to Web technologies.

\subsection{Case Studies}
Our second and third contributions are two domains in which we apply the \WoTA{} in three different case studies. We first look at the field of Wireless Sensor and Actuator Networks and then at Auto-ID networks.

\subsubsection{Bringing Wireless Sensors and Actuators Networks to the Web}
In the last decade, important progress in the field of embedded systems has given birth to a myriad of tiny computers to which virtually any type of sensors/actuators can be attached. By inter-connecting these devices using low-power wireless communication, a whole new world of possible applications is unveiled. Networks of physically distributed computers, usually called Wireless Sensor Networks (WSN), are valuable tools for monitoring the physical world~\cite{Vasseur2010}. Likewise the same type of devices can also be used for actuating the world, such as controlling security systems, traffic lights, etc. These networks are then called Wireless Sensor and Actuator Networks\footnote{We will subsequently use the term Wireless Sensor Network (WSN) for both, pure sensor and sensor/actuator networks.}

Unfortunately, most projects using WSNs are based on different -- and usually incompatible -- software and hardware platforms~\cite{DeSouza2008,Jammes2005,DeDeugd2006,Akyildiz2002}. Within such an heterogeneous ecosystem of devices, the development of simple applications still requires special skills and a substantial amount of time~\cite{Mottola2011}. Moreover, for each new deployment, a large amount of work must be devoted to re-implement basic functions and application-specific user interfaces, which is a waste of resources that could be used by developers to focus on the application logic. Ideally, developers should be able to quickly build applications only by recombining ready-made building-blocks.

Hence, the world of WSNs is an interesting case-study for the proposed approach as adopting a simple application architecture for these devices contributes to foster their wider usage and applicability. Our contribution here is to look at two WSN platforms and evaluate the validity of our approach by integrating them to the Web based on the \WoTA{}.

\paragraph{Facilitating General Purpose Sensing Applications}
We first look at an all purposes WSN platform called Sun SPOTs~\citeweb{sunspot}. We design, implement and evaluate how each layer of the model can be leveraged to make the developments on top of this platform as accessible as simple Web development is. With this platform we also evaluate the performances when using Web protocols directly on WSNs using embedded Web servers. 

\paragraph{Facilitating Energy Monitoring and Control Applications}
Rising global energy demand and the limitation of natural resources has led to increased thoughts on residential energy consumption. A necessary step towards energy conservation is to provide timely and fine-grained consumption information. This allows for users to identify energy saving opportunities and possibly adjust their behavior to conserve energy~\cite{Mattern2010}. 

Currently available off-the-shelf products that depict the energy consumption in near real-time are helpful, but do not fully meet the user needs as they have high usage barriers and often require complex installations~\cite{Froehlich2009a}. Furthermore, they are not able to provide the most compelling feedback~\cite{Fischer2008-Energy} since they lack the ability to provide an appliance-specific break down of the energy consumption and are not able to compare the consumption of individual devices in an appealing manner. Finally, they do not meet the needs of software developers as they do not offer open APIs, and developing applications on top of them is rather cumbersome.

Hence, we propose an \important{easy to use, deploy and develop upon} device-level energy monitoring platform called \newterm{Energie Visible}~\cite{Guinard2009a,Guinard2010-WoT,Weiss2010}. It is based on an off-the-shelf smart power outlet that we seamlessly integrate into the Web by systematically applying the \WoTA{}. We demonstrate how this facilitates, on the one hand-side, the deployment and usage by home users and, on the other hand-side, the development of novel applications for developers. We present the design and implementation of a Web user interface. We further evaluate the suitability of our approach with the help of a pilot deployment and feedback from several developers using our framework in research projects and prototypes such as a mobile phone energy monitoring application.


\subsubsection{Facilitating the Development and Deployment of Distributed Auto-ID Applications}
The RFID (Radio Frequency IDentification)~\cite{Sarma2001,Finkenzeller2010} standards community has developed a number of communication interfaces and software standards to provide interoperability across different RFID deployments. This extensive standards framework, known as the EPC (Electronic Product Code) Network~\cite{Sarma2001a}, covers aspects such as reader-to-tag communication, reader configuration and monitoring, tag identifier translation, filtering and aggregation of RFID data, and persistent storage of application events.  While there are in total fifteen standards that make up the EPC Network framework, the air interface protocol known as EPCglobal UHF Gen2 has seen the most adoption -- both in large scale supply chain applications as well as niche RFID deployments. 

The adoption of the software standards within the EPC Network has been significantly slower~\cite{Schmitt2008,Guinard2011}. The deployment of RFID applications that implement the EPC Network standards often remains complex and cost-intensive mostly because they typically involve the deployment of rather large and heterogeneous distributed systems. As a consequence, these systems are often only suitable for big corporations and large implementations and do not fit the limited resources of small to mid-size businesses and small scale applications both in terms of required skill-set and costs. 

While there is most likely no universally available solution to these problems, the success of the Web in bringing complex, distributed and heterogeneous systems together through the use of simple design patterns appears as a viable approach to address these challenges. Thus, our contribution in this context is to study the pain points of RFID applications and systematically apply the Web of Things model~\cite{Guinard2010d,Guinard2011}.

In particular, we show how Cloud Computing, RESTful interfaces and the real-time web as well as \pMashups{} can simplify application development, deployments and maintenance in a common RFID application. Our analysis also illustrates that RFID/EPC Network applications are an interesting fit for \WoT{} technologies and that further research in this field can significantly contribute to making real-world applications in this domain less complex and cost-intensive. 

\section{Thesis Outline}
The remainder of this thesis is structured as follow: \chapterRef{wot} presents the \WoTA{}. We expose the required components for a successful Web-integration of \sts{}. We further propose a number of optional components that help to create a Web ecosystem in which \pMashups{} are made possible. In particular, we discuss device accessibility, findability, sharing and composition. In this part of the thesis, we also review an alternative architecture and present a user-study in which we build upon the developers' experience to provide guidelines on making the right architectural decision for IoT projects. 

In \chapterRef{wsn} we apply the architecture to the domain of Wireless Sensor Networks. In the first part we discuss the integration of a general purpose sensing platform. We then present the benefits of the \WoTA{} when applied to smart energy monitoring and control. Based on our open-sourcing of the platform, we also discuss a pilot deployment and illustrate the ease of use and integration that the \WoTA{} provides.

In \chapterRef{autoid} we show how the architecture can be applied to the Auto-ID and RFID domain. We apply it to several components of the EPC Network and illustrate, by means of prototypes and studies, how the \WoTA{} has the potential to simplify and foster developments in the RFID domain.

Finally, in \chapterRef{conclusion} we provide an summary of our contributions. Taking a step back, we also discuss some of the open challenges and interesting future directions that we believe the Internet of Things and Web of Things domains will have to take.