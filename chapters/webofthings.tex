\chapter{The \WoTLong{}}
\label{wot}
\cquote{Any sufficiently advanced technology is indistinguishable from magic}{Arthur C. Clarke}
\minitoc

\begin{figure}
\centering 
 \imgLine{wot/wot-architecture-new}
\caption{The four layers of the Web of Things architecture: Accessibility, Findability, Sharing, Composition. Applications can be built on top of each layer, but as we go up the layers they become more accessible to a broader community of developers and users. The figure provides an overview of the deliverables of this thesis. On the left-side are the architectural building-blocks. On the right-side the applications and prototypes.}
\label{fig:wotArchi}
\end{figure}
In this chapter, we present the \WoTA{}. We describe the four layers it is based on: the \devLayer{}, \findLayer{}, \shareLayer{} and \compoLayer{} as shown in \figRef{wotArchi}.

The overall goal of this architecture is to facilitate the integration of \sts{} with existing services on the Web and to facilitate the creation of Web applications using \sts{}. In particular, we formulate the following general requirements~\cite{Guinard2010-Search,Guinard2011} we would like for our architecture to fulfill:
 \begin{enumerate}
 \item It should \important{lower the entry barrier for developers} and foster rapid prototyping. This allows a wider range of developers, tech-savvy users (technologically skilled people) or researchers to develop on top of \sts{} and contributes to fostering third party (public) innovation using \sts{}.
 
  \item It should \important{offer direct access for users}. Users should be able to access and use \sts{} without the need for installing additional software. From a Web browser (or an HTTP library in the case of a software client) they should further have means to directly extract, save and share \sts{} data and services. This ensures the usability of the architecture and minimizes the entry barriers for users.
 
 \item It should offer a \important{lightweight} access to \sts{} data. This enables creating applications in which real-world data are directly consumed by \important{resource-constrained devices} such as mobile phones or wireless sensor nodes without requiring dedicated software on these devices.
\end{enumerate} 

Unlike traditional layered architectures such as the OSI (Open Systems Interconnection) model~\cite{Day1983}, the layers in the presented \WoTA{} are not strictly defined and do not literally hide the previous layers. Rather, the architecture should be seen as an ecosystem of different services that ease, step by step, the creation of applications using \sts{}. The architecture proposes services that address each layer required to consider \sts{} as first-class citizen of the Web. However, applications can be built on top of the services offered by the implementation of each layer or on top of a combination of them depending on the requirements of a particular use-case.

As illustrated on \figRef{wotArchi} the development of applications using \sts{} on top of their native operating systems, protocols and libraries still requires specific skills and is, for the greater part, only accessible to embedded systems experts~\cite{Mottola2011}. The goal of each layer of the \WoTA{} is first to bring this development closer to Web developers and technically skilled hobbyists~\cite{Hartmann2008}. Then, it brings the usage and development of Internet of Things applications closer to end-users, enabling them to create simple applications tailored to their needs.

In this chapter, we describe each layer. Focusing first on the architecture of the components in these layers, we then look at the services and APIs they offer and propose implementations of these services. The proposed components are evaluated in a generic way in this chapter. These evaluations are complemented by \chapterRef{wsn} and \chapterRef{autoid}, where we apply the architecture to two specific domains: Wireless Sensor Networks and RFID tagged-objects and evaluate it within these domains.


% These are the actual sections of this chapter...
\section{\devLayer{}}\label{deviceLayer}
\begin{center}
\includegraphics[width=0.7\linewidth]{figures/wot/devLayer}
\end{center}
In the first layer of our \WoTA{} (see \figRef{wotArchi}), we address the access to \sts{}: \important{How can we, from an application point of view, enable a consistent access to all kinds of connected objects?}

Our proposal is to integrate things to the core of the Web, making them first-class citizens just as Web pages are. For this, we use the REST architectural style and its Web implementation. In the first part of this section, we illustrate how we can model the functionality and services of \sts{} using the RESTful principles. Then, we discuss the integration of devices that are not capable of connecting to the Internet. Finally, we propose a way for \sts{} to push data to clients rather than having them constantly polling data. Parts of this section have been published in~\cite{Guinard2011b,Guinard2010-WoT,Guinard2009-INSS,Guinard2009}.

\subsection{A Web API for \stsBig{}}\label{APIforSmartThings}
We begin by briefly summarizing the principles of RESTful architectures. We then focus on how a systematic application of the RESTful principles to \sts{} leads to an API that can be consumed and understood by a broad number of clients.
  
\subsubsection{REST in a Nutshell}\label{rest}
Initially proposed by Roy Fielding in his Ph.D. dissertation~\cite{Fielding2000}, REST is an architectural style that was used as a set of guidelines to implement the second wave of Web standards and in particular HTTP 1.1 and URIs (Uniform Resource Identifiers). The goal of this second wave of standards was to move from a Web serving documents as of HTTP 0.9 to a Web as a true application layer wit HTTP 1.1. The REST guidelines where created to make sure that the new architecture would support \quote{scalability of component interactions, generality of interfaces, the independent deployment of components as well as intermediary components to reduce interaction latency, enforce security, and encapsulate legacy systems}~\cite{Fielding2000}.

As such, REST is independent from the Web and can be implemented in other systems. However, in the remainder of this thesis we focus on the Web implementation of REST.

The central idea of REST revolves around the notion of resource as \important{any component of an application that needs to be used or addressed}. Resources can include physical objects (e.g., a temperature sensors) abstract concepts such as collections of objects, but also dynamic and transient concepts such as server-side state or transactions. 

A system can be basically considered as RESTful if it respects the five following constraints~\cite{Fielding2000}:

\begin{itemize}
\item[C1] \emph{Resource Identification:} the Web relies on \emph{Uniform Resource Identifiers (URI)} to identify resources, thus links to resources (C4) can be established using a well-known identification scheme.

\item[C2] \emph{Uniform Interface:} Resources should be available through a uniform interface with well-defined interaction semantics, as is \emph{Hypertext Transfer Protocol (HTTP)}. HTTP has a very small set of methods with different semantics (\emph{safe}, \emph{idempotent}, and others), which allows interactions to be effectively optimized. It also allows for a clean decoupling of the interface (RESTful interface) and the actual service implementation. Unlike in WS-*, where methods (also known as service operations) take arbitrary names and semantics, in HTTP, the uniform interface has 5 main methods:
\begin{enumerate}
\item \code{GET} is used to retrieve the representation of a resource.
\item \code{PUT} is used to update the state of an existing resource or to create a resource by providing its identifier.
\item \code{DELETE} is used to remove a resource.
\item \code{POST} creates a new resource.
\end{enumerate}

While these verbs definitely cover very well CRUD (Create Read Update Delete) types of applications, they are also supposed to explicit every action a client can execute on a resource, whatever the type of application is.

\item[C3] \emph{Self-Describing Messages:} Agreed-upon resource representation formats make it much easier for a decentralized system of clients and servers to interact without the need for individual negotiations. On the Web, media type support in HTTP and the \emph{Hypertext Markup Language (HTML)} allow peers to cooperate without individual agreements. For machine-oriented services, media types such as the \emph{Extensible Markup Language (XML)} and \emph{JavaScript Object Notation (JSON)} have gained widespread support across services and client platforms. JSON is a lightweight alternative to XML that is widely used in Web 2.0 applications and directly parsable to JavaScript objects.

\item[C4] \emph{Hypermedia Driving Application State (Connectedness):} Clients of RESTful services are supposed to follow links they find in resources to interact with services. This allows clients to \quote{explore} a service without the need for dedicated discovery formats, and it allows clients to use standardized identifiers (C1) and a well-defined media type discovery process (C3) for their exploration of services. This constraint must be backed by resource representations (C3) having well-defined ways in which they expose links that can be followed.

\item[C5] \emph{Stateless Interactions:} This requires requests from clients to be self-contained, in the sense that all information to serve the request must be part of the request. HTTP implements this constraint because it has no concept beyond the request/response interaction pattern; there is no concept of HTTP sessions or transactions. It is important to point out that there might very well be state involved in an interaction, either in the form of state information embedded in the request (HTTP cookies), or in the form of server-side state that is linked from within the request content (C3). Even though these two patterns introduce state into the service, the interaction itself is completely self-contained (does not depend on the context for interpretation) and thus is stateless.

\end{itemize}

Tying together C2 and C3, HTTP also supports \newterm{content negotiation}, allowing both clients and servers to communicate about the requested and provided representations for any given resource. Since content negotiation is built into the uniform interface, clients and servers have agreed-upon ways in which they can exchange information about available resource representations, and the negotiation allows clients and servers to choose the representation that is the best fit for a given scenario.

The seminal work on REST~\cite{Fielding2000} on which these guidelines are based, presents REST as a meta-architecture~\cite{Richardson2007}. This offers the advantage of being able to use the thesis as a set of tools for judging how RESTful systems are but it lacks practical guidelines on how to actually implement a RESTful system on the Web. In~\cite{Richardson2007}, Leonard Richardson and Sam Ruby propose the concept of ROA or Resource Oriented Architectures, which is a Web architecture that can be used to create loosely-coupled services on the Web.

The design goals of ROAs and their advantages for a decentralized and large-scale service architectures align well the field of pervasive computing: millions to billions of available resources and loosely coupled clients, with potentially millions of concurrent interactions with one service provider. Based on these observations, we argue that RESTful architectures are the most effective solution for the Web of Things, as they scale better and are more robust than RPC-based architectures such as WS-* Web services.

In the next sections we illustrate how the 5 constraints of REST as well as the concept of Resource Oriented Architectures can be applied an adapted to fit the requirements of a global and distributed ecosystem of smart things offering a comprehensive and interoperable service layer.

\subsubsection{RESTful Things: A Resource Oriented Architecture for Things}\label{ROA-for-things}
\begin{figure}
\imgHalf{wot/sensor-abstract}
\caption{An generic sensor node, offering a number of sensors and actuators.}
\label{fig:abstractSensor}
\end{figure}
The \WoT{} can be realized by applying principles of Web architecture, so that real-world objects and embedded devices can blend seamlessly into the Web. Instead of using the Web as a transport infrastructure we aim at making devices an integral part of the Web and its infrastructure and tools by using HTTP as an application layer protocol. In this section, we describe the use of REST as a universal interaction architecture, so that interactions with smart things can be built around universally supported methods. We describe the process of Web-enabling smart things into four main steps:
\begin{enumerate}
 \item Resource Design: identify the functionality or services of a \st{}, organize the hierarchy of these services and link them together, fulfilling constraints C1 and C4.
 \item Representation Design: decide which representations will be served for each service, fulfilling constraint C3.
 \item Interface Design: decide on the actions allowed for each service, fulfilling constraint C2 and C5.
 \item Implementation Strategy: choose a strategy to integrate the \sts{} to the Internet and the Web, either directly or through a \sg{}.
\end{enumerate}

In the following, we provide a set of guidelines to Web-enable smart things based on these four main steps. As case study, we describe how it can be used to bring wireless sensor nodes to the World Wide Web. The abstract sensor node we use as an illustration is shown in \figRef{abstractSensor}.


\subsubsection{Resource Design: Modeling Functionality as Linked Resources}
As mentioned before, the central idea of REST revolves around the notion of resources. In our context, a resource is any component of an application that is worth being uniquely identified and linked to. On the Web, the identification of resources relies on Uniform Resource Identifiers (URIs), and representations retrieved through resource interactions contain links to other resources, so that applications can follow links through an interconnected web of resources. Clients of RESTful services are supposed to follow these links, just like one browses Web pages, in order to find resources to interact with. This allows clients to \important{explore} a service simply by browsing it, and in many cases, services will use a variety of link types to establish different relationships between resources.

\paragraph{Resource Identification}In the Web of Things we have several levels of resources. While some of them represent physical objects, others are virtual only. Resources on the Web are often organized in a hierarchy, the hierarchical way of organizing and linking resources is also very relevant in the physical world and can be used as a basis to identify the resources of a smart thing.
\begin{figure}
\centering
\includegraphics[width=\linewidth]{figures/wot/sensor-hierarchy}
\caption{An example of resource hierarchy deduced from the abstract sensor. This hierarchy forms a tree where each resource has $0..n$ child resources.}
\label{fig:sensorHierarchy}
\end{figure}
As an example, from the abstract sensor node in \figRef{abstractSensor} we can extract resources as shown in \figRef{sensorHierarchy}. From this hierarchy we understand that each node has sensors (light, temperature, etc.), actuators (speakers, LEDs, etc.). Each of these components is modeled as a resource and assigned a URI which is deduced from the name of the current resource and its predecessors in the hierarchy. For instance, the light sensor gets the URI:
\RESTURL{/generic-nodes/1/sensors/light}.

In an HTTP context, these identifiers or URIs are also known as URLs. However, since the term URL has been officially deprecated we use URI in the remainder of this thesis. Nevertheless, the widespread use of the term URL lead to a contemporary definition in which \quote{a URL is a type of URI that identifies a resource via a representation of its primary access mechanism (e.g., its network location), rather than by some other attributes it may have}~\citeweb{uris}. On the Web a URL is a URI beginning with the \code{http:} scheme and resolvable through the HTTP protocol.

We can form the absolute URI (or URL) of a smart thing's resource by adding a protocol scheme and root domain to the identifier. An important and powerful consequence of this is the addressability and portability of resource identifiers: They become unique (Internet or Intranet-wide, depending on the domain-name or assigned IP address) and can be resolved by any HTTP library or tool (e.g., a browser), bookmarked, exchanged in emails, instant messaging tools, encoded in QR-codes (Quick Response), etc.

For instance, typing a URI such as:\\
\RESTURL{http://<DOMAIN>:<PORT>/generic-nodes/1/sensors/light}\\ 
in a browser requests a representation of the resource light of the resource sensors of generic-node number 1.

Since there are no rules on the semantics of resources' identifiers, we cannot deduce a strict rule for naming physical resources. However, we suggest naming them according to two simple guidelines:
\begin{enumerate}
 \item Use descriptive names: as the resource names appear in the URIs using names with some semantic value can be of great help to developers and users.
 \item Use the plural form for aggregate resources: for instance if a smart thing has several sensors, then there should be a parent resource called \RESTURLInLine{sensors} from which every sensor is accessible with hyperlinks. 
\end{enumerate}

\paragraph{Linking}A resource should also provides links back to its parent and forward to its children as well as to any related resource. As an example, the resource \RESTURL{generic-nodes/1/sensors/}
provides a list of links to all the sensors offered by generic-node 1. This interlinking of resources that is established through both, resource links and hierarchical URI, is not strictly necessary, but well-designed URIs make it easier for developers and users to \quote{understand} resource relationship and even allow non-link based ad-hoc interactions, such as hacking a URI by removing some structure and still expecting for it to work somehow. In some browsers this URI hacking is even part of the UI, where a \quote{go up} function in the browser simply removes anything behind the last slash character in the current URI and expects that the Web site will serve a useful representation at that guessed URI.

Links are very important in Resource Oriented Architectures since they help clients to \important{discover} related resources. Using these links the client can discover other related services, either by browsing in the case of a human client or by crawling in the case of a machine. Thus, links in resource oriented architectures fulfill the constraint (C4) and enable the dynamic discovery of resources.

However, as we will see below, resources are not bound to a particular format but can be served using several formats. When a client requests an HTML representation then representing links is very straightforward as HTML has a standard mechanism for specifying links. With other formats such as JSON, however, there is no single standard format for providing links.

One could argue that the client can always fall back to an HTML representation when it is interested in related resources. However, this is inefficient in terms of HTTP calls required for a request which is especially relevant in resource-constrained environments such as the Web of Things. A good practice is thus to embed links consistently across all provided representations. One shortcoming of this approach is that the lack of standard link representation in formats such as JSON leads to a tighter coupling between the client and provided services.


\subsubsection{Representation Design: Formatting the Resources}
\begin{figure}
\imgHalf{wot/spot-html-rep}
\caption{HTML representation (as rendered by a Web browser) of the temperature resource of a sensor node containing links to parent and related resources.}
\label{fig:genericNodeHTML}
\end{figure}
Resources are abstract entities and are not bound to any particular representation. Thus, several formats can be used to represent a service of a \st{}. However, agreed-upon resource representation formats make it much easier for a decentralized system of clients and servers to interact without the need for individual negotiations. 

On the Web, media type support in HTTP and the Hypertext Markup Language (HTML) allow peers to cooperate without individual agreements. It further allows clients to navigate amongst the resources using hyperlinks. For machine-to-machine communication, other media types, such as XML and JSON have gained widespread support across services and client platforms.


In the case of smart things, we suggest support for at least an HTML representation to ensure browsability by humans. Note that since HTML is a rather verbose format, it might not be directly served by the things themselves, but by intermediate reverse proxies, called \sgs{} and described in \sectRef{gateways}. 

For machine-to-machine communications, we suggest using JSON. Since JSON is a more lightweight format compared to XML, both is terms of message size and parsing time~\cite{Xu2009}, it is better adapted to devices with limited capabilities such as \sts{}. Furthermore, it can directly be parsed to JavaScript objects. This makes it an ideal candidate for integration into Web Mashups and thus for creating physical mashups (see \sectRef{physicalMashups}).

In the example of our generic-sensor, each resource provides both, an HTML and a JSON representation. As an example, \lstRef{JSONRepresentation} shows the JSON representation of the temperature resource and \figRef{genericNodeHTML} shows the same resource represented as an HTML page with links to parents, subresources, and related resources.


\begin{lstlisting}[caption=JSON representation of the temperature resource of a generic node, label=lst:JSONRepresentation, breaklines, numbers=left, numberstyle=\tiny, language={}, xleftmargin=0.8cm, basicstyle=\small\ttfamily, backgroundcolor=\color{gray}, captionpos=b]
{"resource":
  {"methods":["GET"],
  "name":"Temperature",
  "links":["/feed", "/rules"],
  "content":
  [{"description":"Current Temperature",
  "name":"Current Ambient Temperature",
  "value":"24.0",
  "unit": "celsius"}]}
}
\end{lstlisting}

\subsubsection{Interface Design: Servicing Through a Uniform Interface}\label{interfaceDesign}
In REST, interacting with resources and retrieving their representations all happens through a uniform interface which specifies a service contract between the clients and servers. The uniform interface is based on the identification of resources, and in case of the Web, this interface is defined by the HTTP protocol. We focus on three particular parts of this interface that can be used to model a smart thing's API: operations, content-negotiation, and status codes.

\paragraph{Operations on Resources}
As mentioned before, HTTP provides five main methods to interact with resources, often also referred to as \newterm{verbs}. Constraining operations to these methods is one of the keys to enable loose-coupling of services, as clients only need to support mechanisms to handle these methods~\cite{Pautasso2009}.

In the Web of Things, these operations map rather naturally, since smart things usually offer quite simple and atomic services. As an example:
\begin{itemize}
 \item \code{GET} can be used to retrieve the current consumption of a smart meter.
 \item \code{PUT} can be used to turn an LED on or off.
 \item \code{POST} can be used to create a new feed used to trace the location of an RFID tagged object.
 \item \code{DELETE} can for example be used to delete a threshold on a sensor or to shutdown a device.
\end{itemize}

More concretely, as an example, a \code{GET} on
\RESTURL{/generic-nodes/1/sensors/temperature}
returns the temperature observed by node 1, i.e., it retrieves the current representation of the temperature resource. A \code{PUT} on
\RESTURL{/generic-nodes/1/actuators/leds/1}
with the updated JSON representation \{''status'':''on''\} (which was first retrieved with a \code{GET} on \code{/leds/1}) switches on the first LED of the node, i.e., it updates the state of the LED resource. A \code{POST} on
\RESTURL{/generic-nodes/1/temperature/rules} 
with a JSON representation of the rule as \{``threshold``:35\} encapsulated in the HTTP body, creates a rule that will notify the caller whenever the temperature is higher than 35 degrees, i.e., it creates a new rule resource without explicitly providing an identifier. Finally, a \code{DELETE} on
\RESTURL{/generic-nodes/1} is used to shutdown the node, or a \code{DELETE} on 
\RESTURL{/generic-nodes/1/sensors/temperature/rules/1} is used to remove rule number 1.

Additionally, another less-known verb is specified in HTTP and implemented by most Web servers: \code{OPTIONS} can be used to retrieve the operations that are allowed on a resource as well as metadata about invocations on this resource. In a programmable Web of Things, this feature is very useful, since it allows applications to find out at runtime what operations are allowed for any URI. As an example, an \code{OPTIONS} request on \RESTURL{/generic-nodes/1/sensors/humidity/rules}
returns \code{GET}, \code{POST}, \code{OPTIONS} as shown in the full HTTP response in \lstRef{HTTPOptions}.

\begin{lstlisting}[caption=HTTP response of an \code{OPTIONS} request on a resource. It informs the client about the operations (\code{GET} and \code{POST}) available for the resource., label=lst:HTTPOptions, breaklines, numbers=left, numberstyle=\tiny, language={}, xleftmargin=0.8cm, basicstyle=\small\ttfamily, backgroundcolor=\color{gray}, captionpos=b]
HTTP/1.1 200 The request has succeeded
Content-Length: 0
Allow: GET, POST, OPTIONS
Date: Tue, 19 Apr 2011 12:17:42 GMT
Accept-Ranges: bytes
Server: Noelios-Restlet-Engine/1.1.7
Connection: close
\end{lstlisting}

\paragraph{Content Negotiation}
Since resources are representation agnostic there is a need for clients and servers to be able to negotiate the right format for the right purpose. As a consequence, HTTP specifies a mechanism for clients and servers to communicate about the requested and provided representations for any given resource; this mechanism is called content negotiation. Since content negotiation is built into the uniform interface of HTTP, clients and servers have agreed-upon ways in which they can exchange information about requested and available resource representations, and
the negotiation allows clients and servers to choose the best representation for a given scenario.
For the abstract-node, a content negotiation message exchange looks as follows. The client begins with a \code{GET} request on
\RESTURL{/generic-nodes/1/temperature/rules}.
It also sets the \code{Accept} header of the HTTP request to a weighted list of media types it understands, for example to: \quote{\code{application/json;q=1, application/xml;q=0.5}}. The server then tries to serve the best possible format it knows about and specifies
it in the \code{Content-Type} of the HTTP response. In our case, the generic-node cannot offer XML and would thus return a JSON representation and set the HTTP header to \code{Content-Type: application/json}.

While this is the standard way of negotiating a representation in HTTP, it has two drawbacks when implemented. First, it is unfortunately not implemented evenly by all the Web servers. More importantly, it encapsulates the required format in the HTTP packet directly and does not expose it to the users.  Since the required format is a key parameter, we suggest supporting content negotiation directly in the URI as well in order to make it more natural for everyday users as well as directly testable and bookmarkable. 

Thus, requests such as \RESTURL{/generic-nodes/1/sensors/temperature.json} should be supported as well and should return the temperature resource in the JSON format as shown in \lstRef{JSONRepresentation}. In case the smart thing does not accept this format it should return the closest possible format (e.g., XML in this case). Furthermore, it should set the appropriate response header: \code{Content-Type: application/json} just as with standard content negotiation. 

\paragraph{Status Codes}
HTTP also offers a way of expressing errors and exceptions. Indeed, the status of an HTTP response is represented by standardized status codes sent back as part of the header in the HTTP response message. There exist several dozens of codes which each have well-known meanings for HTTP clients, these codes and their meanings are listed in the specification of HTTP 1.1~\cite{HTTP1999}. Furthermore, in~\cite{Richardson2007} these codes are analyzed and explained in the context of ROAs with valuable examples.

In the Web of Things, these codes a very valuable since they provide a lightweight but yet powerful way of notifying abnormal and successful request execution. 

As an example, a \code{POST} request on \RESTURL{/generic-nodes/1/sensors/humidity} returns a \code{405} status code. The client understands the status code as the notification that \quote{the method specified in the request is not allowed for the resource identified by the request URI}.

A concrete example of mapping domain-specific exceptions to Status Codes is provided in ~\chapterRef{EPCISWebadapter} where RFID exceptions are mapped to HTTP status codes.

\subsection{Implementation Strategy: Connecting Things to the Internet}\label{gateways}
\begin{figure}
\imgLine{wot/smart-gateway-new}
\caption{Web and Internet integration with \sgs{} (left), direct integration (right). The \sgs{} are small software application servers containing: Device Drivers to understand the low-level \sts{}, core services to create Web-APIs, pluggable services to offer additional functionality and a Web server.}
\label{fig:smartGateways}
\end{figure}
For a device to be part of the Web of Things, there are two basic requirements:
\begin{enumerate}
 \item Implementation of the TCP/IP protocols ideally over a IEEE 802 (Ethernet) or IEEE 802.11 (WiFi) network.
 \item Implementation of a Web server supporting the HTTP 1.1 protocol.
\end{enumerate}
While an increasing number of embedded devices are supporting these two requirements natively, not all of them do, mainly because their computational, memory and communication bandwidth are too limited. Hence, in this section we propose two alternatives to integrate smart things to the Internet and the Web.

\subsubsection{Native Internet and Web support}
Research has shown that TCP/IP stacks can be implemented to meet the constraints of embedded devices. In~\cite{Dunkels2003} Dunkels implemented a full TCP/IP stack for 8 bits embedded devices with a footprint in the order of 10 kilobytes. More recent developments worked on adapting the IPv6 protocol to meet the energy constraints of these devices and proposed the 6LoWPAN~\cite{Hui2008} architecture. Similarly, an increasing number of sensor nodes and embedded devices are equipped with native low-power WiFi support (over IEEE 802.11) modules and embedded HTTP servers. This makes them seamlessly integrated to the Internet. 

Previous work has also shown that embedding Web servers on resource and energy constrained devices is feasible~\cite{Duquennoy2009,Lin2004,Guinard2009-INSS}. Hence, it is a reasonable assumption that smart things will increasingly understand and implement the TCP/IP and HTTP 1.1 protocols. As an example, the off-the-shelve FlyPort shown in \figRef{flyport} is an embedded device from the Openpicus open-source project~\citeweb{flyport}. It features a low-power WiFi module with full TCP/IP support and a Web server implementing HTTP 1.1. Similarly, the RN-131 nodes from Roving Networks~\citeweb{roving} have TCP/IP over IEEE 802.11 connectivity.
\begin{figure}
\imgMedium{wot/flyport}
\caption{The Flyport embedded device offers a low-power WiFi module, full TCP/IP support and a native Web server supporting HTTP 1.1. (Reproduced with the kind authorization of OpenPicus, \url{www.openpicus.com})}
\label{fig:flyport}
\end{figure}
With world-wide consortia of industrial key-players appearing such as the IPSO Alliance~\citeweb{ipso}, it is very likely that most of the future devices will have all the required elements with no need to translate HTTP requests from Web clients into the appropriate protocol for the different devices, as shown in the right part of \figRef{smartGateways}.

For these types of smart things to be truly part of the Web, their functionality should be available through a RESTful interface, i.e., they should implement the \devLayer{} of the presented \WoTA{}.

\subsubsection{Reverse Proxies: \sgs{}}\label{smartGateways}
However, not all things can fulfill the requirements for TCP/IP and HTTP support. Indeed, for a number of \sts{}, these protocols are too demanding in terms of computation, memory, required bandwidth or battery life. As an example, it will probably take years until RFID tags will be powerful enough to implement these protocols and even then, it is unlikely for tags to communicate directly over IEEE 802.11. Similarly, for some sensor networks, ultra-optimized communication is a requirement and in these terms, dedicated low-power protocols such as Zigbee (over IEEE 802.15.4) or Bluetooth (over IEEE 802.15.1) or Ultra-Wideband
(UWB, over IEEE 802.15.3)~\cite{Lee2007} with dedicated transport protocols are a better choice, even if native TCP/IP is increasingly being supported on some of these platforms, e.g., for IEEE 802.11 through 6LoWPAN~\cite{Hui2008}.

Hence, when native TCP/IP and HTTP support is not possible or not desirable, we suggest that Web integration takes place using a software bridge. On the Web, similar bridges are called \newterm{reverse-proxies}. A reverse proxy takes requests from the Internet and forwards them to servers in an internal network. Reverse proxies have various interesting features, first they basically hide the internal network to the clients on the Internet. As a consequence, they can operate on the requests before they actually reach the services and are used for caching and load-balancing in several service oriented architectures.

For the Web of Things, we suggest taking a similar approach and propose the concept of \newterm{\sgs{}}~\cite{Trifa2009,Guinard2009,Guinard2010-WoT,Guinard2010-Search} to capture the fact that it is an application level component that does more than only data forwarding. A \st{} basically hides the (proprietary) low-level protocols that smart things natively use and make them available on the Internet through TCP/IP support and on the Web through a Web server. From the Web clients' perspective, the actual Web-enabling process is fully transparent, as interactions are based on HTTP in both cases.


\paragraph{System Architecture}
As shown in the left-most part of \figRef{smartGateways}, a \sg{} is a software component composed of three basic layers. First, core to the concept of \sgs{} are \newterm{Device Drivers}. Indeed, a \st{} can support several types of devices through a driver architecture as shown in \figRef{smartGateways} where the gateway supports three types of devices and their corresponding communication protocols. To maximize re-usability, a Device Driver should be composed of two different software components: a \newterm{Transport Driver} and an \newterm{Application Driver}. The Transport Drivers are responsible for providing an API to communicate through a particular protocol such as IEEE 802.15.4 or Bluetooth. On top of these, the Application Drivers are responsible for the sometimes proprietary service protocols of devices. As an example, a Device Driver for an energy metering sensor node, would be composed of a Bluetooth Transport Driver (that can be reused for other devices) and an Application Driver that understands the proprietary service or application protocol of the node.

Application drivers are in charge of mapping the functionality of a device to a RESTful API. For this, they use a \newterm{REST Application Framework} which provides methods for binding URIs to functionalities of the proxied smart things and formatting the responses using Web representations such as JSON or HTML.  

Closely bound with the REST Application Framework, the \newterm{Embedded Web Server} serves the service requests through the HTTP protocol. Ideally, it should also feature an Atom-server (or at least Atom representations) and be a non-blocking Web server with support for HTML5 WebSockets (see \sectRef{push}).

Through these components, clients can use HTTP for requesting services on non-IP and non-Web devices. As an example, consider a request to an energy sensor node coming from the Web through the RESTful API to:
\RESTURL{/energy-nodes/living-room/consumption.json}.
The request is captured by the Embedded Web Server of the \sg{} unmarshalled into an object and further sent to the method previously bound to \RESTURLInLine{/living-room/consumption}. This method is located in the Device Driver representing this particular energy sensor node. The Device Driver then translates the request in the appropriate format (e.g., a Bluetooth service call) to the Transport driver. The response is then transmitted back to the Device Driver which uses the REST Application Framework to marshal it into a Web format and further transmit it back to the embedded Web server. This process is summarized in \figRef{gateway-seq}.
\begin{figure}
\imgLine{wot/smartgateway-seq}
\caption{Simplified sequence diagram of the interaction between a client and a \st{} through a \sg{}. The \sg{} framework delegates the invocation to the corresponding Device Driver and takes care of converting the results into the appropriate format (e.g., JSON) and wrapping them into an HTTP packet returned to the client.}
\label{fig:gateway-seq}
\end{figure}


\paragraph{Software Implementation}
We implemented several \sgs{}. Our first implementation was based on a small foot-print C++ software that was used to Web-enabled sensor nodes capable of measuring electricity consumption~\cite{Guinard2009} (see also \sectRef{smartMetering}). Based on these implementations we realized that a lot of the written code could be re-used to Web-enable other \sts{}. We discuss two important points to foster the rapid integration of new devices and functionality to the \sgs{}: 

\subparagraph{Modular Device/Application Drivers}
First, modular Device and Application Drivers prevents \sts{} integration from becoming too complex and fosters re-using standard features (e.g., Bluetooth communication or binding a URI to a method). To ensure a high degree of modularity of these drivers, we implemented them using Java and in particular the OSGi framework~\citeweb{osgi}. OSGi is a modularization system built on top of Java that fosters re-usability through \newterm{bundles}~\cite{Hall2011}. Particularly interesting, is the concept of \newterm{Declarative Services} which facilitate the integration of different bundles. One of the big advantages of these service declarations is the ability to load a new (unknown) Bundle at run-time and having other components directly using it.

As a consequence, Device Drivers can be injected in a running \sg{}. This enables for instance the dynamic and remote injection of drivers to support new types of devices in an ad-hoc manner. Alternatively, the concept of Java Enterprise Application Servers~\cite{Goncalves2010}, such as for example the Glassfish Application Server~\citeweb{glassfish} can be used to create component-based \sgs{}. Indeed, the latest generations of these applications servers has become more lightweight and components can be injected in a managed run-time environment that features a Web server.

\subparagraph{Automatic Generation of Web Boilerplate Code} Furthermore, we realized that a lot of the code necessary to generate these Drivers (and OSGi) bundles and the mappings from Web resources to methods in Java code could be automated. Hence, in the \newterm{AutoWoT} project~\cite{Mayer2011} we propose a toolkit that enables developers to easily create new Device Drivers compatible with our \sg{} architecture. The toolkit was open-sourced and is available online~\citeweb{autowot}.

An editor let's developers specify the resources of a \st{} in a visual manner. The editor then generates an XML description of the resource tree that is used to generate the interfaces and OSGi-specific code to create a driver. Then, all the developer has to do is to fill the callback methods triggered whenever a Web resource is invoked, with the \sts{} specific code, e.g., implementing the \code{doGetTemperature()} method called when a client invokes a \code{GET} request on \RESTURLInLine{/temperature} in order for it to get the temperature data from the sensor node.

\subparagraph{Deployment}
Ideally \sgs{} should have a small memory footprint to be integrated into embedded computers already present in the infrastructure. For instance, when used to provide access to \sts{} in a home or office environment, the \sgs{} can be deployed on devices such as Wireless routers or Network Attached Storage (NAS). Our implementation of the \sg{} OSGi software was tested successfully on MicroClients SR from Norhtec~\citeweb{norhtec} featuring 500Mhz CPUs and 512 Mo of RAM each as well as an Asus WL-500gP Wireless router running the OpenWRT embedded Linux distribution~\citeweb{openwrt}.

\subsection{Pushing Data from \stsBig{} and \sgs{}}\label{push}
\begin{figure}
\imgLine{/wot/client-pull-sensor-push}
\caption{Sequence diagram of the communication between a client and a \st{}. On the left a traditional HTTP client-pull communication is started. The client has to constantly pull the \sts{} for updates. On the right a real-time Web approach is taken where the client is informed about the changes by the \sts{}.}
\label{fig:clientServer}
\end{figure}
HTTP was designed as a client-server architecture, where clients can explicitly request (pull) data and receive it as a response. This makes REST and HTTP \important{well suited for controlling} \sts{}, but this client-initiated interaction models seems unsuited for event-based systems, where data must be sent asynchronously to the clients as soon as it is produced.

This type of interaction is not really natural for some \sts{} applications and especially for \important{monitoring} applications~\cite{Duquennoy2009a,Trifa2010}. Consider for instance a sensor node used to detect a fire condition. As shown in \figRef{clientServer}, in the protocol proposed by HTTP 1.1, the client constantly has to request updates (a.k.a. \newterm{polling}). With this protocol, in the best case most requests end up with empty responses (\code{304 Not Modified}) as the temperature did not change. In the worst case, the server (i.e., \st{}) transmits the same data after each request. This is sub-optimal for two reasons: First, it generates a great number of HTTP calls and a great part of these calls are void. Since reducing the number of HTTP calls to the minimum is key in scaling Web sites~\cite{Souders2007}, this model raises scalability issues when considering monitoring applications in which several clients are connecting to a \st{}. Beyond scalability, numerous HTTP calls have more important consequences in the case of \sts{} such as their relatively high energy consumption which is important in the case of embedded devices running on batteries~\cite{Yazar2009}. 

Furthermore, although near real-time can be simulated by polling the \sts{} very regularly, a protocol in which the \sts{} could push asynchronously to the client as soon as a condition is met enables providing real-time information in the \WoT{}. In this section we discuss three architectural enhancements that contribute to solving these issues, while making sure that the proposed mechanisms can be integrated with the Web.

\subsubsection{Feeds of \sts{}}
\begin{lstlisting}[caption=Example of usage of the Atom format for providing historical information about a \st{}., label=lst:feedSmartThing, breaklines, numbers=left, numberstyle=\tiny, xleftmargin=0.8cm, basicstyle=\small\ttfamily, backgroundcolor=\color{gray}, captionpos=b]
<feed xmlns="http://www.w3.org/2005/Atom">
  <title type="text">[Title of Aggregation]</title>
    <author><name>[Name of the SmartGateway]</name></author>
    <link href="[Parent]" rel="related" type="[Type of representation]"/>
    <link href="[Child1]" rel="related" type="[Type of representation]"/>
    <link href="[Child2]" rel="related" type="[Type of representation]"/>
    <id>[uuid]</id>
    <entry>
      <title>[Content Description]</title>
      <id>[Unique ID for this event]</id>
      <author><name>[ID of the Smart Thing]</name></author>
      <uri>[Root URI of the Smart Thing]</uri>
      <published>[Date and time of event]</published>
      <content type="[text|html|xml|...]"</content>
    </entry>
    <entry>
      [...]		
    </entry>
</feed>
\end{lstlisting}

With Atom\footurl{tools.ietf.org/html/rfc4287}, the Web has a standardized and RESTful model for interacting with collections, and the \newterm{Atom Publishing Protocol (AtomPub)} extends Atom's read-only interactions with methods for write access to collections. Because Atom is RESTful, interactions with Atom feeds can be based on simple \code{GET} operations which can then be cached.
While initially created to aggregate content on the Web, feeds have two interesting features for the \WoT{}: First, they allow to create aggregates of \sts{} monitoring information. As an example a feed could be created to aggregate all information about energy sensors in a particular location (e.g., a building). Then feeds contain historical information and can thus be used to get not only the latest value of a sensor but rather its values over a period of time. Event more advanced scenarios can be based on feeds supporting query features, but this is an active area of research and there are not yet any standards~\cite{Wilde2009}. 

Using feeds to contain data provided by \sts{} is rather straightforward as shown in \lstRef{feedSmartThing} and thus feeds can be supported (through content-negotiation) as a representation for any resource or aggregate of resources of \sts{}.

More importantly, in the case of the \WoT{}, feeds can also be used to decouple clients from the actual resources. Indeed, the task of creating feeds can be delegated to completely external AtomPub compliant servers. As an example, it can be outsourced to the \sgs{} we introduced before.

However, using Atom feeds as a representation format does not provide a solution to the fact that Web clients have to poll the data. Instead of polling it directly from the sensors, they now have to poll Atom servers. It is worth noting, that new mechanisms such as Pubsubhubbub (PuSH)~\citeweb{pubsubhubbub} propose protocols to push feed updates back to the clients. However, these protocols requires additional infrastructure nodes (called hubs) and additional libraries on the client-side.



\subsubsection{HTTP Callbacks (Web Hooks)}

The most straightforward way to allow pushing to clients on the Web is to transform them into servers. This technique is often referred to as \newterm{Web Hooks} or \newterm{HTTP Callback} and is a very simple mechanism that can be used to have \sts{} pushing information to Web clients.

First, the client has to \newterm{subscribe} to a resource, for instance the energy consumption resource of a smart meter, it does so by \code{POST}ing a message to the \RESTURLInLine{/subscribe} resource of the smart meter, alongside with a callback URI and usually a threshold (e.g., $> 50$ Watts). As a result, the \sts{} will \code{POST} data to the Web client whenever the threshold is met.

However, a very important issue with such a mechanism is that it places a rather hard constraint since every client also has to become an HTTP server. This constraint prevents clients such as Web browsers to interact with HTTP Callbacks directly unless some additional libraries or plugins are used. Furthermore, when clients are behind (corporate) firewalls traffic coming from the \sts{} through callbacks will very often be blocked.


\subsubsection{WebSockets for the Real-time \WoT{}}\label{real-time-and-tpusher}
As a consequence of this constraint, several techniques appeared in order for servers to push data back to clients without having clients explicitly requesting it. Since browsers were not designed with server-sent events in mind, Web application developers have tried to work around several specification loopholes often referred to a \newterm{Comet} techniques~\citeweb{comet}. Comet is an umbrella term for most work-around, two of which are used quite often in practice: \newterm{Long Polling} and \newterm{Streaming}.

In the first technique, \newterm{Long Polling}, the client issues a request that will end only when the server is ready to send some data. Directly after the response the client will re-issue a request and so forth. In the second technique, \newterm{Streaming}, the client issues a request and the server never signals the end of this request, instead it keeps sending data over the TCP connection. In the absence of events, some servers will regularly send dummy data to prevent the connection from being closed.
 
While these work-around are used in practice they have two drawbacks: First, they generate unnecessary traffic~\cite{Lubbers2010}. More importantly, they are extremely resource demanding for the vast majority of Web servers. Indeed, most currently deployed Web servers allocate one thread or process for each connected client. Unlike in traditional HTTP requests, Comet requests do not end and thus quickly overload the memory of servers. To prevent this, a new generation of Web servers sometimes called non-blocking servers feature routines that let them suspend connections and manage several of them in a single thread. Researchers have been implementing such a server for wireless sensor nodes that can manage up to 256 Comet connections~\cite{Duquennoy2009a}.

More recently, WebSockets (part of the HTML5 drafts~\citeweb{websockets}) were proposed. WebSockets propose duplex communication with a single TCP/IP connection directly accessible from any compliant browser through a simple JavaScript API. The increasing support for HTML5 in Web and Mobile Web browsers and makes it a very good candidate for pushing data in the \WoT{}. Furthermore since WebSockets basically consist of an initial handshake followed by basic message framing, layered over TCP, they can be implemented in a straightforward manner on all platforms supporting TCP/IP, not only browsers.

\paragraph{\tpusher{}}
We propose to add support for WebSockets to the \WoTA{} in order to offer a Web real-time eventing mechanism to communicate with \sts{}. Rather than implementing the protocol directly on \sts{} we propose adding it as a component of our \sg{} architecture as shown in \figRef{smartGateways}. We call this new component \tpusher{} (things pusher) as introduced in~\cite{Guinard2011}.



\subparagraph{System Architecture}
\begin{figure}
\imgLine{wot/websockets-gw-smartthing}
\caption{Sequence diagram of the real-time communication between Web clients and \sts{} through \sgs{} and the \tpusher{} component. \tpusher{} is deployed on a \sg{} where it is used to serve content from \sts{} through a WebSocket interface.}
\label{fig:tpusherAndSgs}
\end{figure}
 \tpusher{}'s integration and usage is summarized by the sequence diagram \figRef{tpusherAndSgs}. The sequence of events to enable a real-time communication between Web clients and \sts{} is as follow:
\begin{description}
 \item[\sg{} Subscription to the \stsBig{}] First, the \sg{} needs to establish a communication with the \sts{}. This is done through an HTTP Callback (Web Hook) subscription. Alternatively, for \sts{} not supporting TCP/IP and HTTP communication, this can be done through the synchronization-based driver approach that we will present in \sectRef{sync-based} where the \sg{} polls the \sts{} regularly using its Device Driver. From this point on, the \sg{} will regularly get data either by getting it pushed by the device or by pulling it.

 \item[Client Upgrade to WebSockets] The Web client then issues a \code{POST} request on the \sg{} (e.g., on \RESTURLInLine{/topic/temperature}) and asks for a protocol \code{Upgrade} to WebSockets, note that the protocol \code{Upgrade} is a standard HTTP mechanism. The \code{Upgrade} is accepted and WebSocket messages can be sent back and forth between the \sg{} and the Web Client.

 \item[WebSocket Push] From this point on, the \sg{} relays (through the \tpusher{} module) the data to the Web client over the same TCP/IP socket that is being kept open.
 \end{description}

The WebSocket specification also offers a JavaScript API that allows creating clients directly in browsers. The simplicity of this API (that should be supported by most browsers when HTML5 is finalized) is the power of WebSockets. Indeed, a shown in \lstRef{jsWebSockets} within 6 lines of simple JavaScript code, Web applications can open a WebSocket connection and thus, in our case, have a standard Web real-time communication with \sts{}.

\lstinputlisting[caption=WebSockets JavaScript Client API. These lines of code are enough for a Web page to subscribe to a WebSocket and react on all possible incoming events., label=lst:jsWebSockets, breaklines, numbers=left, numberstyle=\tiny, xleftmargin=0.8cm, basicstyle=\small\ttfamily, backgroundcolor=\color{gray}, captionpos=b]{code/ws-api.js}

\subparagraph{Software Implementation}
Our implementation is based on Atmosphere~\citeweb{atmosphere}, a Java abstraction framework for enabling push support on most Java Web servers. One of the advantages of this approach is to be able to deploy \tpusher{} on recent Web Servers such as Grizzly~\citeweb{grizzly}, which are highly optimized to push events on the Web because of their usage of non-blocking threads for each new client. In order to support browsers or other clients that do not support HTML5 WebSockets yet, we use a client-side abstraction JavaScript library called Atmosphere JQuery Plugin which falls back to a Comet type of connection in case WebSockets are not supported by the client.

\subsection{Summary and Applications}
In this section we discussed the integration of \sts{} to the Internet and the Web. First, we applied the RESTful principles in a systematic manner. These guideline are then applied to Web-enable several \sts{}. In particular in \sectRef{spots} where they are applied to Wireless sensor nodes, or in \sectRef{EPCISWebadapter} where we apply them to RFID systems.

Then, we discussed the concept of \sgs{} to bring non TCP/IP and HTTP objects to the Web. The generic architecture of the \sg{} we described is used as a basis for the WSN Web-enabling described in \sectRef{sync-based} as well as a guideline to implement a \sg{} for smart meters described in \sectRef{SmartGateway4SmartMeter}.

Finally, we discussed several ways of adding support for \sts{} to push events to Web clients and proposed the \tpusher{} service as an extension of \sgs{}. The \tpusher{} service is evaluated in \sectRef{tagpusher} where is it used to push data from RFID readers to mobile phones.

\section{\findLayer{}}\label{findLayer}
\begin{center}
\includegraphics[width=0.7\linewidth]{figures/wot/findLayer}
\end{center}
By applying the architectural design presented in the \devLayer{}, \sts{} become seamlessly part of the Web. While this presents several advantages, it also raises important challenges. Amongst these is searching and finding relevant services: \important{Given an ecosystem of billions of \sts{}, how do we find their services to integrate them into composite applications?}

The Web faced similar challenges when it moved from an hypertext of several thousands of documents to an application platform interconnecting an unprecedented number of documents, multimedia content and services. Rapidly, search engines such as Altavista, Yahoo and more recently Google appeared to offer search and indexes services.

The \WoT{} will face similar problems, while finding \sts{} by browsing HTML pages with hyperlinks in a home environment is suitable and desirable, on a city, country or world-wide scale it becomes literally impossible. Hence, the ambient findability~\cite{Morville2005} of \sts{} need to be addressed, we need to make them searcheable and findable. 

While we do not pretend providing the ultimate solution to this complex and heavily-researched problem~\cite{Tan2008,Romer2010}, we report on two aspects that we studied in the context of the \WoTLong{}. First, we look at the integration of \sts{} to existing search engines and propose the use of a description model implemented with semantic annotations to enable this. 

Then, we illustrate the shortcomings of basing the findability of \sts{} solely on existing search engines and propose a lookup and registration infrastructure adapted to the particular needs of the \WoTLong{} and building upon the proposed description model. The combination of both solutions enables users and developers to run search queries such as looking for all the nearby temperature sensors or finding a device that can read video content in a particular building.

\subsection{Search Engines and the Internet of Things}
A Web page really becomes usable on the Web once it has been indexed by search engines. Thus, the most straightforward way of enabling the search for \sts{} is through these search engines. However, searching for things is significantly more complicated than searching for documents.

First, \sts{} have no obvious easily indexable properties, such as human readable text in the case of documents. Then, they are tightly bound to contextual information, such as their absolute location (i.e., latitude and longitude), their abstract location (e.g., Room B, Floor 1) or current owner. 

\subsubsection{A Smart Things Description Model and Microformats for the WoT}\label{std}
\begin{figure}
\imgLandscape{wot/things-description}
\caption{The \st{} metadata model, containing the most important elements of the description of a \st{} required for their findability on the Web. This graph contains the static properties of \sts{}, \figRef{thingsDescription2} contains the dynamic properties.}
\label{fig:thingsDescription}
\end{figure}
Hence, \sts{} need a mechanism to describe themselves and their services to be (automatically) discovered and used. Since both humans and machines are going to use the things, we need a mechanism to describe a \st{} on the Web so that both, humans and machines, can understand what services it provides. This problem is not inherent to \sts{}, but more generally a complex problem of describing services, which has always been an important challenge to be tackled in the Web research community, usually in the area of the Semantic Web. For the \WoT{}, the problem has also roots in the notion of \important{context} in Ubiquitous Computing~\cite{Schmidt1999-Context,Salber1999}. 

Here, we propose a model~\cite{Guinard2010-Search} (called Smart Things Metadata model) of the contextual information and metadata a \st{} should disclose on the Web to be searcheable and integrable into composite applications. We base our model on several surveys~\cite{Hong2010,Aguilar2010,Dober2009} of existing languages for semantically describing real-world objects and in particular, sensors~\cite{Botts2007} and industrial machines~\cite{Jammes2005-DPWS}. We describe a \st{} along 2 clusters of information each containing two sub-groups:
\begin{itemize}
 \item First, static properties, as shown in \figRef{thingsDescription} are metadata that will not evolve over the life-cycle of the object:

    \begin{enumerate}
    \item Product: contains a description of what the \sts{} is in terms of object.
    \item Services: contains a description of the services a \st{} offers (e.g., temperature monitoring, MP3 play-back, etc.)
    \end{enumerate}

 \item Then, dynamic properties, as shown in \figRef{thingsDescription2} are those changing regularly depending on the context the object is located in:
    \begin{enumerate}
    \item Location: contains information about the place where the thing is currently located.
    \item QoS (Quality of Service): contains information about how well the thing performs and performed.
    \end{enumerate}
\end{itemize}
This model is not exhaustive but, according to our experience~\cite{Guinard2010-Search}, it covers the basic information required to describe \sts{} on the Web in order to make them and their services searcheable. Furthermore, the idea is to use this information as search engines do, i.e., in a best effort manner where the absence of some metadata does not mean the \st{} is not indexed. Rather it means that customization of its rendering or indexed keywords will simply be limited. 

The description proposed here can potentially be materialized into several formats such as WSDL (Web Service Definition Language) files, DPWS Metadata~\cite{Jammes2005-DPWS} or SensorML~\cite{Botts2007} documents. Unfortunately they are not exposed on the Web as Web-browsers and search engines, for the most part, do not understand them.

To overcome the rather limited descriptive power of resources on the Web, several languages have been proposed as standards. Two of them, RDFa~\citeweb{rdfa} and microformats~\citeweb{microformats} have the interesting feature of being used to semantically enhance the elements of HTML pages.

Designed for both, human and machines, microformats provide a simple way to add semantics to Web resources~\cite{Allsopp2007}. There is not one single microformat, but rather a number of them, each one for a particular domain; a \code{geo} and \code{adr} microformat for describing places or an \code{hProduct} and \code{hReview} microformat for describing products and what people think about them.

Microformats are especially interesting in the Web of Things for three reasons; first, like RDFa they are directly embedded into Web pages and thus can be used to semantically annotate the HTML representation of a thing's RESTful API. 
Secondly, each microformat undergoes an open community-driven standardization process. This ensures that the number of formats stays relatively small and that their content is to be widely understood and used when accepted.  
Finally, many microformats are already supported by search engines, such as Google and Yahoo, where they are used to enhance search results and render them differently. For example, the \code{geo} microformat is used to localize search results close to a user or \code{hReview} is used to rank search results according to the users' opinion. 

As a consequence, we propose the use of microformats to describe \sts{} in the Web of Things. Rather than proposing a new microformat encompassing the model described in \figRef{thingsDescription}, we can re-use a compound of existing, standardized microformats. This helps the things to be directly searcheable by humans using existing general purpose or dedicated search engines, but it also helps them being discovered and understood by software applications in order to automatically use them and render adapted user interfaces.

To illustrate this, we show that by using a compound of 5 microformats and by leveraging the structure of RESTful APIs, we can create a description of a sensor node that fulfills the model presented in \figRef{thingsDescription} and \figRef{thingsDescription2}.
\begin{figure}
\imgLine{wot/things-description2}
\caption{Second part of the \sts{} metadata model, containing the most important dynamic properties of the description of a \st{} required for the findability of their services on the Web. \figRef{thingsDescription} contains the static properties of the \stm{} model.}
\label{fig:thingsDescription2}
\end{figure}
\paragraph{Product Description} One of the most important metadata required in order to enable the search for \sts{} and their services is a description of what object they are. Web sites such as e-commerce services, are often based on unstructured product data which makes it hard for browsers and search engine to render and index useful metadata about products. The \important{hProduct}~\citeweb{hProduct} microformat was created to give a structure to this metadata. Through its vast usage, browsers, search engines and other Web applications have a way to help facilitate the best product choice for consumers. It also gives a way for manufacturers and retailers to better describe their products. Although it is officially still a draft microformat~\citeweb{hProduct} at the time of writing, hProduct is already widely used and implemented on the Web. As an example, the BestBuy e-commerce site uses it for providing metadata about all its products and Google~\citeweb{richSnippets} supports it to better render the results of product searches.
 
Interestingly enough, the hProduct microformat provides information about the object itself and its manufacturer and covers most of the fields required in the Product description of the \stm{} model. As shown in \tableRef{product}, except for the Owner and Manufacturer, hProduct covers all the \stm{} model Product related fields. The Owner and Manufacturer is implemented using the hCard microformat that we will present below.
\begin{table}
\begin{center}
\small
  \begin{tabular}{ | l l l p{4.0cm} | }
    \hline
    \textbf{\stm{} element} & \textbf{Microformat} &\textbf{MF Attribute} & \textbf{Meaning}\\
\hline
    Unique ID & hProduct & identifier (type, value) & unique identifier for this object\\
\hline
    Name & hProduct & fn & human-friendly name of the \st{}\\
\hline
    Brand & hProduct & brand & company name\\
\hline
    Description & hProduct & description & human-friendly description\\
\hline
    Picture & hProduct & picture & image of the product\\
\hline
    Authoritative URL & hProduct & url & manufacturer's Web page containing information about the product\\
\hline
    Tags & hProduct and rel-tag & category (rel-tag) & tags describing the \st{}\\
\hline
    Owner & hCard & see \tableRef{location} & name and address of the owner\\
\hline
    Manufacturer & hCard & see \tableRef{location} & name and address of the manufacturer\\
\hline
  \end{tabular}
  \caption{Elements of the \stm{} model for the Product cluster that can be implemented in a Web-oriented way using standard microformats.}
  \label{tab:product}
\normalsize
\end{center}
\end{table}
Listing~\ref{lst:sensorNodeProductMF} presents an example of how a generic sensor node could be represented using hProduct. Note that the microformats' attributes are directly embedded into the HTML representation of the \st{}. As a results, browsers, search engines and applications discovering the generic node by browsing will be able to render its UI and visualization in a metadata enhanced manner.

In \lstRef{sensorNodeProductMF}, the \st{} unique identifier is implemented using an EPC number. Using Electronic Product Code numbers~\cite{Sarma2001} has the advantage of offering a world-wide, static way of identifying objects which is very valuable in the Web of Things where objects might move from one domain to the other, thus changing their absolute URI over time. We will discuss the properties of EPC numbers in greater details in \chapterRef{autoid}.


\lstinputlisting[caption=Describing a \st{} using hProduct., label=lst:sensorNodeProductMF, breaklines, numbers=left, numberstyle=\tiny, xleftmargin=0.8cm, basicstyle=\small\ttfamily, backgroundcolor=\color{gray}, captionpos=b]{code/sensorNodeProductMF.html}

\paragraph{Location} One of the most important differences between virtual and physical objects is that the latter have a location in a physical context. This information is very valuable and should be leveraged when providing metadata for \sts{}.

Initially created to represent people, companies and organizations, the hCard microformat~\citeweb{hCard-mf} is a also simple and Web interoperable way of representing places. It is based on the vCard specification~\citeweb{vcard} and has reached the status of standard microformat. As a consequence, it is widely implemented on sites across the Web and used for adapted rendering and context extraction by several Web resources and applications. As an example, it is used by Google both to render special results for businesses, showing their location on a Map, as well as to enable their customers to \quote{export places} from Google Maps~\citeweb{googleSupportshCard}. Similarly, the Yahoo Local Search engine uses hCards to render the location of businesses and organizations~\citeweb{yahooSupportsMF}.

In addition to hCard, the geo microformat~\citeweb{Geo-mf} makes it possible to embed absolute location information in Web pages in the form of geographic coordinates.

In the context of \sts{}, we use hCards and geo to implement three parts of the \stm{} model. First, for the static properties, hCards are used to describe the owner and manufacturer of an object. Then, for dynamic properties, we use hCard and geo to describe the location of a \st{}. The mapping of the \stm{} model location properties to hCards attributes is quite natural and shown in \tableRef{location}. 

\begin{table}[h]
\begin{center}
\small
  \begin{tabular}{ | l l p{3cm} p{5cm} | }
    \hline
    \textbf{\stm{} element} & \textbf{Microformat} &\textbf{MF Attribute} & \textbf{Meaning}\\
\hline
    Latitude & geo & latitude & current latitude of the \st{}, owner or manufacturer\\
\hline
    Longitude & geo & longitude & current longitude\\
\hline
    Address & hCard & adr (street-address, locality, postal-code, country-name)  & comprehensive postal address\\
\hline
  \end{tabular}
  \caption{Elements of the \stm{} model for the Location cluster that can be implemented in a Web-oriented way using the hCard and Geo standard microformats.}
  \label{tab:location}
\normalsize
\end{center}
\end{table}
It is worth noting that these microformats do not cover relative abstract locations. The reason behind this is that this part of a location cannot be leveraged globally in a standard way (e.g., by a search engine) as it requires a specific knowledge of the current environment. In \sectRef{lookupInfra} we propose a way to implement this part of the \stm{} model using a lookup infrastructure.

\paragraph{Quality of Service} In a Web of Things populated by billions of \sts{} quality of service information can be of great help to choose the right \st{} for the right application. Parameters such as bandwidth, up-time, average response time help taking the right decision. These data can be based on monitoring service~\cite{Guinard2010-Search} or provided by the \st{} manufacturer.

However, with the advent of the Web 2.0, we increasingly rely on external user experiences when choosing our products and services.
The strong influence of recommendation systems on the way people pick Web sites or buy products online has been extensively studied~\cite{Senecal2004} and demonstrated~\cite{Reischach2009}. For \sts{} we can take a similar approach and offer a standard way for providing user-generated reviews as well as performance information.

hReview is a microformat for embedding reviews of products, services, businesses and events in Web representations and especially in HTML~\citeweb{hReview}. Several Web sites already implement hReview for their reviews. As an example, the New York Times Web site~\citeweb{travelny} and Yahoo Local search use it to rate listed venues such as restaurants and businesses.

In the Web of Things context, we can use hReview to implement both QoS properties listed in the \stm{} model. Indeed, as shown in \tableRef{qos} the standard attributes of hReview cover the metadata related to performances and user feedback.
\begin{table}[h]
\begin{center}
\small
  \begin{tabular}{ | l l p{3cm} p{5cm} | }
    \hline
    \textbf{\stm{} element} & \textbf{Microformat} &\textbf{MF Attribute} & \textbf{Meaning}\\
\hline
    Review & hReview & description \& type (product) & feedback from the owner/user of the \st{}\\
\hline
    Rating & hReview & rating (value, worst, best) & owner/user rating between worst and best or 1.0 to 5.0 if scale is omitted\\
\hline
    Address & hCard & adr (street-address, locality, postal-code, country-name)  & comprehensive postal address\\
\hline
    Service health & hReview \& rel-tag & tag (rel-tag) & specifies that a review is a service health parameter\\
\hline
    Network latency & hReview \& rel-tag & tag (rel-tag) & specifies that a review is a service health parameter\\
\hline
  \end{tabular}
  \caption{Elements of the \stm{} model for the QoS cluster that can be implemented in a Web-oriented way using the hReview standard microformat.}
  \label{tab:qos}
\normalsize
\end{center}
\end{table}
\lstRef{sensorNodeQoSMF} shows how the QoS properties of the \stm{} model can be implemented using hReview. The listing contains two QoS elements: First, the owner of the \st{} published a user feedback. Then, the \st{} generated a service health review.

\begin{lstlisting}[caption=Quality of service for a \st{} described using the hReview microformat., label=lst:sensorNodeQoSMF, breaklines, numbers=left, numberstyle=\tiny, xleftmargin=0.8cm, basicstyle=\small\ttfamily, backgroundcolor=\color{gray}, captionpos=b]
<div class="hReview">
  <span><span class="rating">4</span> out of 5 stars</span>
  <h4 class="summary">Good all purpose sensor node</h4>
  <span class="reviewer vcard">Added by owner: <span class="fn">Dominique Guinard</span></span>
  <div class="description item">I use this generic sensor node for monitoring the temperature inside my house. It is quite reliable but I turn it off on weekends.
  </div>
</div>
<div class="hReview">
  <span>
    <a href="http://www.webofthings.com/tags/serviceHealth" rel="tag">Service Health:
      <span class="value">70</span>/
      <span class="best">100</span>
    </a>
    </span>
    <h4 class="summary">Service health is good</h4>
    <span class="reviewer vcard">Added by manufacturer: 
      <span class="fn">Generic Electronics Company</span>
    </span>
    <div class="description item">The service health of this node is good, this means that most requests will succeed within less than 1 second.
    </div>
</div>
\end{lstlisting}


\paragraph{Service Description: Discovery by Crawling}\label{discoveryByCrawling}
The last part of the \stm{} model does not necessarily need to be supported by an explicit semantic description. Indeed, if we consider that the \devLayer{} was implemented as described, we can assume that all the \sts{} will serve their functionality through a RESTful interface.

A direct consequence of respecting the constraints of REST is that useful meta information can be extracted simply by crawling their HTML representation~\cite{Mayer2011,Alarcon2010} and leveraging the HTTP protocol.

From the root HTML page of the \st{}, a crawler typically is able to find a number of the service properties suggested in the \stm{} model. First, to satisfy constraint C4 (Hypermedia Driving Application State, see \sectRef{APIforSmartThings}), the HTML representation of a \st{} should contains links to related and descendant resources. Hence, from this constraint a crawler can extract the Children, Parents and related URIs as specified in the \stm{} model.

With these URIs, the crawler can then use the HTTP \texttt{OPTION} method to retrieve all verbs supported for a particular resource, e.g., \texttt{PUT}, \texttt{POST}, \texttt{GET}, implementing the Operations property of the \stm{} model. Finally, with content-negotiation as described in \sectRef{interfaceDesign}, the crawler gets information about the Service Format, Input and Output properties. 

We implemented and empirically tested the described crawling algorithm in~\cite{Guinard2010-sharing,Mayer2011}. The pseudo code of this algorithm is shown in \lstRef{serviceCrawlingAlgo}.
\begin{lstlisting}[caption=The \sts{} Service Description Crawling Algorithm., label=lst:serviceCrawlingAlgo, breaklines, numbers=left, numberstyle=\tiny, language={Java},
xleftmargin=0.8cm, basicstyle=\small\ttfamily, backgroundcolor=\color{gray}, captionpos=b]
crawl(Link currentLink) {
  new Resource() r;
  r.setUri = currentLink.getURI();
  r.setShortDescription = currentLink.text();
  r.setLongDescription = currentLink.invokeVerb(GET).extractDescriptionFromResults();
  r.setOperations = currentLink.invokeVerb(OPTIONS).getVerbs();
  foreach (Format formats: currentFormat) {
    r.setAcceptedFormats = currentLink.invokeVerb(GET).setAcceptHeader(currentFormat);
  }
  if (currentLink.hasNext()) crawl(currentLink.getNext());
}
foreach (Link currentPage.extractLinks(): currentLink);
\end{lstlisting}

The crawling approach to extract service metadata is interesting because it does not require the semantics of services to be represented in an additional format. As a consequence valuable information can be extracted even if the \st{} to index implements the \devLayer{} only. In fact the information that can be extracted by crawling is rich enough to index a \st{}, a few keywords and to locate all its resources. Hence, the \st{} already becomes searcheable.

However, the crawling approach has two main limitations. First, the approach strongly relies on the respect of the REST constraints. As a consequence, for some \sts{}' APIs such as those based on hybrid architectures~\cite{Richardson2007} not fully respecting the constraints of REST, the service metadata might be only partially extracted~\cite{Alarcon2010}.

Furthermore, the crawling approach requires many HTTP calls to extract a metadata profile that matches the service properties of the SDT model. This is problematic since HTTP calls are the most costly parts of the communication between clients and services~\cite{Souders2007}. As a consequence, a single HTTP call returning a significant amount of data is more efficient than several calls returning the same total amount of data. This is especially important in the context of the Web of Things where clients will interact with services deployed on resource constrained devices such as \sgs{} or \sts{}.

Several solutions exist to provide service metadata for RESTful APIs. The most well known one is called WADL (Web Application Description Language~\cite{Richardson2007}). This language, directly inspired from the WSDL (Web Service Description Language), provides a way of describing HTTP based Web applications. A WADL document is an external document describing, from a client point of view, how to interact with a given HTTP based service. The main drawback of the approach in our context is that it requires clients to understand a new format. Furthermore, when compared to microformats, the rather low adoption rate of the WADL format~\cite{Richardson2007} does not enable applications to leverage existing search engines.

hRESTs is a microformat~\cite{Kopecky2008} sharing similar goals with WADL, with the advantage of being directly embedded in the HTML representations of services. Because most of the metadata it offers is implicitly available in a well designed RESTful API, hRESTS is sometimes criticized by the Web community. More importantly, hREST is the work of three researchers and, at the date of writing it is not yet an official community-driven microformat which severely hinders its support by services such as search engines. 

However, in practice hRESTs offers the advantage of providing the service metadata without crawling the resources. As a consequence, the metadata is more strictly organized and easier to extract which reduces the number of required HTTP calls. To benefit from this, hRESTs should be used to annotate a global description of the \st{}'s services, for example accessible at the root HTML page of the \st{}.

%\todo{1: maybe describe how we would use hREST, in the API documentation.}

\paragraph{Understanding the Benefits of Microformats}
Following our guidelines, a \st{} is best described by a compound of five microformats covering the \stm{} model: hProduct, hCard, hReview, rel-tag and possibly hRESTs.

The benefits of this approach are manifolds. First, \sts{} become directly searcheable with traditional search engines such as Google or Yahoo. Moreover, these search engines can use the metadata to provide contextual search results. As an example, searching for a temperature sensor nearby from a mobile phone can use the geo microformat of the \sts{} to match the GPS coordinates of the mobile phone. Search engines will also be able to render the search results differently based on the metadata of \sts{}.

Similarly, based on this metadata, clients such as Web browsers or mobile phone applications are able to render the user interfaces to \sts{} in a customized way, we illustrate this with examples of dynamically rendered UIs and mashup modules for Wireless Sensor Nodes in \chapterRef{wsn}.

\paragraph{Towards an \transService{}}\label{translationService}
While we suggest implementing the \stm{} model using the proposed compound microformat for the best current integration experience, it is clear that this is not a one-size-fits-all format. From the history of metadata formats such as DPWS, WSDL, WADL, SensorML~\cite{Botts2007} or SA-REST~\cite{Sheth2007} it is quite clear that no metadata language can impose itself up to a point where others vanish, because there is no single best way of describing a \st{}. Hence, \sts{} will most likely be and already are described using other metadata formats.

However, the format should not matter, what should is the metadata. Hence, the idea is to introduce a level of indirection in order to support a broad spectrum of metadata formats. As introduced in a common work with Simon Mayer~\cite{Mayer2011} as well as in~\cite{Guinard2010-Search,Aguilar2010}, we implement a \transService{} that acts as a converter. On the one hand-side it can extract (or crawl) information from several metadata formats and on the other end, through a RESTful Web API, it offers to clients to retrieved the extracted metadata using the representation they wish (if supported).

\subsection{A Web-Oriented Discovery and Lookup Infrastructure}\label{lookupInfra}
Relying on search engines to enable searching for \sts{} is interesting because it uses existing, well-known and widely adopted services. However, the approach has a number of limitations. First, because of their mobile nature, \sts{} tend to be moved from one context to another on a regular basis: sensors attached to shipments move from the factory to a warehouse. Mobile phones entirely change their context several times per day. Environmental monitoring systems are moved from one observation area to the other. While improving support for real-time search (e.g., through integration with real-time information services such as Twitter), search engines still largely function based on scheduled indexing and might not reflect the latest context of registered \sts{}. Thus, the need for \important{local search engines and lookup services for \sts{}}.

Moreover, the bootstrapping of \sts{} is a problem: \important{How does a \st{} announce its existence in a particular context?} Currently, search engines discover new resources by following links. For the Web of Things, we need to be able to access \sts{} as soon as they connect thus the need for a \important{discovery service for the Web of Things}.

We present a distributed and Web-oriented registration and lookup infrastructure for the \WoT{} federating our joint work in~\cite{Mayer2010,Trifa10book} as well as our work on defining a search and discovery process for real-world services running on \sts{}~\cite{Guinard2010-Search}\footnote{Parts of the described process were patented in~\cite{GuinardPatent2010}.}. We begin by demonstrating how the infrastructure offers a discovery protocol for \sts{}. We then show how this infrastructure can be used to perform local search queries.

\subsubsection{Distributed Infrastructure}\label{distributedInfra}
\begin{figure}
\imgMedium{wot/infrawot-hierarchy}
\caption{Hierarchical organization of the Local Lookup and and Discover Units (LLDU). Virtual LLDUs are can be located anywhere whereas Physical LLDUs are coupled with Smart Gateways.}
\label{fig:thingsHierarchy}
\end{figure}
Our Discovery and Lookup infrastructure is composed of several Local Lookup and Discovery Units (LLDUs)~\cite{Guinard2010-Search,Mayer2010}. These software components allow \sts{} to announce themselves and clients to search for specific (local) services offered by connected \sts{}. The internal structure of an LLDU is shown in \figRef{llduImpl} and will be described in details in the next sections.

Based on common work with Trifa et al.~\cite{Trifa2010-location} we suggest than rather than having a flat structure for LLDUs, we deploy them in a hierarchical way, reflecting the abstract locations of the current context in the resources' URIs. Using abstract location information in URIs has a great value since it facilitates browsing for \sts{} in a particular context. For instance it lets you navigate through all the \sts{} in a building, floor or room simply by pointing to the correct URI of following the provided hyperlinks structure.

A typical hierarchy is shown in \figRef{thingsHierarchy}. The first three levels are virtual LLDUs. Virtual LLDUs can be deployed on any machine anywhere in the world and one physical machine can host several virtual LLDUs. Smart things do not directly communicate with them as their purpose is only to encapsulate the hierarchy of abstract locations and to serve search queries for these locations. As an example, in \figRef{thingsHierarchy} the \code{ethz, ifw, cab} and \code{floor-h} are virtual LLDUs all hosted on the same physical machine.

Like virtual LLDUs, physical LLDUs serve search queries for the abstract locations they represent, however physical LLDUs also serve as Discovery Services for \sts{}. A physical LLDU is a software component that can be loaded in a Smart Gateway. As shown in \figRef{thingsHierarchy} the \code{office-107.1} LLDU node covers the abstract location encapsulated in the following URI: \RESTURL{/ethz/cab/floor-h/office-107.1/}. Directly below this node are attached the resource trees (see Section \ref{ROA-for-things}) formed by \sts{} managed by the smart gateway in \code{office-107.1}.

Concretely, the LLDUs can be deployed and configured using a \code{PUT} request to the root URI of a running LLDU with a payload specifying its configuration and context. As an example \lstRef{configInfraWoT} is a JSON document that configures a new LLDU located at \RESTURLInLine{/eth}. The rest of the specified contextual information (e.g., latitude, longitude) will be inherited by \sts{} connected to this LLDU that cannot deliver a full \stm{} model, for instance because they do not have a GPS module.

\begin{lstlisting}[caption=JSON document that configures a LLDU called \quote{LLDU for ETH} and located at ETH., label=lst:configInfraWoT, breaklines, numbers=left, numberstyle=\tiny, xleftmargin=0.8cm, basicstyle=\small\ttfamily, backgroundcolor=\color{gray}, captionpos=b]
{
  "resourceUrl": "http://webofthings.com:2401",
  "uuids": [{"uuidType": "infraWoT", "uuidValue": "eth"}],
  "name": "LLDU for ETH", 
  "context": {
    "hierachical": {"hierarchyString": "eth/", "hierarchyDelimiter": "/"}, 
    "postal": "Universitaetsstrasse 6, CH-8092 Zurich, Switzerland",
    "geographical": {"longitude": 8.550003, "latitude": 47.367347}
  }
}
\end{lstlisting}


\subsubsection{Discovery Services}
\begin{figure}
\centering
\imgLine{wot/discovery-sequence}
\caption{Sequence diagram of the discovery process. The LLDU Discovery Service uses a \transService{} to extract metadata for the \st{}. The returned information is sent to the LLDU Registry Service which indexes and stores the metadata of the discovered \st{}.}
\label{fig:discoverySequence}
\end{figure}
To solve the bootstrapping problem of \sts{}, physical LLDUs offer a Discovery Service. The discovery process is started by a \st{} wanting to be part of the Web of Things infrastructure. 

As shown in \figRef{discoverySequence}, once connected to the local network, the \st{} or gateway issues a \code{POST} request to the \RESTURLInLine{/resources} end-point of the physical LLDU. As a payload of the request, the \st{} can either send its root URI or a payload describing its resources. The Discovery Service sends this to the \transService{} which will extract semantic metadata based on a best-effort principle, supporting several types of metadata formats. The \transService{} returns a JSON representation of the extracted data.

The Discovery Service then binds the \st{} to the physical LLDU's absolute URI. As an example, after the discovery process, the \code{abstractNode1} in \figRef{discoverySequence} gets bound to \RESTURL{/ethz/cab/floor-h/office-107.1/smart-things/abstractNode1}.
Then, the resources tree of the \code{abstractNode1} itself is also accessible through the LLDU as shown in \figRef{thingsHierarchy} below the \RESTURLInLine{abstractNode1} resource.

Furthermore, to enable keywords-based search, the Discovery Service passes the \stm{} model to a Registry Service. The role of this latter is to store the representation of the model as well as to extract two inverted indexes from it. Inverted indexes are a central component of search engines algorithms and are well suited for keywords-based textual searches~\cite{Zobel1998}. For each resource the Registry Service creates three different entries, as show on \lstRef{invertedindex}. First, it store the JSON string corresponding to the metadata of the resource in a file. It then adds entries into two inverted indexes. In the first index, it adds all the keywords that could be extracted from the resource's metadata. In the second index, it adds keywords extracted from the metadata related to the output of the resource.

\begin{lstlisting}[caption={Indexing the resources in two inverted indexes based on extracted keywords of the \stm{} model.}, label=lst:invertedindex, breaklines, numbers=left, numberstyle=\tiny, language={}, xleftmargin=0.8cm, basicstyle=\small\ttfamily, backgroundcolor=\color{gray}, captionpos=b]
// Store resource in table
writeStringToFile(resource.toJSONObject().toString(), databaseCoreTable);			

// Get keywords for the resource and add inverted index entries
for (String keyword : resource.getKeywords()) writeReverseEntryToFile(keyword.toLowerCase(), resource.toJSONObject().toString(), keywordReverseTable);

// Get REST Output from entity and add inverted index entries
for (String restOutput : entity.getRESTOutput()) writeReverseEntryToFile(restOutput.toLowerCase(), entity.toJSONObject().toString(), restOutputReverseTable);
\end{lstlisting}

\subsubsection{Lookup Services}
One of the benefits of deploying an infrastructure of LLDUs is the ability to perform localized search queries. The lookup service offers a query interface for clients such as developers looking for real-world services to integrate into their composite applications, end-users wanting to discover the registered services for a particular place or applications dynamically looking for simple services.

\paragraph{Types of Parameters}
The Query Service of LLDUs is built on top of the Registry Service. Clients can access it by sending a \code{POST} request to the \RESTURLInLine{<LLDU-URI>/query} resource on an LLDU. The actual query should be specified either through \RESTURL{application/x-www-form-urlencoded} parameters or as a JSON payload. Since queries will be distributed amongst the infrastructure of the LLDUs (traveling down or up the resources tree), LLDUs also pass queries to each-other using the same mechanism.

Queries can be formulated according to the following parameters basically corresponding to most relevant fields of the \stm{} model that were extracted by the \transService{} during the discovery process:

\begin{description}
 \item[Keywords] A number of free-text, unstructured keywords can be provided. The matching algorithm is a traditional keywords search process iterating through the following properties that were extracted from the device's representation of the \stm{} model: name, category, brand, description and user provided tags. These keywords can then be extended by the system using external services as explained in \sectRef{queryAug}.
 
 \item[Name] As several \sts{} support user provided names, searching for these might be really valuable to users and thus is offered by the API.

 \item[Unique ID] Queries by universal unique identifiers e.g., Bluetooth IDs, Zigbee MAC addresses, IPs, Electronic Product Codes (EPC, see \sectRef{epcNet}), are a straightforward way for applications to search for a particular \st{}.

 \item[Ratings] Clients can use user generated or \sts{} provided quality of service ratings as specified in the \stm{} model. For instance this type of query parameters can be used to find the most reliable wireless sensor node of a certain type as in practice it is often the case that one node is more reliable than the other.

 \item[REST Service] The matching algorithm activated by this parameter leverages the metadata enhanced description of RESTful APIs based on the hRESTs microformat. 
\end{description}

When performing a search, the results sets for each type of parameter are fetched and the intersection of all the sets is returned to the client through the RESTful Web query interface.

\paragraph{Types of Queries}
Parameters define the keywords of a query but thanks to the tree-structure formed by all LLDUs the locality of searches or scope can also be leveraged. This concept is encapsulated in the \code{queryType} parameter that has to be provided by clients when using the querying API.

We consider three types of queries that can be performed on the LLDU infrastructure and illustrate their particular interest when looking for services provided by \sts{}:

\begin{description}
 \item[Exhaustive Queries] These queries start at the LLDU where the request originated and are pushed to all children LLDU nodes, eventually returning all the resources that matched within the subtree. Thus, such a query will go down to the leafs and up again through every node until the originator is reached again. As an example such a query can be used to retrieve all the temperature sensors in a city in order to compute an average temperature.

  \item[Cardinality Queries] These queries are used to find exactly $n$ resources corresponding to the query parameters. The query process is launched on the children LLDU nodes and will be stopped as soon as $n$ services are found in the result set. However, since the process is distributed amongst several subtrees in its current implementation the process may retrieve more than $n$ services, hence the result set is eventually filtered to keep only $n$ results, giving more weight to LLDUs located higher in the subtree. Such a query could be used for instance to find pairs of smart meters that monitor a certain type of device (e.g., a fridge) to compare their actual energy consumption.
 
  \item[Best Effort Queries] Such a query is in fact a Cardinality Query with a stopping condition of $n=1$. It is used to find the first resource that fits the user needs. As an example it can be used to find a usable printer in a facility.

  \item[Located Queries] Since the other types of queries will start their tree-traversal only from the location of the originator LLDU, there is a need to support an arbitrary starting point. Located Queries implement this feature. When a hierarchical location is specified in a query, a Located Query is triggered and sent to the corresponding LLDU in the hierarchy where the query is started.
  Such a query can be used for instance in the case a user is located in a particular room (bound to the physical LLDU in this room) but wants to query for the energy consumption of the whole department he is located in.
 \end{description}

%\todo{1: Figure with a tree of the queries}

\paragraph{Query Augmentation Service}\label{queryAug}
To provide better results without requiring additional semantics on the \sts{} side, the Query Service can be extended with a query with a Query Augmentation Service we proposed in~\cite{Guinard2010-Search}.

In conventional service discovery applications, the keywords entered by the user are sent to a service repository to find types of services corresponding to the keywords. The problem with this simple keyword matching mechanism is that it lacks flexibility required in the special case of real-world objects. As an example lets assume a developer or a user who wants to find services offered by a \important{smart meter}, a term often used to describe a device that can measure the energy consumption of other devices and possibly control them depending on built-in logic. Typing \quote{smart meter} only, will likely not lead to finding all the corresponding services, because services dealing with energy consumption monitoring might not be tagged with the \newterm{smart meter} keywords but simply with \newterm{electronic powermeter}. However, since we want to avoid the construction of domain specific ontologies, and to minimize the amount of data that \sts{} need to provide upon network discovery and service registration, we propose a system that uses services on the Web to extend queries without involving communication with the \sts{} or requiring domain specific service descriptions from them.

The basic idea is to use existing knowledge repositories such as Web encyclopedias (e.g., Wikipedia), search engines (e.g. Google, Yahoo! Web Search) or domain-specific portals (e.g., the Metering portal~\citeweb{metering}), in order to extract \important{lightweight ontologies}~\cite{Hepp2007} or vocabularies of terms from the Web resources' semi-structured results. The basic concept of the Query Augmentation is to call $1..n$ Web search engines or encyclopedias with the search terms provided by the user, for instance \quote{smart meter}. The HTML result page from each Web resource is then automatically downloaded and analyzed. The result is a list of keywords, which frequently appeared on pages related to \quote{smart meter}. A number of the resulting keywords are thus related to the initial keyword i.e., \quote{smart meter} and therefore can be used when searching for types of services corresponding to the initial input.

\subparagraph{Software Architecture} An invocable Web-resource together with several filters and analysis applied to the results is called a \important{Query Strategy}. The structure is based on the Strategy Pattern \cite{Gamma1995},  which enables us to encapsulate algorithms into entirely independent and interchangeable classes. This eases the implementation of new strategies based on Web resources containing relevant terms for a particular domain. 
A simplified class diagram of the Query Strategy framework is depicted on \figRef{archiStrategies}. Any Query Strategy has to implement the \code{AbstractStrategy} class which provides the general definition of the algorithm. As an example the \code{YahooStrategy} is a concrete implementation of this algorithm using the Yahoo! Search service. Furthermore, strategies can have extensions, adding more specific functionality to a concrete instance of a Query Strategy. As an example the \texttt{WikipediaStrategy} can be extended with the \code{WikipediaBacklinks} class. This particular extension is using the backlinks operation offered by Wikipedia in order to know what pages are linking to the currently analyzed page similarly to what the well-known PageRank used to rank websites \cite{Brin1998}. This information is then used by the \code{WikipediaStrategy} to browse to related pages and gather relevant keywords. As such, our approach builds on top of existing ranking and connectivity approaches on the Web.

Furthermore, Query Strategies can be combined in order to get a final result that reflects the successive results of calling a number of Web-resources. The resulting list of related keywords is then returned to the user, where he can (optionally) remove keywords that are not relevant. The implementation of the Query Strategy architecture makes it easy to test combinations of several strategies together with their extensions. We implemented a number of these, and their evaluation is presented in \sectRef{searchEval}.
\begin{figure}
\imgLine{wot/query-strategies-class}
\caption{Architectural overview of the Query Strategies based on the Strategy and Template software design patterns.}
\label{fig:archiStrategies}
\end{figure}

\subparagraph{Context Extractor}
One of the main differences between services provided by \sts{} and virtual services is that \sts{} services are directly linked to the physical world. As a consequence, the context in which a service exists as well as the context in which the user or user initiates the discovery of a service are highly relevant. Context is information that qualifies the physical world, and it can help in both reducing the number of services returned to the user, as well as in finding the most appropriate services for the current environment \cite{Balke2004}.

To satisfy the requirements of real-world service discovery, we propose modeling the context into two distinct parts inspired from \cite{Schmidt1999-Context}: the \newterm{Digital Environment}, which we define as everything that is related to the virtual world the user is using, and the \newterm{Physical Environment}, which refers to properties of the physical situation the user currently is located in or wants to discover services about.

The \newterm{Digital Environment} is composed of Application Context and Quality of Service. The \newterm{Application Context} describes the business application the user uses when trying to discover services, e.g., the type of application he is currently developing or the language currently set as default. Such information co-determines the services a user or developer is looking for and can reduce the discovery scope. 
The \newterm{QoS Information} reflects the expectations of the user (or of the application he is currently using) in terms of how the discovered service is expected to perform. Our current implementation supports service health and network latency, i.e., the current status of the service and the network transmission delay usually measured when calling it.

The Physical Environment is mainly composed of information about location. Developers are likely to be looking for real-world services located at a particular place, unlike when searching most virtual services. We decompose the location into two sub-parts following the Location API for Mobile Devices (as defined in Java Specification Request JSR-179). The Address encapsulates the virtual description of the current location, with information such as building name, floor, street, country, etc. and the Coordinates are GPS coordinates. In our implementation the location can either be automatically extracted e.g., if the user looks for a real-world service close to his location, or it can be explicitly specified if he wants a service located close to a particular location e.g., in a form of radius. 

Extraction of the context on the user side is done when starting the query in the \sts{} lookup service Web user interface, the user can also influence these parameters by setting up preferences. It is worth noting that the context on the user side is meant to reflect the expectations or requirements with regard to the services that are going to be returned. As an example, during this phase the user can express the wish for a service to be physically close to his current location, or he can quantify the importance of context parameters such as Quality of Service. 

This user-quality information is then going to be compared with the context stored on the LLDUs' indexes extracted by the \transService{} when discovering the \sts{}. This is done by the Service Ranking component in order to select and rank the most relevant resources.

\subparagraph{Context of \stsBig{}}
The Digital Environment context parameters such as the device description or Quality of Service, are extracted upon discovery by the Discovery Service based on the microformat implementation of the \stm{} model. 

Getting the context parameters related to the Physical Environment of a service instance is slightly more complicated. Indeed, as an example it can not be expected from each \sts{} to know its location. Thus, we suggest taking a best effort strategy, where each actor of the discovery process is trying to further fill-in the context object. As an example, consider a mobile sensor node without a coordinates-resolving module (e.g., a GPS). 

Upon discovery by a LLDU, the sensor node does not know its location and thus can not fill-in the Address and Coordinates fields of the \stm{} model. The LLDU however, is a usually immobile component and is configured at setup time with its location and current address as explained in \sectRef{distributedInfra}. As a consequence the LLDU can provided the Address and Coordinate information of the sensor node based on its own location (within a specific radius). While not entirely accurate with respect to the sensor's exact location, this information will already provide a useful approximation. Similarly, since we can not expect every LLDU to provide a full contextual profile, the LLDUs can also share their contextual information to complement one another. 

\subparagraph{Ranking Service Lookup Results}\label{ranking}
The Service Ranking component is responsible for sorting the resources according to their compliance with the context specified by the user or extracted from his machine. This component receives a number of service lookup results alongside with their context profiles. It then uses a Ranking Strategy to sort the list of results. For instance, a Ranking Strategy can use the network latency so that the services are listed sorted according to their network latency; another could rank instances according to their compliance with the current location of a user or the target location he provided. 

As for Query Strategies, Ranking strategies can be well modeled using the Strategy pattern. In this way, new strategies can be easily implemented and integrated. Furthermore, we extend the pattern to support chained ranking strategies, in order for the resulting ranking to reflect a multi-criteria evaluation. Each ranking criterion can use both the context information of the instances gathered during the discovery process, and the context information extracted on the user side. Thus, instances can be ranked against each other and/or against the context of the user (e.g., his location). The output of the ranking process is an ordered list of running services offered by resources on \sts{} corresponding both to the extended keywords and to the requirements in terms of context expressed by the user either implicitly or explicitly. 

\subsubsection{Lookup Process Summary}
\begin{figure}
\imgLandscape{wot/lookup-sequence}
\caption{Sequence diagram of the lookup process. Clients contact the \RESTURLInLine{/query} resource on an LLDU, the keywords of the query can then be extended using the \code{QueryExtension} service. Once a lightweight ontology of keywords has been extracted, the query is distributed along the LLDU tree and results are aggregated and ranked before sending them back to the clients.}
\label{fig:lookupSequence}
\end{figure}
A simplified summary of the complete lookup process is provided in \figRef{lookupSequence}. First the client (e.g., a user or client application) sends a query request to the LLDU of his choice. Then, the LLDU can contact the Query Extension service which will enrich the user query with a number of related keywords extracted from relevant Web services. Then, these keywords are packed with the query and sent to the relevant LLDUs, which LLDUs are relevant is determined by the type of query. This initiates a recursive tree exploration. Eventually, a result set is returned to the LLDU where the query started. There, the LLDU can use the Ranking service which will sort the results based on a chained list of ranking strategies and on the user specified (or extracted) context. A ranked list of resources (and their provided services) is returned to the client.

\subsubsection{Software Implementation}
\begin{figure}
\imgLine{wot/lldu-and-sg}
\caption{The modules of the Lookup and Discovery Infrastructure are integrated in the \sg{} framework through OGSi bundles. A LLDU is formed of 5 main services that are implemented as OSGi bundles running locally. Two additional services (Query Augmentation and Translation Service) can run outside the local environment (e.g., on the Web) as they can be used by several LLDUs.}
\label{fig:llduImpl}
\end{figure}
To facilitate integration with \sg{} framework presented in \chapterRef{wot}, \sectRef{gateways}, the Lookup and Discovery infrastructure is implemented as several OSGi bundles that communicate with each other via OSGi declarative services (OSGi DS). The integration of these services to the \sg{} framework is shown in \figRef{llduImpl}. 

Basically, an LLDU can run on any machine. As mentioned before, physical LLDUs are coupled with \sgs{} to simplify the deployment virtual LLDUs, that do not need physical access to the devices can be deployed anywhere.

The implementation is based on 5 internal services (Registry Service, Lookup Service, Infrastructure Service, Discovery Service) that are to be deployed with each LLDU (virtual or physical). The two other services (Query Augmentation Service and the \stm{} Model Translation Service) are ideally deployed outside (e.g., on the Web) because there is no benefit to run them locally as all LLDUs can use the same instance of these services.

\subsection{Evaluation}
We structure the evaluation into two parts. First, in a quantitative evaluation we analyze the response time when querying the lookup infrastructure. Then, we evaluate the Query Extensions mechanism with real-world data generated by 17 experienced developers during a user-study~\cite{Guinard2010-Search}.

\paragraph{Evaluation of the Lookup Service}\label{searchEval}
\begin{figure}
\imgLine{wot/infrawot-eval}
\caption{Tree representation of the LLDU infrastructure deployed for the evaluation. The \code{/cnb} node is a physical LLDU to which two concrete wireless sensor nodes are connected.}
\label{fig:infraWoTEvalTree}
\end{figure}
A scenario was implemented in order to assess the feasibility of running service lookup queries on top of proposed distributed infrastructure of LLDUs.

The tree structure of the implemented scenario is shown in \figRef{infraWoTEvalTree}. Each node in this three represents an instance of an LLDU. However, all instances were deployed on the same machine located in the \RESTURLInLine{cnb} building at ETH Zurich. A \sg{} is deployed in this LLDU and connected to two sensor nodes (\sunspots{} nodes, see \chapterRef{wsn}). One sensor node binds itself to the \RESTURL{europe/ch/ethz/cnb} LLDU and one to the \RESTURL{europe/ch/ethz/cnb/h/107-2/} LLDU. As a consequence, their resources trees becomes part of the overall resource tree of the infrastructure as show in \figRef{infraWoTEvalTree}. However, the depth of the tree used in a lookup comprises only the hierarchy of LLDUs and thus has a maximal depth of 6 because the rest of the actual \sts{} resources trees where already indexed upon discovery by LLDUs. To generate some noise, a total of 61 virtual resources were attached to the virtual and physical LLDUs.

The machine on which the LLDU resources tree is deployed is a Linux Ubuntu Intel dual-core PC 2.4 GHz with 2 GB of RAM. The Web server used for this implementation is based on the Noelios Restlet Engine 1.1.7~\citeweb{restlet}.

We perform 10000 keywords queries looking for services matching the \newterm{light} keyword. With this setup, the minimal observed response time is 12 ms, the maximum 3753 ms with an average response time of 619 ms as detailed in \figRef{infraWoTEval}.

\begin{figure}
\imgLine{wot/lldu-response-time}
\caption{Response times when running a keyword query on the test deployment. Most queries get answered within 250 to 750 milliseconds.}
\label{fig:infraWoTEval}
\end{figure}

The aim of this evaluation is not to prove that our implementation is performing best but rather to illustrate that the response times are reasonable. It is worth noting however, that in this scenario all LLDUs were run by the same machine and network latency between the LLDUs of an infrastructure would have to be taken into account in a real-world deployment.

\paragraph{Evaluation of Types Query and Candidate Search}\label{sec:evalQuery}
\begin{figure}
\imgMedium{wot/queryAugmentation}
\caption{Results for the Query Augmentation with Yahoo! and Wikipedia, the Query Augmentation has a positive impact on the number of services found but it also generates more false positives.}
\label{fig:queryAugmentation}
\end{figure}
In this second part we evaluate the impact of the proposed query extensions mechanisms on the search for services provided by \sts{}.

In order to have a neutral base of \sts{} and their services on which to perform the evaluation we selected seventeen experienced developers and asked them to write the description of a selected device and of at least two services it could offer. The developers were given the documentation of a concrete device according to the projects they were currently working on. Based on these descriptions we generated thirty types of services offered by sixteen different \sts{} ranging from RFID readers to robots and sensor boards. Out of these, 1000 devices were simulated on a host PC.

It is worth noting that the \stm{} model based service descriptions were generated as DPWS metadata~\cite{Guinard2010-Search}. However, as the expressive power of the microformat implementation of the \stm{} model is greater than what can be expressed with DPWS metadata and as a \transService{} translates all metadata formats into a single internal representation, the results are applicable to any implementation of the \stm{} model.

The main idea of the evaluation was to find out whether: 
\begin{enumerate}
\item Augmenting users' input with related keywords could help in finding more services on \sts{}. 
\item What type of combination of query strategies is the most suitable. 
\end{enumerate}

Two types of strategies were used. In the first we used a human generated index (i.e., Wikipedia), and in the second a robot generated index (i.e., Yahoo! Web Search). The input keywords were selected by seven volunteers, all working in IT. They provided seventy-one search terms (composed of one to two words) reflecting what they would use if they were to search for services provided by the seventeen \sts{} when wanting to develop new applications with these \sts{}. These terms were entered one by one and all the results were logged.

The trends extracted from these experiments is shown in \figRef{queryAugmentation}. Two results can be drawn. First the Query Augmentation process does help in finding more \sts{} services. Without augmentation 75\% (plain gray line in \figRef{queryAugmentation}) of the resources corresponding to the queries were found and using the Query Augmentation up to a 100\%. 

However, the Query Augmentation generates a number of false positives, i.e., resources that are returned even if they are not related to the provided keywords (depicted by the two lines at the bottom of \figRef{queryAugmentation}). Thus we need to restrict the number of keywords added to the initial ones. The observed optimum is between 5 and 10 added keywords, leading to less than 20\% false positives out of 95\% services found. The second result can be seen in \figRef{queryAugmentation} which reveals that using Yahoo!, the approach performs slightly better than when using Wikipedia. 

Looking more at the details we see that approximately 50\% of the keywords used against Wikipedia did not lead to any page, simply because they do not have yet dedicated articles, even if Wikipedia is growing at a rate of more than 1000 articles per day (as of 2011)~\citeweb{wikipediasize}. However, when results where extracted from Wikipedia pages they were actually more relevant for the searched real-world services. Thus, a good solution would be to chain the strategies so that first human generated indexes are called and then robot generated ones, in case the first part did not lead to any results.

The Ranking Service Lookup was evaluated based on a proof of concept implementation. We tested two chained ranking strategies for the generated services; one comparing service health and given weight of 30\% as well as one comparing network latency and given a weight of 50\%. They performed as expected, sorting the lists of retrieved service instances according to the ranking strategies which, we believe helps users finding their way across the results, but would need to be tested with neutral volunteers. 

We implemented the sorting using the merge sort algorithm which has a complexity of $O(n \log n)$, and since the strategies can be chained we have an overhead for the ranking of $O(m n \log n)$ where $m$ is the number of strategies and $n$ the number of resource descriptions.

\subsection{Summary and Applications}
In this section we proposed a metadata model for describing \sts{} and their services. Furthermore, we proposed an implementation of the model based on microformats that are well understood on the Web for example by search engines. In \chapterRef{wsn} (see \sectRef{findLayerSpots}) we apply this model and its implementation to the description of a general purpose wireless sensor platform and illustrate how it can be leveraged to dynamically render UIs for interacting with \sts{} or to make them searcheable on the Web and in our lookup infrastructure. Furthermore, we will see the benefits of such a model in the next layers, the \shareLayer{} and the \compoLayer{}.

We also presented an infrastructure that can be deployed together with \sgs{} in order to encapsulate the abstract location of \sts{} as well as to offer a localized discovery and lookup infrastructure. A concrete usage of this infrastructure is evaluated in \sectRef{findLayerSpots} as well.
\newpage



\section{\shareLayer{}}\label{sharing}
\begin{center}
\includegraphics[width=0.7\linewidth]{figures/wot/sharingLayer}
\end{center}
With the \devLayer{} of the \WoTA{} we ensure that digitally augmented everyday objects are seamlessly integrated to the Web. With the \findLayer{} we enable humans and applications to find the \st's services they look for directly from the Web and leveraging contextual information.

% The success of Web as a service platform is closely dependent upon the trend for Web 2.0 services to provide access to some of their services through relatively simple and often REST-based open APIs on the Web. 

Enabling this model for the \WoTLong{} requires a sharing mechanism for \sts{}, by allowing access to services offered by devices as Web resources. An implementation of the two previous layers fulfills this requirement as devices become openly available to the world directly from the Web and without restrictions. For example, one could share the energy consumption sensors in his house with the community. However, since these devices are part of our everyday lives, their unrestricted public sharing might result in serious privacy violations~\cite{Mazurek2010,Langheinrich2005}. In this section we propose a Web architecture that tackles these challenges.

\subsection{Requirements for a \WoT{} Sharing Platform}
HTTP already provides authentication mechanisms for securely sharing resources. The HTTP Basic Access Authentication~\cite{Franks1999} is a method that allows Web clients (and in particular Web browsers) to provide credentials (user names and passwords) when making an HTTP request on a server.  In practice, HTTP Basic Access Authentication~\cite{Franks1999} is coupled with SSL/TLS in order to make sure that the user names and passwords are not transmitted in clear text over the wire. HTTP Digest Authentication on the other hand, ensures that the credentials are always encrypted.

While these two solutions are already available on (embedded) Web servers they present a number of drawbacks. First, when considering a large number of \sts{} it becomes quite unmanageable to create and share credentials for each of them and for each contact one wants to share with. Then, the credentials used in these systems are often impersonal and do not reflect any social or trust structures already in place. Then, as the shared resources are not advertised anywhere, sharing also requires the use of (unsecured) secondary channels such as sending emails containing credentials to people. Looking at pitfalls of the existing solutions we propose three requirements for a sharing platform for the \WoT{}:

\begin{description}
 \item[Security] The most basic requirement for a \WoT{} sharing platform is to be secure in order to make sure that access to \sts{} is not granted to attackers.
 \item[Ease of Use] People are concerned with the security of their private data, for example in home environments~\cite{Mazurek2010}. However, it has been shown that the ease of use of a secure sharing system has a significant influence on its adoption and effective usage~\cite{Ion2010}. Hence, a \WoT{} sharing platform should be straightforward and easy to use. 
 \item[Reflect existing trust models] The sharing platform should also reflect mental models users are already familiar with~\cite{Mazurek2010}. In particular it should as much as possible reflect the existing trust and social models of users. 
 \item[Interoperability] In order not to hinder the benefits of adopting an interoperable Web architecture, as for the \devLayer{} and \findLayer{}, the protocols used by the sharing platforms should be interoperable with the Web and understood by most Web tools and clients. Furthermore, the sharing platform should not be bound to a specific social model but should be able to adapt to several systems. 
 \item[Integrated Advertisement] A \WoT{} sharing platform should also support advertising the shared things directly on the Web. In order to reduce the load for users and improve security, sharing a \st{} and advertising the fact that it was shared should occur on the same channel without explicitly disclosing credentials.
\end{description}

First meant for creating groups of people and enabling communication amongst these groups, social networks rapidly evolved into data sharing hubs~\cite{Boyd2008}. Social networks make it very easy to share data (e.g., pictures) with groups of people such as family and friends~\cite{Miller2007}. The social network takes care of the authentication of these individuals and manages \important{access control lists} for the users' data.

We propose leveraging social networks as sharing hubs for \sts{}. In the \shareLayer{} we introduce an architecture and its implementation in a platform that enables the selective sharing of \sts{}. It uses social networks and their social graphs already in place for sharing \sts{} with people relevant to \sts{} owners, creating a \important{Social Web of Things}. 

Furthermore we illustrate how social networks can be used as service advertising platforms and how they support the implementation of the physical feeds aggregators components of the \WoTA{}.

\subsection{Social Access Control: An Architecture for the Social Web of Things}\label{SAC}
\begin{figure}
\imgLine{wot/sac-fat-architecture}
\caption{Simplified components architecture of the Social Access Controller. SAC serves as authentication proxy between clients and embedded devices. It holds the credentials for accessing \sts{} and provides access to selected trusted connections of the owners' social networks. It further offers an API upon which applications can be built.}
\label{fig:componentArchitecture}
\end{figure}
A promising solution to the problem of sharing \sts{} is to leverage existing social structures and build upon social networks (e.g. Facebook, Linkedin, Twitter, etc.) and in particular their social graphs accessible through data access APIs (e.g., OpenSocial) and their authorization APIs (e.g., OAuth). 

Using social networks enables users to share things with people they know and trust such as relatives, friends, colleagues, fellow researchers, etc. This is achieved without the need to recreate yet another social network or user database from scratch on a new online service. Additionally, it enables advertising and sharing through a unique channel: users can tell their friends about the sensors they shared with them by automatically posting messages to their profile or newsfeeds. 

We propose a system to share things and facilitate access to real-world services offering a RESTful Web API.
Our core contribution is a Web architecture and its implementation called \sacLong{}~\cite{Guinard2010-sharing,MathiasFischer2009} (\sac{}) which offers the following functionality:

\begin{description}
 \item[Authentication Proxy] A \sac{} identifies users based on existing credentials rather than requiring the creation of new, impersonal credentials for each \st{} or \sg{}.
 \item[Authorization Proxy] A \sac{} is an authorization proxy that sits between clients and \sts{} and authorizes clients applications (e.g., browsers) to access the \sts{}.
 \item[Access Control Manager] A \sac{} helps users to fine-tune the nature of interactions they want to allow for their objects (e.g., read-only, read-write, etc.) and manages access control based on existing social graphs.
 \item[Advertisement Channel] A \sac{} can advertise shared \sts{} using the notification services of social networks such as user newsfeeds or walls.
\end{description}

Overall, the architecture enables owners of Web-enabled \sts{} to easily share them on the Web. Consider for example an smart meter that implements the \devLayer{}. Sharing the energy consumption recorded by all these smart meters on the Web, enables the creation of very interesting applications. For instance a mashup on a map can show the the consumption of each individual in a group of friends. Similarly, a Web-enabled Hi-Fi system can enable songs to be played remotely through a RESTful interface. Sharing it with close friends enables them to remotely play songs for you. You can also use the system to inform all your friends, on their favorite target device (e.g., mobile phone, laptop, TV, etc.) that you will be a little late. The global system architecture shown in \figRef{componentArchitecture} addresses these use-cases. It is composed of a central Web application, the \sac{} server, which creates the link between social networks and \sts{}. 

In this section, we define \newterm{owners} as people owing or administrating \sts{} (e.g., a Web-enabled sensor node) and \newterm{trusted connections} as the people owners share their \sts{} with (e.g., friends, colleagues or relatives). It is worth noting, however, that owners and trusted connections can also be applications. 

\subsection{Retrieving the Owners' Social Graphs}
The process of sharing \sts{} with trusted connections occurs in three phases. In the first phase, the owner accesses the \sac{} server by logging in using one or several of his social networks credentials as shown in the step 1 of \figRef{sacRegisterShare}. Then, the owner's lists of trusted connections (i.e., social graphs) need to be retrieved using delegated authorization on the social networks as shown in step 2 of \figRef{sacRegisterShare}.


\subsubsection{Leveraging Web Authorization Protocols: OAuth}
As mentioned before a \sac{} is an authorization and authentication proxy between clients (e.g., Web browsers) and the \sts{}. Rather than maintaining its own database or list of trusted connections and credentials -- as it would be done with Basic or Digest HTTP Authentication -- it connects to a number of social networks to extract all potential users and groups one could share with. As a consequence owners need to be authenticated to their social networks and a \sac{} need to be authorized to retrieve lists of trusted connections.

To achieve this first step, we use the OAuth 1.0 protocol~\cite{Eran2010-oauth}. OAuth is a delegated authorization protocol that enables users to allow client applications to access their data without revealing the users credentials~\cite{Allamaraju2010}. OAuth has the advantage of fulfilling our requirements for a \sacLong{}. First, it is supported by a vast majority of social networks. Secondly, as an open standard it is backed by a large community of developers and security experts and is being monitored for security breaches which makes it a rather secure option. Thirdly, it is built on top of HTTP and thus is well integrated to the Web and interoperable. Finally, since the user-facing part of the protocol is relying on the existing login systems of social networks, users are familiar with them and thus the system can be considered as relatively easy to use.
\begin{figure}
\imgHalf{wot/oauth-sac}
\caption{Using OAuth for a \sac{} to retrieve an owners' trusted connections: 1) Click on social network button 2) RequestToken? 3) OK, RequestToken + RequestSecret 4) Redirect to social network with RequestToken 5) Login + Grant permission to SAC 6) OK, Redirect to SAC Server 7) Exchange RequestToken to AccessToken? 8) Owner login OK, AccessToken + AccessSecret 9) List of Friends? 10) AccessToken + ConsumerKey OK, List of Friends}
\label{fig:oauth}
\end{figure}
%\todo{1: possibly provide a better figure...} 

The OAuth protocol is based on a so called three-legged scheme because there are three parties involved in the protocol~\cite{Allamaraju2010}. The client, the service provider and a user. In the context of a \sac{}, the client is the \sac{} server, the user is the owner of a \st{} and the service providers are the social networks the owner has an account with. The communication to authenticate the user and authorize a \sac{} to access the trusted connections of the owner is shown in \figRef{oauth} and detailed here:

\begin{enumerate}
 \setcounter{enumi}{0}
 \item The \sac{} server gets a \code{ConsumerKey} and \code{ConsumerSecret} from the social network.
 \item The owner selects a social network by clicking on its corresponding button on the \sac{} server login page.
 \item The \sac{} server asks the selected social network for a \code{RequestToken}.
 \item The \code{RequestToken} is granted together with a \code{RequestSecret}.
 \item The owner is redirected to the social network where he logs in and grants the permissions to the \sac{} server.
 \item The login was successful and the owner is redirected to the \sac{} server.
 \item The \sac{} server asks to exchange its \code{RequestToken} for an \code{AccessToken}.
 \item Since the owner login was successful the \code{AccessToken} is granted by the social network.
 \item The \sac{} server can now request owner related data from the social network.
  \item The data is granted since the \sac{} has a valid \code{AccessToken} and \code{ConsumerKey}.
\end{enumerate}

\paragraph{Ensuring Statelessness}
It is worth noting that in order to respect the constraint of Stateless Interactions of RESTful architecture described in \sectRef{deviceLayer}, a \sac{} should not store the access tokens and secrets given by the social networks upon successful authentication and authorization. Indeed, this ensures that requests to a \sac{} can be cached and proxied~\cite{Richardson2007}. Hence, the \sac{} requires clients to store this information. This is best achieved by storing it in the form of cookies~\cite{Kristol1997-cookies}. These cookies are then sent in the header of each subsequent request to a \sac{}. As a consequence, the transmission of this information should occur over an encrypted channel (e.g., HTTPS) as an attacker could use the unencrypted tokens to impersonate a user.

However, an attacker could not use these tokens to compromise and use the user's social data. Indeed, access to the data is only granted to particular server application here, a \sac{} server. This is ensured through signing the request with an API secret key only known to the \sac{} server.



\subsubsection{Leveraging Social Network APIs}
OAuth is meant to authorize applications to access, on users' behalves, other applications such as social networks. However, the specification does not propose a standard way of accessing the social network data once authorized and authenticated, it does not standardize the reads/writes API of a social network. Nevertheless, providing an open Web API is one of the success factors of social networks themselves. Indeed, these APIs allow third-parties Web applications to be built using partial data extracted from the social networks and thus to enhance the functionality and value of the social networks. 

\paragraph{OpenSocial}The OpenSocial~\cite{OpenSocial2010} specification was created to fulfill the need for standard social network APIs. It defines a common API for application to access data across several social networks. OpenSocial uses OAuth for authorizing an application to access the social network. Then, the access to the standard social API is using a REST or RPC architecture.

When building a \sacLong{}, this type of standard is central as it enables to retrieve lists of trusted connections from any compliant social network and thus keeps to architecture open for integrating new and existing social networks. Once authorized to access the social data with OAuth, a \sac{} server has a standard way of accessing trusted connections. As an example, the OpenSocial RESTful API~\citeweb{OpenSocRESTSpec} call for downloading all the contacts of a user is a \code{GET} request on: \RESTURL{/people/{USER_ID}/@all}.

Unfortunately, some major social networks such as Facebook or Twitter do not comply with OpenSocial. For these networks, proprietary APIs such as the Facebook (Connect) API have to be used. While similar to OpenSocial in terms of functionality, the APIs significantly differ, making it impossible to access the data of these networks in a uniform manner.

Thus, we suggest for \sac{} servers to support access to social network data using a plugin architecture, enabling the support of both OpenSocial based and proprietary APIs. We further describe this plugin architecture in \sectRef{sac-architecture}.

For each social network a user is currently logged in with, the \sac{} server uses the Web API of the social networks (through an OAuth authorization) and queries them for lists of friends and other connections as shown in \figRef{sacRegisterShare}. All these connections are then compiled into a global list of potential connections that the owner can share with.

%\todo{3: Explain how a social network API works in a generic way}

\subsection{Registering and Sharing \stsBig{} and \sgs{}}
In the second phase of the sharing process, the owner registers the \sts{} and \sgs{} he owns.
\begin{figure}
\imgLine{wot/sac_fat_register_share}
\caption{Process for registering and sharing a \st{} using a \sac{} server. The owner authenticates himself using a social network (1,2). He then provides his credentials to access the \st{} (3). The \st{} is crawled for resources (4) that the owner can share (5). The shared resources are advertised on social networks (6).}
\label{fig:sacRegisterShare}
\end{figure}
The prerequisites for a \st{} to be shared with a \sac{} are based on a relaxed subset of the \devLayer{} presented in \sectRef{deviceLayer} that can be summarized as follow:
\begin{itemize}
 \item \textbf{Addressability:} All the shareable functions offered by \sts{} should be modeled as resources~\cite{Fielding2000} which are addressable and identified by resolvable URIs.
 \item \textbf{Uniform Interface:} The actions available on resources should comply with the HTTP verbs (e.g., a \code{GET} on a resource retrieves a representation of that resource).
 \item \textbf{Resource Description:} The embedded Web servers on \sts{} (or \sg{}) should support one of the service metadata description methods proposed in \sectRef{discoveryByCrawling}. It is worth noting that the only real requirement is to respect the \important{connectedness} constraint as the core information required to share \sts{} with a \sac{} can be obtained simply by crawling the RESTful Web API. Additional metadata can be used to provide owners and trusted connections with more detailed descriptions of the resources thus improving the system's usability.
 \end{itemize}
Additionally, to ensure that the \sts{} are secured, their direct access should be restricted as shown in the rightmost part of \figRef{componentArchitecture} using a standard HTTP method (e.g., HTTP Basic Authentication with SSL/TLS or HTTP Digest Authentication). If not provided \quote{out-of-the-box}, this can be done by setting up the Web server at the device level or at the gateway level to accept only authenticated HTTPS traffic and require credentials for any incoming request.  

The actual registration and sharing process is depicted in \figRef{sacRegisterShare}. First, the owner logs in to one or more of his social networks using the \sac{} cross-network OAuth client. He can start registering the \sts{} and \sgs{} that belong to him and sharing them with the trusted connections retrieved by the \sac{} server (step 2 in \figRef{sacRegisterShare}). To do so, he provides the credentials to access a \st{} or the credentials of a \sg{} that manages several \sts{} as shown on step 3 of \figRef{sacRegisterShare}. 

Using these credentials, the \sac{} server accesses the \sts{} and crawls them (step 4 of \figRef{sacRegisterShare}). This is done by using an \transService{} (see \sectRef{translationService}). Using this service, the \sac{} server is able to identify the available services and expose them for sharing in a user-friendly manner as show in \figRef{FAT-share}.

The owner can then share the discovered services of \sts{} with trusted connections such as friends, relatives, or colleagues (step 5 of \figRef{sacRegisterShare}). He can either share complete \sts{} (e.g., a sensor node) or their sub-resources only (e.g., the temperature sensor of the sensor node only). Furthermore, he can choose to share resources in read-only (i.e., allowing the \code{GET} verb only) or read-write (i.e., giving access to all the available HTTP methods). \figRef{FAT-share} shows a user interface to share \sts{} implemented on top of a \sac{} server.  

Finally, for each shared resource of a \st{}, the \sac{} server uses the corresponding social network messaging system to post a notification to the trusted connection the resource was shared with as shown in step 6 of \figRef{sacRegisterShare}. As for retrieving lists of trusted connections, this is done through the social network API. In the case of an OpenSocial compliant social network, this is done simply with a \code{POST} request on:
\RESTURL{/messages/{USER-ID}}.

In the case of other social networks, the \sac{} server has to use the proprietary API for sending notifications through the network. In our implementation, for Facebook, the \sac{} server publishes a message to the newsfeed of the trusted connection. In the case of Twitter it simply tweets a message to the trusted connection. Note that the posted message does not contain any credentials but a link pointing back to the \sac{} server where the data of the shared resource can be fetched by the trusted connection once authenticated.

\subsection{Accessing Shared \stsBig{}}\label{sac-accessing}
\begin{figure}[!htb]
\centering
\includegraphics[width=\linewidth]{figures/wot/sac-fat-access.png}
\caption{Accessing shared objects using delegated authentication through the \sacLong{}. The trusted connection requests the shared resource's URI (1). If not logged in with the corresponding social network, the \sac{} server asks the trusted connection to login (2). The \sac{} server then access the \st{} and redirects the results to the trusted connection's client (3, 4).}
\label{fig:sac-access}
\end{figure}
Once a trusted connection gets notified of the fact that resources of a \st{} were shared with him, he uses the provided URI to access it as shown in step 1 of \figRef{sac-access}.

The shared URI points back to an instance of a \sac{}. When receiving the HTTP request, the \sac{} server prompts the trusted connection for log in if no cookie corresponding to a successful authentication on one of the \sac{} recognized social network is found. Indeed, just as \st{} owners need to be authenticated and to grant access to their social network data, trusted connections wanting to access the shared \sts{} need to get authenticated. Because it authorizes applications in the name of users, OAuth can also be used to authenticate trusted connections. However, as trusted connections simply use the system as a proxy there is no need for a complete delegated authorization process.

In this case, the \sac{} server simply needs to confirm that the trusted connection is the person it pretends to be as shown in step 2 of \figRef{sac-access}. A simple, user-friendly manner to ensure this on the Web is through delegated authentication.

\paragraph{Leveraging Delegated Authentication}
A delegated authentication for a \sac{} presents two advantages. First, trusted connections do not need special credentials or a dedicated registration for accessing the shared resources, as they can use the credentials of any social network or service on the Web that supports delegated authentication. Second, the \sac{} does not need to hold profile information about the users (a user ID is enough) and can support several social networks for a single trusted connection. 

OpenId is the dominating protocol for delegated authentication on the Web. Its core idea is to offer Web users a mechanism for transporting their identity from site to site, avoiding for them to have to go through a registration process for each site. Unlike OAuth, which is both a user authentication and an application authorization protocol, OpenId does not grant data access to the client Web site beyond a limited user profile.

After the \sac{} server successfully confirmed the identity of the trusted connection using a delegated authentication client (OpenId or OAuth), it internally checks whether this person also has access to the requested resource. If it is the case the \sac{} server logs on the shared resource using the credentials provided by the owner when registering the resource. It then redirects the HTTP request of the trusted connection to the shared resource as shown in step 3 of \figRef{sac-access}. Finally, it redirects the result  directly to the HTTP client of the trusted connection (step 4), e.g., to a Web browser. 


\subsection{Physical Feeds Aggregation}
%Explain how we enable grouping and simple composition through the use of Atom
%Explaing Twitter.
%Talk about the way apps can use that.
While direct access to a single device might be interesting for control scenarios, as for instance playing a song on a Hi-Fi system or turning off a device remotely, monitoring use-cases require a system that allows to group several events coming from \sts{} together and publish them to a messaging or feeds server on the Web.

Thus, \sac{} provides a syndication mechanism that can be used to monitor several \sts{} at once. It consists of an \code{Updater} component which periodically polls the \sts{} for updates and sends the updates to a syndication server (e.g., an Atom server). \code{Updater}s can be parameterized by specifying the amount (number of characters or percentage of change) of content that should be changed in order to generate a new event. A regular expression which should be satisfied can also be specified. Finally, another regular expression can be used to reformat the content of the event before publishing it. The new events are then published by the \code{Publisher} components which are abstraction of Web publishing mechanisms. Similarly to the \code{NetworkConnector}s, \code{Publisher}s rely on an extensible architecture to be able to quickly add support for new services.  

An example scenario for this system is a friend who can be informed when you leave work. By monitoring the energy consumption of your computer, a notification will be generated and transmitted when your computer is turned off. Another scenario is a friend who creates a simple physical mashup with Google Maps that shows friends available in the neighborhood. This mashup could be simply based on an Atom feed to which the \code{Publisher} Component sends update events whenever a friend leaves his workplace.


%\todo{1: this should be described more, with examples, with screen-shots and it should be explained
%as an additional component, that allows first level physical mashups. Tal about period, reg exp (see %page 17 of Mat. report)}


\subsection{Software Architecture}\label{sac-architecture}
\begin{figure}
\imgMedium{wot/sac-components-archi}
\caption{Overview of the \sac{} software architecture. At the core of the architecture, the \code{AuthenticationManager} is responsible for managing a list of the connected users. The \code{NetworkConnector}s are used to authenticate and authorize users with all supported social networks. The \code{GatewayManager} manages the \sgs{} and the \code{ProxyRedirector} fetches resources from the \sgs{} in the name of trusted connections. Finally, \code{Updater}s and \code{Publisher}s manage the feeds.}
\label{fig:sac-archi}
\end{figure}
In this section, we present the software architecture of our \sac{} implementation, based on a Resource Oriented Architecture implemented using the Object Oriented paradigm. 

Our architecture further uses the EJB (Enterprise Java Beans) patterns~\cite{Monson-Haefel2006} and thus is organized around a data model describing the actors, or entities of the system, managed by a number of managers implementing the business logic. We first present the data model and then the most important manager components.

\paragraph{Data Model}
A \sac{} server is implemented around a relatively small set of data classes, also called \code{Entities} in the Object Oriented terminology, that we briefly describe here. 

\begin{description}
  \item[Administrators and Gateways] The primary purpose of \sac{} is to enforce rules for accessing \WoT{} devices. Hence, the \code{Gateway}s data classes represent either actual \sgs{} or \sts{} since they both expose the same type of \devLayer{}. Owners of \sts{} i.e., users who registered the \sts{}, are called \code{Administrator}s.
  
  \item[Social Networks and Accounts] \code{SocialNetwork} entities represent social networks that can be used for authentication and authorization. These entities are linked to users through the \code{Account} classes. \code{Account} classes represent all the social network a user (owner or trusted connection) logged in successfully with using the \sac{} OAuth client.

  \item[Resources and Shares] These entities represent the \sts{}. A \code{Resource} is a shareable functionality on a \st{} (e.g., the \code{temperature} resource of a sensor node). Once shared, a \code{Resource} is linked to a \code{Share} entity which is turn is linked to \code{Permission} entities representing whether the resource is shared in read-only or read-write mode.

  \item[Publish Systems and Updater] \code{Publishsystem}s represent Web services where updates of a resource can be published. A \code{RegisteredFeed} is a scheduled periodic task that takes care of posting an update to a \code{PublishSystem} if the new state of the \code{Share} it monitors meets the specified conditions.
\end{description} 

The entities are stored and loaded through a \code{PersistenceLayer} as shown in \figRef{sac-archi} which is a simplified overview of the software architecture of our \sac{} implementation. Following the EJB patterns~\cite{Monson-Haefel2006}, entities are managed by a number of \code{Manager} components that can be seen in \figRef{sac-archi}.

\paragraph{Business Logic} Just as \sts{} in the Web of Things are accessible through RESTful APIs, access to \sac{} functionality is done through a REST architecture. As shown in \figRef{sac-archi}, incoming HTTP requests reach the \code{ResourceRouter}. This component is based on a REST framework enclosing an HTTP 1.1 compliant Web server and is in charge of forwarding the incoming requests to the matching components which can be a resource, another router or a guard. A resource component simply corresponds to the implementation of a REST resource. Guards are used to protect sensitive components of the architecture and authorize their access only to authenticated and authorized requests.

At the core of the platform lies the components which implement the business logic of the \sac{} and use the entities of our data model. We briefly describe the most important business logic components of the \sac{} architecture:

\begin{figure}
\imgLine{wot/sac-network-connectors}
\caption{Simplified class diagram of the network connectors architecture of \sac{}. Thanks to the modular architecture new connectors can be easily added to the system.}
\label{fig:sac-network-connectors}
\end{figure}
\begin{description}
  \item[Proxies and Gateways Managers] The \code{ProxyRedirector} is a central component of \sac{} as it implements the access to shared \sts{} as described in \sectRef{sac-accessing}. The \code{ResourceRouter} redirects all requests from trusted connections to access \sts{} to the \code{ProxyRedirector}. The \code{ProxyRedirector} then adds the required credentials to the request and forwards it to the secured \sts{}. It finally returns the results back to the client of the trusted connection. When an owner shares a resource with a trusted connection he also, by default, allows access to all resources below the shared one. As a consequence, the \code{ProxyRedirector} must also parse the responses and replace all the absolute links (as well as \code{Location} headers) pointing directly to the \sts{} in order for them to point to the \sts{} proxy.

  The \code{GatewayManager} is in charge of securely holding the credentials to access the \sts{} as well as discovering the services they provide. To do so, it either uses the internal \code{CrawlingModule} or uses an external \transService{} (see \sectRef{translationService}). 
  
  \item[Network Connectors] The \code{NetworkConnector}s are the \sac{} link to social networks: They encapsulate the necessary logic to login to a social networks using delegated authentication as well as to extract data from it using delegated authorization. While OAuth and OpenSocial compliant networks can be accessed using a generic Network Connector, other networks need dedicated connectors that understand their specific (non-standard) API. The \code{NetworkConnector} architecture of \sac{} is presented in \figRef{sac-network-connectors}. It features an abstract \code{NetworkConnector} which can be extended to created dedicated connectors with OAuth, OpenSocial support or with proprietary authentication and authorization protocols (e.g., \code{FacebookConnect}).
 
  \item[Authentication Manager] The \code{AuthenticationManager} works between the proxies and the network connectors. It manages a list of the connected users and their respective social networks. As mentioned before, \sac{} does not store the access tokens and secrets returned by each social network. These are rather sent along by clients in the HTTP header of each request. As a consequence, the authentication manager is responsible for extracting the information from the cookie and using it on the corresponding \code{NetworkConnector} to ensure that the user is still authenticated.

  \item[Updaters and Publishers] In order to enable the aggregation of physical feeds from shared \sts{}, \sac{} uses an \code{Updater} which is in charge of monitoring a resource and pushing their updates to a \code{Publisher}. As shown in \figRef{sac-publisher-archi}, \code{Publisher} is an abstract component that can be extended to support a concrete publishing mechanism. We implemented an \code{AtomPublisher} supporting any AtomPub compliant server and a \code{TwitterPublisher} to publish notifications through the Twitter service. 
\end{description}
\begin{figure}
\imgHalf{wot/sac-publisher-archi}
\caption{Simplified class diagram of the publishers architecture of \sac{}.}
\label{fig:sac-publisher-archi}
\end{figure}

\paragraph{RESTful API}
\begin{figure}
\imgLine{wot/sac-api}
\caption{Tree visualization of the \sac{} RESTful Web API.}
\label{fig:sac-api}
\end{figure}
In order for applications to leverage it, a \sac{} server should implement a RESTful Web API. Our implementation of the \sac{} server offers an API that can be used to: share, add, manage, syndicate, and interact with shared \sts{}. This makes \sac{} an integral part of the \WoT{} since its API is accessible on the Web and can be used to further build applications. For example, a new Smart Gateway or \st{} can be added to \sac{} for sharing by sending a \code{POST} request to \url{/gateways} alongside with the data of the new element (URI, name, description, etc.) as payload.

All the resources available in the API are shown in \figRef{sac-api}. Each resource can be delivered as an HTML, JSON or XML representation.

\subsection{Friends and Things: A Social WoT Web Application}
To exemplify how applications can be built on top of the implementation of \sac{}, we developed the Friends and Things (FAT) application. FAT is a Web application that allows users to share and use shared \sts{} in a user-friendly manner. FAT is written in JavaScript and HTML and uses the \sac{} RESTful Web API. It basically provides a user interface to access the main functionality of the \sac{} server. It helps owners to login with several social networks in parallel, lets them manage \sts{} and \sgs{} and share them with trusted connections (see \figRef{FAT-share}). Finally, it provides owners with usage statistics (see \sectRef{epcSharing} for a concrete example).
\begin{figure}
\imgLine{wot/fat-share}
\caption{Screenshot of the FAT (Friends and Things) application built on top of \sac{}. The owner can select the \sts{} he wants to share with some of his trusted connections.}
\label{fig:FAT-share}
\end{figure}

Furthermore, FAT helps trusted connections leveraging \sts{} that were shared with them. Upon login, trusted connections find a list of \sts{} that were shared with them. As shown in \figRef{FAT-invoke}, using the RESTful Invocation Tool, they can directly test all the authorized HTTP methods (e.g., \code{PUT}, \code{POST}, \code{GET}) and add HTTP payloads to their requests.
\begin{figure}
\imgLine{wot/fat-invoke.png}
\caption{Using the RESTful Invocation Tool, a trusted connection can directly test all the HTTP methods available to him and add HTTP payloads.}
\label{fig:FAT-invoke}
\end{figure}
Finally, trusted connections can create feeds combining different \sts{} that were shared with them and have these feeds automatically updated according to rules specified using regular expressions.


%-------------------------------------------------------------------------
\subsection{Summary and Applications}
Bringing devices to the Web paves the way for a new breed of applications that seamlessly blend the physical world with existing services on the Web. \sac{} and FAT are simple examples of the possibilities revealed by Web-enabling physical objects.

In this section we have presented a system for sharing and controlling access to resources in the \WoTLong{}. The core idea is to leverage existing online social structures rather than relying on closed databases of credentials. Thus, the \sac{} architecture provides a framework which builds upon fast growing social networks such as Facebook, Twitter or LinkedIn to allow users to share physical objects with actual friends, relatives or colleagues.

\sac{} also provides a programmable basis upon which composite Web applications can built. Thanks to the RESTful API of \sac{}, physical mashups and other Web applications can use the \sac{} functionality to share and use shared \sts{}. To illustrate this we introduced the Friends and Things Web application which directly builds upon the API for our implementation of a \sac{} server.

The architecture described in this section and its implementation are evaluated with Wireless Sensor Networks in \chapterRef{wsn} (see \sectRef{wsn-sharing}) where the overhead of using the architecture and its underlying protocol is assessed. Finally, it is used to share access to traces of RFID-tagged objects in \chapterRef{autoid} (see \sectRef{epcSharing}).

\newpage

\section{\compoLayer{}}\label{physicalMashups}
\begin{center}
\includegraphics[width=0.7\linewidth]{figures/wot/compoLayer}
\end{center}
At large, the \WoTLong{} materializes into an open ecosystem of digitally augmented objects on top of which applications can be created using standard Web languages and tools. With the previous layers, we allowed developers to access and search for Web-enabled \sts{} and owners to have a simple and scalable mechanism to share them. Much is to gain from Web integration as it drastically eases the usually rather tedious development of applications on top of \sts{}.

In this last layer, we would like to push further the boundaries of the \WoT{} so that from getting close to developers, it also gets closer to end-users and enables them to create simple composite applications on top of \sts{}. Indeed, the previous layers also deliver more flexibility and customization possibilities for end-users. 

\subsection{Physical Mashups in the Web of Things}
In this section we look at the concepts of Web 2.0 Mashups and further define the notion of \pMashups{}. We then discuss the special requirements of the \WoTLong{} and propose a \pMashups{} architecture based on these requirements.

\subsubsection{Web 2.0 Mashups}
Web Mashups are defined as: \quote{Web applications generated by combining content, presentation, or application functionality from disparate Web sources. They aim at combining these sources to create useful new applications or services} by Yu et al.~\cite{Yu2008a}.

Yee~\cite{Yee2008} characterizes mashups along the combinations of three actions or patterns:
\begin{enumerate}
 \item Data is extracted from a source web site.
 \item The data is translated into a form meaningful to the destination web site.
 \item The repackaged data is sent to the destination site.
\end{enumerate}

Following this pattern, Housingmaps~\citeweb{housingmaps} is one of the most well-known Web mashups. It extracts the list of apartments, rooms or flats that are available for rent or sale on the Craigslist Web site~\citeweb{caigslist} and displays them on Google Maps~\citeweb{googlemaps} according to their location~\cite{Yee2008}. The result is a new service which helps people visually and geographically identifying real estate listings.

Mashup creators often also share their mashups on the Web (sometimes through directories such as the Programmable Web API directory~\citeweb{programmableweb}) and expose them through open APIs as well, making the ecosystem grow with each application and mashup.

The creation of composite applications is key in the idea of mashups. However, according to the literature~\cite{Yu2008a,Pautasso2008,Hartmann2008,Brandt2009}, there are several differences between Web mashups and traditional composite applications:
\begin{description}
  \item[Lightweightness and Simplicity] The technologies used for mashups mainly involve Web standards (e.g., HTML, HTTP, Atom, RSS, Microformats), scripting languages (e.g., JavaScript) and Web programming languages (e.g., PHP, Ruby, Python, etc.)~\cite{Yee2008,Yu2008a}. As a consequence mashups are rather lightweight applications that can be brought to several clients through Web (and mobile Web) browsers. 
 
  \item[Accessibility to a Larger Public] A direct consequence of the simplicity of mashups is their accessibility to a larger public than traditional composite applications. Manual mashups~\cite{Yu2008a}, i.e., mashups that are created without the use of dedicated tools are still mostly targeted towards (Web) developers. However, through the use of mashup editors, lightweight Web composition is brought closer to tech-savvies thanks to the use of visual metaphors and wizard assistants. 
 
  \item[Prototypical and Opportunistic Nature] Traditional composite applications in the enterprise software business are often achieved either using proprietary programming solutions or WS-* services with composition standards such as BPEL (Business Process Execution Language)~\cite{Hoyer2008,Pautasso2008}. On the contrary mashups are often used for more ad-hoc applications such as rapid prototypes or to create applications that fits the needs of individuals or a handful of people with more relaxed quality of service and security requirements~\cite{Pautasso2008,Brandt2009}. However, in the last few years, mashups have evolved to be also considered as a valid development technique for the world of enterprise applications~\cite{Hoyer2008}. 
\end{description}

\subsubsection{\pMashups{}}
We propose a unified view of the Web of today and tomorrow's Web of Things in applications called \important{\pMashups{}}~\cite{Wilde2007,Guinard2009-INSS,Guinard2010-WoT}. Tech-savvies, i.e., end-users at ease with new technologies, can create \pMashups{} by composing virtual and physical services. Following the trend of Web~2.0 participatory services and in particular Web mashups, users can create applications mixing real-world devices such as home appliances or sensors with virtual services on the Web. As an example, a Hi-Fi system could be connected to Facebook or Twitter in order to post the songs one listens to the most.

We distinguish three mashup development approaches for \pMashups{} and relate them to their main target groups:\begin{description}
 \item[\mashupLevelA{}] Introduced in~\cite{Yu2008a}, we refine this type of development in a \WoT{} context as: development of composite applications that involve \sts{} by means of Web technologies such as HTML, HTTP, Atom and JavaScript but without requiring the use of specific mashup tools. This type of development is meant to be undertaken by actual developers. However, thanks to the previously presented layers and approaches, \sts{} are brought to Web developers rather than embedded systems specialists~\cite{Mottola2011}. We used this type of development approach for instance to realize the \newterm{Energie Visible} mashup presented in \sectRef{evisibleUserInterface}.

 
 \item[\mashupLevelB{}] In this type of development a software framework, sometimes called \important{portal} communicates with the \sts{} and makes their data available through a black-board~\cite{Yu2008a} approach where the data are constantly written to variables in memory. The developers then simply have to create widgets (or portlets) that read and write to these variables. These widgets are usually written using a combination of HTML and JavaScript code. Since it fully abstracts the communication with \sts{} this model is a \important{higher abstraction level}. A direct consequence of this development model is that domain experts (e.g., experts in supply chain management) with IT skills can build composite applications for their domain without having to learn the subtleties of embedded systems. This development approach was used to create the \newterm{EPC Dashboard Mashup} presented in \sectRef{EPCDashboardMashup}.

 \item[\mashupLevelC{}] This development approach enables end-users to create their own composite applications. In the case of Web 2.0 Mashups this type of application is usually developed through a mashup editor, e.g., Yahoo Pipes~\citeweb{pipes}, which is a Web platform that enables people to visually create simple rules to compose Web sites and data sources. A similar approach can be applied to empower users to create small applications tailored to their needs on top of their \sts{}. In the next section we discuss the specific requirements of Physical Mashup Editors and describe the architecture of a platform for building these editors. 
\end{description}

\subsection{From Web 2.0 Mashups Editors to Physical Mashup Editors}
While Web 2.0 Mashup techniques and tools can be largely re-used for the \WoTLong{}, the physical world has some special constraints that need to be addressed when designing \pMashups{} editors. We deduce these constraints based on a case-study in which we adapted an existing Web 2.0 Mashup Editor to be used as a \pMashups{} editor. Then, for the identified constraints we propose a number of requirements~\cite{Guinard2010-mashup-home}.

\subsection{Adapting a Web 2.0 Mashup Editor to the \WoTLong}\label{clickscript}
\begin{figure}
\imgLine{wot/cs-light-sensor}
\caption{A \pMashup{} with a modified version of the Clickscript Mashup editor. The mashup turns a lamp on whenever the light level observed by a real-world sensor is above a threshold.}
\label{fig:cs-light-sensor}
\end{figure}
To better understand the requirements of a \pMashups{} editor, we adapted an existing Web 2.0 Mashup Editor to include building-blocks featuring access to \sts{}. Our case-study is based on the Clickscript project~\cite{Naef2009}. Clickscript if a Web platform~\citeweb{clickscript} written in JavaScript and HTML on top of two popular JavaScript libraries (Dojo~\citeweb{dojo} and JQuery~\citeweb{jquery}). Clickscript allows people to visually create Web 2.0 Mashups by connecting building-blocks of resources (e.g., Web pages, strings, etc.) and operations (e.g., greater than, if/then, loops, etc.).

We decided to use Clickscript for two main reasons: First, since the editor was created only with client-side Web technologies its deployment and extension is very straightforward and can illustrate well the integration of \sts{} to pure Web scripting languages. Then, Clickscript was created with the aim of teaching young children the basics of programming. As a consequence, its usage is very simple and accessible even to non-technical people~\cite{Naef2009}.

Since Clickscript is written in JavaScript and running in the browser, it cannot use resources based on low-level protocols such as Bluetooth or Zigbee. However, it offers full HTTP support and hence can easily access RESTful services. As \WoT{} devices implemented using the architecture described in the \devLayer{}, \findLayer{} and \shareLayer{} are fully accessible through a RESTful Web API, it is straightforward to create Clickscript building-blocks supporting \sts{}. 

We used this approach to create Clickscript building-blocks for all the devices we present in the case-studies of this thesis (see \chapterRef{wsn} and \chapterRef{autoid}). The generic JavaScript code required to integrate a \st{} as a ClickScript building-block is shown in \lstRef{csBuildingBlock}. This concise snippet of code is a template of all that is required to integrate any \sts{} that implements, at least, the \devLayer{} of the \WoTA{}.

The result of this script is a new ClickScript building-block that can be used by end-users to create simple \pMashups{}. As an example, the mashup shown in \chapterRef{wsn} (\figRef{clickscriptRule}) gets the light level in a room by \code{GET}ting the light resource of a sensor. If it is bigger than a given threshold, it turns the light off by sending it a \code{PUT} request.

\lstinputlisting[caption=Generic JavaScript code required to integrate a new smart thing to the Clickscript mashup editor as a Clickscript building-block, label=lst:csBuildingBlock, breaklines, numbers=left, numberstyle=\tiny, xleftmargin=0.8cm, basicstyle=\small\ttfamily, backgroundcolor=\color{gray}, captionpos=b]{code/clickScriptBuildingBlock.js}


This readily illustrates the simplicity of adapting an existing mashup editor to \WoT{} devices thanks to their Web integration. However, it also illustrates the shortcomings of the approach. First, while creating mashups can be done by end-users, creating new building-blocks is still only accessible to the community of Web developers. To prevent this, a \sts{} discovery mechanism should be implemented in order to automate the creation of the corresponding building-blocks.

Second, for the mashup shown in \figRef{clickscriptRule}, the mashup editor has to constantly pull the temperature from the device which is sub-optimal. Hence the need to support push mechanisms as described in \sectRef{push}. This lack of push support is a common characteristic of client-side mashup editors since an HTTP Callback (Web Hook) approach is not possible in this case. However, we adapted Clickscript to support HTML5 WebSockets. 

The original version of Clickscript offers two ways of executing mashups: First, the end-user can manually start the mashup by pressing a \code{Run} button triggering the execution process. Alternatively, he can use a \code{Repeated-Run} button which runs the mashup in an infinite loop. We added an asynchronous, push-based execution model. The extended execution method using WebSockets is shown in \figRef{seqMashup}. When using this option, the user is prompted for the URI of an HTML5 WebSocket enabled server, e.g., an instance of the \tpusher{} service running on a \sg{} (see \sectRef{real-time-and-tpusher}). This URI is used to register the Clickscript client to the WebSocket server.

Each incoming HTML5 WebSocket message triggers the execution process. Furthermore, following a black-board approach~\cite{Yu2008a}, the payload of each incoming HTML5 message is extracted and written to a variable accessible to all building-blocks. 
\begin{figure}
\centering
\imgLine{wot/sequence-push-clickscript}
\caption{Sequence diagram of a typical Web-push triggered execution in the extended version of the Clickscript mashup editor. The editor subscribes to a WebSocket (or Comet) topic and writes the incoming messages to a black-board variable which is then read by building-blocks.}
\label{fig:seqMashup}
\end{figure}
A concrete \pMashups{} prototype based on this Web-push enabled version of Clickscript is described in \chapterRef{autoid}, \sectRef{boxesAndPointers}.

\subsection{Requirements for Physical Mashup Editors}
As a result of the prototype built on top of the Clickscript Web 2.0 Mashup editor, we propose a number of requirements for editors of \pMashups{}~\cite{Guinard2010-mashup-home}:
\begin{description} 
  \item[Support for event-based mashups] The current Web, and thus the vast majority of Web 2.0 mashup editors, are based on the concept of clients pulling information from servers~\cite{Yu2008a}. Several studies~\cite{Duquennoy2009a,Trifa2010} have shown that while this model matches the requirements for controlling \sts{}, it is inefficient for real-world monitoring applications. Hence, the need for \pMashups{} editors to offer core support for event-based interactions, where parts of the workflows of mashups can be triggered based on events pushed from \sts{} to the editors using Web push mechanisms.

  \item[Support for dynamic building-blocks] Manually creating building-blocks for each thing does not scale with the heterogeneity of objects in the physical world. Thus, the need for the mashup editors to support partially automated integration through service discovery techniques leveraging the \findLayer{}.

  \item[Support for non-desktop platforms] Web 2.0 mashup editors are for the large part meant to run in Web browsers of desktop computers. However, in the case of the \WoTLong, the in situ development of \pMashups{} e.g., on mobile phones or tablets should be fostered as virtual interactions with the physical world can really benefit from occurring beyond the desktop metaphor~\cite{Brodt2008,Ullmer1998}.

  \item[Support application specific editors] Due to the heterogeneity of use-cases in the \WoTLong, a \quote{one-size-fits-all} mashup editor is very unlikely to use adapted metaphors and tools for a particular domain. Hence, rather than creating a mashup editor, the architecture should be a mashup platform exposing an API that can be used to develop specific mashup editors (e.g., a mashup editor for supply chain related or home automation use-cases). This also lets users create their mashups locally, e.g., on a mobile phone, and export them to a more robust framework for execution.
\end{description}

%\todo{2: Analyze the existing mashup pforms against these requirements}

\subsection{A Platform for Physical Mashups Editors}\label{mashupFw}
\begin{figure}
\imgMedium{wot/pmashupFwOverview}
\caption{Overview of the \pMashupsFw{}. The framework is the mediator between \sts{}, virtual services and clients (domain-specific mashup editors).}
\label{fig:mashupFwork}
\end{figure}
The goal of the \pMashupsFw{}~\cite{Aguilar2010,Guinard2010-mashup-home,Kovatsch2010} is to offer a platform that fulfills the requirements discussed before. Rather than providing a generic mashup editor, the \pMashupsFw{} is a mashup engine, i.e., a Web service capable of running mashups~\cite{Yu2008a}.

As shown in \figRef{mashupFwork}, the framework is a composition environment between Web-enabled \sts{} and virtual services such as messaging or visualization services. It features a RESTful Web API using which actual mashup editors can be built. These editors use the framework for managing the life-cycles of mashups, from the definition of mashups to the discovery of virtual and smart things' services and the actual execution of the mashups.

The \pMashupsFw{} is not a mashup editor itself. Indeed, the idea is for the framework to support the creation of domain-domain specific mashup editors. For instance, in \sectRef{wsn-mashup-editor} we build a mobile mashup editor dedicated to create simple applications that optimize the energy consumption of household appliances. The mobile application uses the \pMashupsFw{} RESTful Web API to create and run the mashups in the framework's engine.

\subsection{System Architecture}
In this section we further describe the functionality of the \pMashupsFw{} by focusing on the most important components of the architecture as shown in \figRef{mashupFworkCompo}.

\subsubsection{Discovery Component}
The \code{DiscoveryComponent} implements the requirement for \important{supporting dynamic mashup building-blocks}. It is an implementation of the \transService{} described in \sectRef{translationService} that uses semantic annotations crawled from the \sts{} HTML representation to generate an internal representation of the Smart Thing Description model.

Mashup editors can then retrieve a serialized version of this description (in the form of a WADL file) that they can use to dynamically generate relevant user interfaces for the building-blocks corresponding to the newly discovered \st{}.

\subsubsection{Workflow Engine}
Core to the \pMashupsFw{} is a mashup engine. This engine is responsible for the life-cycle of \pMashupsFw{}. It compiles the mashups into a runnable workflow composed of several building-blocks and runs it.

Rather than creating an engine from scratch, the \pMashupsFw{} is based on the Ruote workflow engine~\citeweb{ruote}. Ruote is an open-source lightweight workflow engine that is well suited to manage workflows that call several services on the Web, especially when these services are HTTP-based and RESTful.

\paragraph{A Domain Specific Language for Workflows}\label{mashupLanguage}
To create workflows, Ruote provides a Domain Specific Language~\citeweb{ruote} that we reuse in the workflow engine of the \pMashupsFw{}. We briefly describe the most important language construct of the workflow DSL:

\begin{description} 
  \item[Expression] A Ruote-based workflow describes a process composed of \code{Expression}s. Each step in the process is represented as an \code{Expression}.
  
  \item[Workitem] \code{Expression}s communicate with each other based on a message passing mechanism. The message is initialized at the beginning of the process and modified by each \code{Expression}. In the Ruote DSL, such a message is called \code{Workitem}.

  \item[Participant] are the most important form of \code{Expression}s, they perform the business logic of the workflow at each step of the process. The engine manages the orchestration among \code{Participant}s by sending and receiving the \code{Workitem}s. Ruote provides a large set of predefined \code{Participant}s but new ones can be added very easily by implementing two methods: \code{initialize} and \code{consume}. The former is called whenever a participants is added to a workflow whereas the latter is called when a \code{Workitem} is applied by the engine to the \code{Participant}.
\end{description}

Further constructs of the workflow DSL are common language elements such as process-definition constructs, sequences, conditional expressions, loops, subprocesses and listeners. \lstRef{workflow} presents a simple process (i.e., workflow) definition in XML. Ruote supports such definitions in XML, Ruby or JSON.
\lstinputlisting[caption=A typical process (workflow) definition using an XML representation of the Ruote Workflow DSL language., label=lst:workflow, breaklines, numbers=left, numberstyle=\tiny, language={Java}, xleftmargin=0.8cm, basicstyle=\small\ttfamily, backgroundcolor=\color{gray}, captionpos=b]{code/workflow-def.xml}


\subsubsection{\pMashups{} Building-Blocks Library}\label{blocks}
\begin{figure}
\imgMedium{wot/pmashupFw}
\caption{Most important components of the \pMashupsFw{}. The Workflow Engine manages the life-cycle of mashups and runs them as workflows. The Discovery Component supports the dynamic integration of new metadata-annotated \sts{}. The Building-Blocks Library provides the building-blocks for creating \pMashups{} and the Repositories are used to store mashups, building-blocks and data coming from \sts{}.}
\label{fig:mashupFworkCompo}
\end{figure}
To adapt it to \pMashups{}, we extend the Ruote DSL with \WoT{} specific building-blocks. These blocks complement the existing DSL in the form of \code{Participants} constructs. 

The interesting aspect is that rather than specifying them using a programming language, these blocks can be specified through the RESTful API of the \pMashupsFw{} using either XML or JSON representations sent as payloads of HTTP messages. As a consequence, clients (i.e., mashup editors) can build mashups online, block by block using the framework API.

As shown in \figRef{mashupFworkCompo}, we created three types of building-blocks, \newterm{source}, \newterm{processing} and \newterm{target} blocks. We describe each of these tasks below.

\paragraph{Source Blocks}
These building-blocks correspond to the inputs of the mashups, i.e., \sts{} or virtual services on the Web. Four types of source blocks are defined and implemented:

\begin{description}
 \item[REST Blocks] Are the most important building-blocks from a \WoT{} perspective. They encapsulate HTTP interactions with resources. They can be used to add \sts{}, that were previously discovered by the \code{DiscoveryComponent}, to the workflows. Similarly, they can be used to interact with RESTful virtual services on the Web. \lstRef{restBB} shows an example of \code{RESTBlock} used to retrieve the temperature value of a Web-enabled sensor node. 

 \item[SOAP Blocks] In order to support WS-* services, a SOAP building-block is implemented. Since the workflow engine takes care of the invocation, this component is especially interesting in the case of mashup editors running on resource-constrained devices such as mobile phones as these might not be able to invoke WS-* services directly. 

 \item[Repository Blocks] In more complex \pMashups, a solution to persist data is often required. As an example reacting on the power consumption of a particular device might require to store the measurements over time and perform analysis on aggregated data. For this purpose, a \code{RepositoryBlock} allows to create persistent collections of data that can be queried later using processing blocks.

 \item[Subscription Blocks] As we have seen before, \sts{} might provide their functionality asynchronously, pushing it back to the clients whenever it becomes relevant, hence the need to support \important{event-based mashups}. The \code{SubscriptionBlock}s implement this requirement using an HTTP Callback approach (see \sectRef{push}). These blocks can be used to subscribe to events and clients can specify which building-block of the workflow will be the recipient of incoming events. Internally, a \code{SubscriptionBlock} generates a callback URI containing the name of the recipient block that will be caught by the Mashup Entry Point component and routed to the correct block.
 \end{description}

\lstinputlisting[caption=XML definition of a REST building-block retrieving the temperature value from a \WoT{} sensor node., label=lst:restBB, breaklines, numbers=left, numberstyle=\tiny, language={XML}, xleftmargin=0.8cm, basicstyle=\small\ttfamily, backgroundcolor=\color{gray}, captionpos=b]{code/restBB.xml}

\paragraph{Processing Blocks}
As shown in \figRef{mashupFworkCompo}, Processing Blocks represent the logic (e.g., mathematical operations, filtering, querying, etc.) between inputs and outputs. As defined by Yee~\cite{Yee2008} a common operation in mashups is to translate the data into a form meaningful to the destination service. Part of the translation process is the extraction of the relevant data. \code{QueryBlock}s offer a simple language construct to perform the most frequent data extraction operations.

The basic input of a \code{QueryBlock} is a data source or a set of data sources. Data sources can be either XML or JSON representations specified by URIs pointing either to actual resources on the Web or to internal resources stored using a \code{RepositoryBlock}.

A \code{QueryBlock} can be configured by means of several parameters:
  \begin{itemize}
    \item \code{Select} allows to configure which attributes should be extracted as outputs of the query. This parameter is functionally similar to a SQL \code{select}.
    \item \code{First} allows limiting the number of returned attributes (e.g., the first 10 attributes).
    \item \code{Filter} allows to do some simple filtering on the data. It supports logical operators, comparison operators as well as simple regular expressions.
    \item \code{Sort} is similar to the SQL \code{order by} operator.
    \item \code{GroupBy} allows to get a summarized view of the queried data. Similarly to the SQL \code{group by} statement it aggregates the data. The supported aggregation functions are max, min, sum, count and average.
  \end{itemize}

\lstRef{queryBB} provides an example of the definition of a \code{QueryBlock}. The output of the block is a JSON document that contains the values of the last trace recorded by a GPS sensor (e.g., a mobile phone).

\lstinputlisting[caption={XML definition of a Query building-block. This block extracts, from a repository, the latest location sent by a GPS sensor.}, label=lst:queryBB, breaklines, numbers=left, numberstyle=\tiny, language={XML}, xleftmargin=0.8cm, basicstyle=\small\ttfamily, backgroundcolor=\color{gray}, captionpos=b]{code/queryBB.xml}


\paragraph{Target Blocks}
As shown in \figRef{mashupFworkCompo}, \code{TargetBlock}s are the outputs of compositions. They represent services that can be actuated as a result of processing the input data. Four types of \code{TargetBlock}s are supported by the framework:
\begin{description}
 \item[AtomPub Blocks] These components supports publishing JSON, XML or HTML data to any server complying with the AtomPub protocol described in \sectRef{push}. 

 \item[Twitter Blocks] These components encapsulate the Twitter API, letting data coming from \sts{} being pushed to Twitter.

 \item[XMPP Blocks] These components support the XMPP messaging protocol used in several Web messaging clients such as Google Chat and sometimes used to provide real-time data from \sts{}~\cite{Hornsby2010,Hornsby2009}.

 \item[Visualization Blocks] These blocks offer a set of visualization methods such as graphs that are implemented using the Google Visualization API~\citeweb{googlevisu}.
 \end{description}

Additionally, the \code{RESTBlock}s and \code{RepositoryBlock}s can be used as \code{TargetBlock}s as well. \code{RESTBlock}s enable, for instance, \sts{} to \sts{} communication and actuation through the \pMashupsFw{}. \code{RepositoryBlock}s, allows to use the framework as a database for \sts{}, where the real-world data can be preprocessed using \code{QueryBlock}s before storage.

\subsubsection{Repositories}
The \pMashupsFw{} offers two repositories. The Collection Repository is used by \code{RepositoryBlock}s to store JSON or XML data.
The Resource Repository mainly provides a persistent storage for mashups as well as extracted instances of the Smart Things Description model. Through this repository parts of or complete workflows can be referenced and used in other mashups, allowing for a reuse and sharing mechanism.

\subsubsection{RESTful API}
\begin{figure}
\imgLine{wot/mashup-fw-api}
\caption{Tree-representation of the RESTful API of the \pMashupsFw{}.}
\label{fig:mashupFworkAPI}
\end{figure}
As with every component of the \WoTA{}, the \pMashupsFw{} is based on a RESTful architecture and features a RESTful API as shown in \figRef{mashupFworkAPI}. Mashup editors and other clients can use it to build the mashups using an XML or JSON representation of the building-blocks. Furthermore, the API allows client applications to retrieve existing mashups and manage running instances.

The API is centered around the notion of a \code{user} where mashup definitions, discovered \sts{} and collections are associated with a specific user of the framework but can be shared amongst users.

\subsection{Discussion and Summary}
In this chapter we introduced the idea of \pMashups{}. We defined three categories of development approaches for \pMashups{}: First, \newterm{\mashupLevelA{}} which helps developers building upon \sts{} by streamlining the development process to simple Web development. In \sectRef{evisibleUserInterface} we demonstrate how this development approach made it easy to realize the \newterm{Energie Visible} mashup for energy-awareness. Similarly, in \sectRef{epcFind} we illustrate how it made possible to create the EPCFind prototype that leverages real-world RFID data to track and trace someone's belongings. Furthermore, with the \newterm{Ambient Meter} prototype presented in \sectRef{AmbientMeter}, we illustrate how the approach can also help \sts{} to \sts{} communication.

Then, with the \newterm{\mashupLevelB{}} the actual communication with \sts{} is transparent to the developer who simply has to use the incoming data (in a black-board approach) to create new applications encapsulated in JavaScript and HTML Widgets. We will demonstrate this development approach with the \newterm{EPC Dashboard Mashup} presented in \sectRef{EPCDashboardMashup}.

Finally, with the \newterm{\mashupLevelC{}} we explain how visual metaphors and simple editors can be used to enable end-users to create simple compositions. We illustrate this a straightforward adaptation of an existing mashup editor to \WoT{} devices. We further introduce the \pMashupsFw{} architecture that allows to build domain and device specific \pMashups{} editors and run the created mashups in the cloud. In \sectRef{energyMashupEditor} we propose a mashup editor that can be used to create simple applications on top of Web-enabled home appliances in order to make homes more energy aware and efficient.
\section{Developers Perspectives on the WS-* Alternative Architecture}\label{alternatives}

The application architecture proposed in this chapter is definitely not the only way to create a uniform integration layer for \sts{} and we will discuss several alternatives in the related work section (see \sectRef{relatedWork}). 

However, in this section we consider WS-* services as an important alternative\footnote{An introduction to WS-* services is outside the scope of this thesis and we invited the reader to refer to books such as~\cite{Alonso2010} or~\cite{Erl2004} for an exhaustive overview of the underlying technologies.}. Out of the possible alternatives, WS-* services stand out for two reasons: First their current market penetration, especially in the field of enterprise software business is significant~\citeweb{soapundead}. Second, rather than proposing a simple network layer integration, the WS-* galaxy of standards forms an ecosystem that addresses, in a standard way, several layers of a \WoTA{}.

WS-* services declare their functionality and interfaces in a Web Services Description Language (WSDL) file. Client requests and service responses objects are encapsulated using the Simple Object Access Protocol (SOAP) and transmitted over the network, usually using the HTTP protocol. Further WS-* standards define concepts such as addressing, security, discovery or service composition. Although WS-* was initially created to achieve interoperability of enterprise applications, work has been done to adapt it to the needs of resource-constrained devices~\cite{Helander2005,Priyantha2008,Yazar2009,Spiess2009}. Furthermore, lighter forms of WS-* services, such as the Devices Profile for Web Services (DPWS)~\citeweb{ws4d}, were proposed~\cite{Jammes2005-DPWS}. 

While they share similar goals, REST and WS-* are not compatible; they tackle loose coupling and interoperability differently. Consequently, work has been done to evaluate these two approaches. In~\cite{Pautasso2009,Pautasso2008}, REST and WS-* are compared in terms of re-usability and loose coupling for business applications. The authors suggest that WS-* services should be preferred for \quote{professional enterprise application integration scenarios} and RESTful services for {tactical, ad-hoc integration over the Web}. 

Internet of Things applications pose novel requirements and challenges as neither WS-* nor RESTful Web services were primarily designed to run and be used on \sts{}, but rather on business or Web servers. This development thus necessitates assessing the suitability of the two approaches for devices with limited capabilities. Yazar et al.~\cite{Yazar2009} analyze the performance of WS-* and RESTful applications when deployed on wireless sensor nodes with limited resources and conclude that REST performs better. 

However, evaluating the performance of a system when deployed on a particular platform is not enough to make the architectural decision that will foster adoption and third-party (public) innovation. Indeed, studies like the Technology Acceptance Model~\cite{Davis1989} and more recent extensions~\cite{Gefen1998,Mathieson2001} show that the perceived ease of use of an IT system is key to its adoption. As many manufacturers of \sts{} are moving from providing devices with a few applications to building devices as platforms with APIs, they increasingly rely on external communities of developers to build innovative services for their hardware (e.g., the Apple App Store or Android Marketplace). An easy to learn API is, therefore, key in fostering a broad community of developers for \sts{}. Hence, choosing the service architecture that provides the best developer experience is instrumental to the success of the Internet of Things and the Web of Things on a larger scale.

In this section we complement the decision framework that can be used when picking the right architecture for IoT applications and platforms. We supplement previous work~\cite{Pautasso2009,Pautasso2008,Yazar2009} by evaluating, in a structured way, the actual \important{developers' experience} when using each architecture in an IoT context. We analyze the perceived ease of use and suitability of WS-* and RESTful Web service architectures for IoT applications. We base our study on the qualitative feedback and quantitative results from 69 computer science students who developed two applications that accesses temperature and light measurements from wireless sensor nodes. For the one of the applications, the participants used a sensor node implementing the \devLayer{} and \findLayer{} of the \WoTA{}, thus offering a semantically annotated RESTful Web API. In the second case, they were accessing a WS-* (WSDL + SOAP based) sensor node.

Our results show that participants almost unanimously found RESTful Web services easier to learn, more intuitive and more suitable for programming IoT applications than WS-*. The main advantages of REST as reported by the participants are, intuitiveness, flexibility, and the fact that it is lightweight. WS-* is perceived to support more advanced security requirements and benefits from a clearer standardization process.

The following sections are structured as follows. \sectRef{devStudyMethodology} describes the study methodology. \sectRef{devStudyResults} presents and analyses the results. Finally, \sectRef{devStudyConclusion} discusses the implications of our findings and devises guidelines. The results of the presented study were published in \cite{Guinard2011-rest-vs-ws}.

\subsection{Methodology}\label{devStudyMethodology}
\begin{figure}
\imgMedium{wot/study-rest-vs-ws}
\caption{Setup of the user study. Two \sunspots{} sensor nodes are connected to a \sg{} through a sync-based Device Driver (left). Their functionality is exposed through a RESTful Web API. Two other \sunspots{} are accessible through a WS-*, SOAP + WSDL interface deployed in an application server (right).}
\label{fig:studySetup}
\end{figure}

%In this section we discuss out study setup, participants' demographics and data analysis techniques used.
Our study is based on a programming exercise and the feedback received from a group of 69 computer science students who had to learn RESTful and WS-* Web service architecture and implement, in teams, mobile phone applications that accessed sensor data from different sensor nodes using both approaches as shown in \figRef{studySetup}. The exercise was part of an assignment in the Distributed Systems course at ETH Zurich\footnote{The assignment is available online: \url{guinard.org/phd/study-assignment.pdf}}.

To get and parse the RESTful sensor response, participants used the HTTP and JSON libraries, which are already available on Android phones. To perform the WS-* request, the external \newterm{kSoap2} library~\citeweb{ksoap} was used. Averaging over the submissions, the programs had 105 lines of code for the WS-* implementation ($SD=50.19$, where SD is the Standard Deviation), opposed to 98 lines of code for the REST implementation ($SD=48.31$). 

The two coding tasks were solved in teams of two or three members who were able to freely decide how to split up the tasks amongst team members, which models industry practice. The coding tasks were successfully completed by all teams within two weeks. To ensure that every team member had understood the solution of each task, individual reports describing the design and architecture choices had to be submitted. Additionally, every student had to submit a structured questionnaire about both technologies and a voluntary feedback form on the learning process. Students were informed that answers in the feedback form were not part of the assignment, and that responses would be used in a research study. To encourage them to give honest answers about the amount of effort invested in solving both coding tasks, perception, and attitudes towards both technologies, entries in the feedback form were made anonymously. \tableRef{methodology} summarizes the data collection sources.

\textbf{Demographics:}
The participants were from ETH Zurich, in their third or fourth year of Bachelor studies. Teams were formed of two or three members. They were taught both technologies for the first time, in a short introduction during the tutorial class. 89\% reported not having had any previous knowledge of WS-* and 62\% none of REST. From the 35\% that had already used REST before the course, half reported that they had previously been unaware that the technology they used actually was based on the REST architecture. 5\% (2 students) had programmed WS-* applications. 

\textbf{Additional Tasks:}
Subsequent tasks involved creating visualization mechanisms for the retrieved data, both locally and through cloud visualization solutions. 

\begin{table}[htdp]
\begin{center}
\begin{tabular}{|l| c | c |}
%\rowcolors {1}{orange!35 }{ }
\hline
\textbf{Data Source}& \textbf{Type}&\textbf{N} \\
\hline
RESTful and WS-* Applications & Team &25 \\
%WS-* Application & \\
%Other &  &\\
%\hline
%Documentation & Individual& 73\\
\hline
Structured Questionnaire & Individual & 69\\
 \hline
Feedback Questionnaire & Anonymous& 37\\
\hline
 \end{tabular}
\end{center}
\caption{Data was collected from different programming tasks and questionnaires.}
\label{tab:methodology}
\end{table}

\subsection{Results}\label{devStudyResults}
Here, we present our results on the perceived differences, learning curve, and suitability of the technologies for smart things related use-cases.

\subsubsection{Perceived Differences}
Using the structured questionnaire, we collected qualitative data on the perceived advantages of both technologies with respect to each other (see \tableRef{perdiff}). While REST was perceived to be \participantSaid{very easy to understand, learn, and implement,} \participantSaid{lightweight and scalable}, WS-* \participantSaid{allows for more complex operations,} provides higher security, and a better level of abstraction.

\begin{table}[h]
\begin{center}
  \begin{tabular}{ | l  l | l |}
    \hline
    \multicolumn{2}{|l|}{\rule{0cm}{0.3cm}\textbf{REST} ($N=69$)} & \#\\
    \hline
    & Easy to understand, learn, and implement & 36\\
    & Lightweight & 27\\
    & Easy to use for clients & 25\\
    & More scalable & 21\\
    & No libraries required & 17\\
    & Accessible in browser and bookmarkable & 14\\
    & Reuses HTTP functionality (e.g., caching) & 10\\
    \hline
    \multicolumn{2}{|l|}{\rule{0cm}{0.3cm}\textbf{WS-*} ($N=69$)} & \#\\
    \hline
    & WSDL allows to publish a WS-* interface & 31\\
    & Allows for more complex operations & 24\\
    & Offers better security & 19\\
    & Provides higher level of abstraction & 11\\
    & Has more features & 10\\
    \hline
  \end{tabular}
  \caption{Participants felt that WS-* provides better abstraction, but REST is easy to learn and use.}
  \label{tab:perdiff}
\end{center}
\end{table}


\subsubsection{Learning Curve}
\begin{figure}
\imgLine{wot/rest-ws-study-speed}
\caption{A majority of participants reported that REST as \newterm{fast} or \newterm{very fast} to learn and WS-* as \newterm{not fast} or \newterm{average}.}
\label{fig:learning-speed}
\end{figure}
In the feedback form, we asked participants to rate on a 5 point Likert scale how easy and how fast it was to learn each technology (1=not easy at all, ..., 5=very easy).  As shown in \figRef{learning-easiness}, 70\% rated REST \quote{easy} or \quote{very easy} to learn.  WS-* services, on the other hand, were perceived to be more complex: only 11\% respondents rated them easy to learn. Event if all participants were required to learn and explain the concepts of both technologies, compare their advantages, analyze their suitability and explain design decisions, we restricted the sample to participants who reported to have worked on programming both REST and WS-* assignments within their teams (N=19) to avoid bias. We then applied a paired two sample t-test to compare the learning curve for REST and WS-*. Our results show that REST (with an average $M=3.85$ and a Standard Deviation $SD=1.09$) was reported to be significantly easier to learn than WS-* ($M=2.50, SD=1.10, t(19)=4.23, p<0.001$. Similarly, REST ($M=3.43, SD=1.09$) was perceived to be significantly faster to learn than WS-* ($M=2.21, SD=0.80, t(13)=-3.46, p=0.002$).

Furthermore, in the feedback form, we collected qualitative data on the challenges of learning both technologies, asking the participants: \quote{What were the challenges you encountered in learning each of the technologies? Which one was faster and easier to learn?}. Nine participants explained that REST was easier and faster to learn because RESTful Web services are based on technologies, such as HTTP and HTML, which are well-known to most tech-savvy people: \participantSaid{Everybody who is using a browser already knows a little about [REST].} 

WS-* was perceived to be overly complicated: \participantSaid{REST is easy and WS-* is just a complicated mess.} Reasons for such  strong statements were the complexity of extracting useful information out of the WSDL and SOAP files (mentioned by 8), as well as the number and poor documentation of parameters for a SOAP call. The lack of clear documentation was perceived as a problem for REST as well: Seven participants said that further request examples, alongside with the traditional documentation (e.g., Javadoc) for both REST and WS-*, would be very useful. Eight participants explicitly mentioned that they had had previous experience with REST during their spare time. This illustrates the accessibility and appeal of RESTful Web services, and it positions them as an ideal candidate for smart things APIs in terms of lowering the entry barrier for creating applications. 
In the feedback form, 25 participants said that REST made it easier to understand what services the sensor nodes offered. Eight participants explained this by the fact that, for REST, an HTML interface was provided. This emphasizes that RESTful smart things should offer an HTML representation by default. Seven participants found WS-* easier for this matter. They noted that a WSDL file was not necessarily easy to read, but they liked the fact that it was \participantSaid{standard}.
\begin{figure}
\imgLine{wot/rest-ws-study-ease}
\caption{Participants reported REST as easier to learn than WS-*.}
\label{fig:learning-easiness}
\end{figure}

\subsubsection{Suitability for Use-Cases}
In the feedback form, we asked participants to rate on a Likert scale (1=WS-*, ..., 5=REST) which one of the two technologies they would recommend in specific scenarios. REST was considered more suitable than WS-* for IoT applications running on embedded devices and mobile phones (see \figRef{programming-suitability}). The one sample t-test confirmed that the sample average was statistically higher than the neutral 3, and therefore inclined towards REST. This was the case both for embedded devices ($M=3.86, SD=1.03, t(36)=5.10, p<0.001$), and for mobile phone applications ($M=3.51, SD=1.12, t(36)=2.79, p=0.004$). For business applications, however, a higher preference for WS-* was stated but the preference was not emphasized enough to be statistically significant ($M=2.67, SD=1.33, t(36)=-1.48, p=0.07$).

\paragraph{General Use-Cases}
We asked our participants to discuss the general use-cases for which each technology appeared suitable. When asked: \quote{For what kind of applications is REST suitable?}, 23 people mentioned that REST was well adapted for simple applications offering limited and atomic functionality: \participantSaid{for applications where you only need create/read/update and delete [operations]}. 8 participants also advised the use of REST when security is not a core requirement of the application: \participantSaid{Applications where no higher security level than the one of HTTP[s] is needed}. This is supported by the fact that the WS-* security specification offers more levels of security than the use of HTTPS and SSL in REST~\cite{Richardson2007,Kubert2011}. 
6 participants suggested that REST was more adapted for user-targeted applications: \participantSaid{[...] for applications that present received content directly to the user}. Along these lines, 14 users said that REST was more adapted for Web applications or applications requiring to integrate Web content: \participantSaid{[for] Web Mashups, REST services compose easily}.

We then asked: \quote{For what kind of applications is WS-* more suitable?}. Twenty participants mentioned that WS-* was more adapted for secure applications: \participantSaid{applications that require extended security features, where SSL is not enough}. 16 participants suggested to use WS-* when strong contracts on the message formats were required, often referring to the use of WSDL files: \participantSaid{with WS-* [...] specifications can easily be written in a standard machine-readable format (WSDL, XSD)}.

\begin{table}[h]
\begin{center}
  \begin{tabular}{ | l  l | l |}
    \hline

    \multicolumn{2}{|l|}{\rule{0cm}{0.3cm}\textbf{REST} (N=37)} & \textbf{\#} \\
    \hline
    & For simple applications, with atomic functionality & 23\\
    & For Web applications and Mashups & 14\\
    & If security is not a core requirement & 8\\
    & For user-centered applications & 6\\
    & For heterogeneous environments &6\\
   \hline
    \multicolumn{2}{|l|}{\rule{0cm}{0.3cm}\textbf{WS-*} (N=37)} & \textbf{\#}\\
    \hline
    & For secure applications & 20\\
    & When contracts on message formats are needed & 16\\
    \hline
  \end{tabular}
  \caption{REST was perceived to be more suited for simple applications, and WS-* for applications where security is important.}
  \label{tab:suitability}
\end{center}
\end{table}


\paragraph{For \sts{}} 
Both WS-* and RESTful Web Services were not primarily designed to run on embedded devices and mobile phones but rather on business or Web servers. Thus, assessing the suitability of the two approaches for devices with limited capabilities is relevant. 

% Which solution would you recommend for providing services from embedded devices?
% Which technology do you think would be more suitable for low-end devices like embedded computers and mobile phones?
As shown in the first part of \figRef{programming-suitability}, for providing services on embedded devices, 66\% of the participants suggested that REST was either \quote{adapted} to \quote{very-adapted}. When asked to elaborate on their answers, 6 participants suggested that for heterogeneous environments REST was more suitable: \participantSaid{for simple application, running on several clients (PC, iPhone, Android) [...]}. 7 participants said that REST was adapted for embedded and mobile devices because it was more lightweight and better suited for such devices in general: \participantSaid{the mapping of a sensor network to the REST verbs is natural [...]}. To confirm this, we investigated the size of the application packages for both approaches. The average footprint of the REST application was 17.46 kB while the WS-* application had a size of 83.27 kB on average. The difference here is mainly due to the necessity to include the \textit{kSoap2} library with every WS-* application. These results confirm earlier performance and footprint evaluations~\cite{Drytkiewicz2004-pREST,Yazar2009,Guinard2010-WoT}.

\subparagraph{Smart Home Applications}
We then went into more specific use-cases, asking: \participantSaid{Imagine you want to deploy a sensor network in your home. Which technology would you use and why?}. Sixty-two respondents recommended REST to deploy a sensor network in the home, 5 recommended WS-*, and 2 were undecided. Twenty-four participants justified the choice of REST by invoking its simplicity both in terms of use and development: \participantSaid{REST [...] requires less effort to be set up}, \participantSaid{Easier to use REST, especially in connection with a Web interface}. Eight participants said that REST is more lightweight, which is important in a home environment populated by heterogeneous home appliances. Interestingly enough, 14 participants mentioned that in home environments there are very little concerns about security and thus, the advanced security features of WS-* were not required: \participantSaid{I would not care if my neighbor can read these values}, \participantSaid{The information isn't really sensitive}.

\subparagraph{For Mobile Phones}
Since the mobile phone is a key interaction device for creating IoT applications, we asked the participants to assess the suitability of each platform for creating mobile phone clients to smart things. As shown in the second part of \figRef{programming-suitability}, 53\% of the participants would use REST, 16\% would use WS-* and 32\% were undecided. They explained these contrasted results by the fact that mobile phones are getting very powerful. 7 participants explained that the amount of data to be processed was smaller with REST which was perceived as an important fact for mobile clients. Interestingly enough, some participants considered the customers of mobile platforms to have different requirements: \participantSaid{I would use REST, since customers prefer speed and fun over security for smaller devices}. The lack of native WS-* support on Android (which natively supports HTTP) and the required use of external libraries was also mentioned as a decision factor as REST calls can be simply implemented only using the native HTTP libraries.

\paragraph{For Business Applications}
The results are much more inclined towards WS-* when considering \quote{business} applications. As shown in the third part of \figRef{programming-suitability}, the majority of our participants (52\%) would choose WS-* and 24\% REST for servicing business applications. Twenty-one (out of 69, see Table~\ref{tab:methodology}) justify their decision by the security needs of enterprise applications: \participantSaid{I would rely on the more secure WS-* technology}. Eighteen participants talk about the better described service contracts when using WS-*: \participantSaid{I propose WS-* because we could use WSDL and XSD to agree on a well-specified interface early on [...]}. Amongst the participants suggesting the use of REST, 10 justify their decision for its simplicity and 10 for its better scalability.

\begin{figure}
\imgLine{wot/rest-ws-study-suitability-domain}
\caption{Participants reported that REST is better suited for Internet of Things applications, involving mobile and embedded devices and WS-* fit better to the constraints of business applications.}
\label{fig:programming-suitability}
\end{figure}

\section{Discussion and Summary}\label{devStudyConclusion}
A central concern in the Internet of Things and thus in the \WoTLong{} is the interoperability between smart objects and existing standards and applications. Two service-oriented approaches are currently at the center of research and development: REST and WS-*. Decisions on which approach to adopt have important consequences for an IoT system and should be made carefully. In this section, we complement existing studies on performance metrics with an evaluation of the developers' preferences, and IoT programming experiences with REST and WS-*. In the context of the presented study, the results show that REST stands out as the favorite service architecture for IoT applications. Future studies should conduct a long-term assessment of the developers' experience, beyond the initial phase of getting started with the technologies. Furthermore, work could be done to compare the experience of advanced, not just novice developers, possibly within industry projects. However, the more experienced the developers are, the more they are likely to develop a bias towards one or the other technology.
We summarize the decision criteria used by developers in our study and devise guidelines in \tableRef{guidelines}. 

\begin{table}[h]
\begin{center}
  \begin{tabular}{ | p{5cm}  l  l  p{6cm} | }
    \hline
    \textbf{Requirement} & \textbf{REST} & \textbf{WS-*} & \textbf{Justification}\\
    \hline
    Mobile \& Embedded & + & - & Lightweight, IP/HTTP support\\
\hline
    Ease of use & ++ & - & Easy to learn\\
\hline
    Foster third-party adoption & ++ & - & Easy to prototype\\
\hline
    Scalability & ++ & + & Web mechanisms\\
\hline
    Web integration & +++ & + & Web is RESTful\\
\hline
    Business & + & ++ & QoS \& security\\
\hline
    Service contracts & + & ++ & WSDL\\
\hline
    Adv. security& - & +++ & WS-Security\\
   \hline
  \end{tabular}
  \caption{Guidelines for choosing a service architecture for IoT platforms.}
  \label{tab:guidelines}
\end{center}
\end{table}
%Based on the results, in  we propose guidelines for making the right architectural decision when creating an IoT platform. 
While our results confirm several other research projects that take a more performance-centric approach~\cite{Drytkiewicz2004-pREST,Yazar2009,Guinard2010-WoT}, they contradict several industry trends. In home and industrial automation, standards such as UPnP, DLNA or DPWS expose their services using WS-* standards. One of the reasons for this, also noted by participants, is the lack of formal service contracts (such as WSDL and SOAP) for RESTful services. This is an arguable point as Web experts~\cite{Richardson2007} already illustrated how a well-designed RESTful interface combined with HTTP content negotiation results in a service contract similar to what WS-* offers~\cite{Richardson2007,Kubert2011}, with no overhead and more adapted to services of \sts{}~\cite{Guinard2010-WoT}.
Yet, this illustrates an important weakness of RESTful Web services identified by participants: RESTful Web services are a relatively \important{fuzzy} concept. Even if the basics of REST are very simple, the lack of a clear stakeholder managing \quote{standard} RESTful architectures is subject to many (wrong) interpretations of the concept. Until this is improved, resources such as~\cite{Fielding2000,Richardson2007} profile themselves as de-facto standards.

In cases with strong security requirements, WS-* has a competitive advantage~\cite{Richardson2007,Kubert2011}. The WS-Security standard offers a greater number of options than HTTPS (TLS/SSL) such as encryption and authentication beyond the communication channel, endpoint-to-endpoint. In theory, these could also be implemented for RESTful Web services. However, the lack of standard Web support of these techniques would result in tight coupling between the secured things and their clients. Nevertheless, in the context of \sts{}, it is also important to realize that WS-Security standards are much more resource intensive than those of HTTPS and, thus, rarely fit resource-constrained devices.

Furthermore it is important to consider \important{how accessible smart things should be}. Participants identified that RESTful Web services represent the most straightforward and simple way of achieving a global network of smart things because RESTful Web services seamlessly integrate with the Web. This goes along the lines of recent developments, such as 6LoWPAN~\cite{Mulligan2007} and the IPSO alliance~\citeweb{ipso}, CoRE~\citeweb{core} and CoAP~\cite{Shelby2010}, or the \WoTA{} presented in this thesis,
where \sts{} are increasingly becoming part and leveraging the infrastructure of the Internet and the Web.

Lastly, it is worth noting that while the decision of adopting a WS-* or RESTful architecture for a \sts{} platform is important, bridges between the two architectures can be created with some efforts~\cite{Engelke2010,Guinard2011a,Karnouskos2010}, for instance through the use of application layer gateways or \sgs{} drivers (see \sectRef{gateways}) as we suggested in~\cite{Guinard2010-Search}.


\section{Related Work}\label{relatedWork}
In this section we discuss previous research related to the \WoTA{}. It is structured according to the layers of \WoTA{} beginning with work related to creating a global network of \sts{} (\devLayer{}) and ending with ways of composing the services of \sts{} in an easy and accessible manner (\compoLayer{}).

\subsection{\devLayer{}}
Linking the Web and physical objects is an attractive idea that has already been in the mind of researchers for years. Early approaches started by attaching physical tokens (such as barcodes) to objects to direct the user to pages on the Web containing information about the objects~\cite{Want1999}. These pages were first served by static Web Servers on mainframes, then by early gateway system that enabled low-power devices to be part of wider networks~\cite{Schramm2004}. The Cooltown project pioneered this area of the physical Web by associating pages and URIs to people, places and things~\cite{Kindberg2002} and implementing scenarios where this information could by physically discovered by scanning infrared tags in the environment. Similarly, the SPREAD physical/spatial computing model~\cite{Couderc2003} consisted in spreading, with the help of wireless technologies, information that could then be retrieved by mobile users, pointing them to contextually related Web resources. The key idea of these works was to provide a virtual counterpart of the physical objects on the Web. URIs to Web pages were scanned by users e.g., using mobile devices and directed them to online representations of real things (e.g., containing status of appliances on HTML pages or user manuals).

A number of projects proposed solutions to expose the functionality of \sts{} through APIs in order to build applications upon real-world devices. Among them, JINI, UPnP, DNLA, etc. The advent of WS-* Web Services (SOAP, WSDL, etc.) led to a number of works towards deploying them on embedded devices and sensor networks~\cite{Priyantha2008,DeSouza2008,Marin-Perianu2007}. DPWS is a subset of the WS-* standards that allows minimal interaction with web services running on embedded devices. DPWS specifies a protocol for seamless interaction with the services offered by different embedded devices. DPWS is aligned (and for the most part compatible) with WS-* technologies. The various specifications DPWS include support for messaging, service discovery, service description, and eventing for resource-constrained devices.

We joined this research and proposed a middleware that bridges the gap between DPWS embedded-devices and enterprise applications, taking care of cross-cutting concerns such as search, service composition, dynamic provisioning of services and data storage~\cite{Spiess2009,DeSouza2008}. However, in \sectRef{alternatives} we discussed the shortcomings of the approach in the case of the \WoT{}.

Several systems (or middleware) for integration of \sts{} with the Web have been proposed such as SenseWeb~\cite{Grosky2007,Luo2008}, the Global Sensor Network (GSN)~\cite{Aberer2007}, Pachube~\citeweb{pachube}, ThingSpeak~\citeweb{thingspeak}, Sen.se~\citeweb{sense}, ThingWorx~\citeweb{thingworx} or Sensor.Network~\cite{Gupta2010}. These offer platforms for people to share their sensory readings using Web services to transmit data onto a central server and thus cover several layers of the \WoTA{}. However, these approaches are focused on building central repositories on top of which services can be built. 

The idea to push the Web and the Internet as close as possible to \sts{} was explored early on in the Internet0 project~\cite{Gershenfeld2006}. Gershenfeld and Cohen understood early on the importance of having a single protocol for \sts{} and proposed adapting Internet protocols to embedded devices. 

With advances in computing technology, tiny Web servers can be embedded in most devices~\cite{Dunkels2003,Hui2008,Duquennoy2009a} and thus not only Internet but also Web technologies can be pushed to \sts{}. Hence, in the \devLayer{}, we propose using the Web and its technologies as the integration backbone of \sts{}. We build upon works towards the Web integration of wireless sensor networks such as~\cite{Luckenbach2005,Drytkiewicz2004-pREST,Dickerson2008} and extend them by systematically applying the constrains of RESTful architectures to \sts{} and discussing the concept of \sgs{} to integrate non-IP enabled devices~\cite{Guinard2011b}. Furthermore, we complement this work by proposing the three other layers that streamline the development of applications using \sts{} to what can be done today with Web 2.0 mashups. 

\subsection{\findLayer{}}
Search has become such a central commodity that it is hard to imagine the Web without search engines. The findability of information~\cite{Morville2005}, people and places has become a central concern of most information architectures~\cite{Morville2006}. However, while search in the Web of documents is already significantly advanced, searching the real-world remains one of the biggest open challenges for the \WoTLong{} to materialize. 

When considering the \WoTLong{} beyond micro use-cases such as home or factory automation, focusing on macro use-cases where billions of things are available and connected to the Web, then discovery simply by browsing HTML pages with hyperlinks becomes difficult. Searching for things is significantly more complicated than searching for documents~\cite{Romer2010}, as things are tightly bound to contextual information such as location, are often moving from one context to another and their HTML representations are less keywords-rich than traditional Web pages. 

As a consequence, several researchers have been looking at specific ways of describing \sts{} and domain-specific standards have been proposed: SensorML~\cite{Botts2007} is a standard XML language that can be used to describe sensor network applications and devices. Similarly, the Extended Environments Markup Language (EEML)~\citeweb{eeml} is a language for describing sensor data in digitally enhanced environments.

Not specific to Wireless Sensor Networks and thus closer to the concerns of this thesis, DPWS~\cite{Jammes2005-DPWS} proposes a \newterm{device metadata language}~\citeweb{dpws} which offers a semantic description of what a real-world device is, by what company it was produced and what it has to offer in terms of functionality. DPWS metadata is often embedded in a WSDL (Webservice Description Language) file which in turns contains the interface description or API of the offered WS-* Webservices. Based on DPWS we proposed the SOCRADES Lookup and Search Infrastructure~\cite{Guinard2009b}.

The metadata offered by these languages is the basis of our proposed \stm{} model. However, these formats are not well supported and understood on the Web and for some parts overlap what HTTP already has to offer. As a consequence, they do not leverage the search infrastructure already in place on the Web (e.g., existing search engines). 

Closer to Web languages, researchers have been proposing the use of domain-specific ontologies to support semi-automated mashups of \sts{}~\cite{Vermeulen2007a,Katasonov2010}. Vermeulen et al. used this approach to implement mashups of tagged physical objects with services on the Web~\cite{Vermeulen2007}. While the expressive power of these approaches is significant so is their complexity. Our goal is to enable users to search the real-world and machines to understand the basics of \sts{} (e.g., in order to generate building-blocks of mashups). As a consequence we would like to consider descriptions that can be implemented in a lightweight manner with a constrained but sufficient descriptive power so that they can be understood by a vast number of existing infrastructure services. 

SA-REST~\cite{Lathem2007} proposes a lightweight alternative language that can be used to describe RESTful services sharing similarities to what the WSDL language can describe for WS-* services. This is also the proposal of the WADL (Web Applications Description Language)~\cite{Hadley2006} or of some extensions in the WSDL 2.0 language. However, SA-REST goes beyond these languages as it proposes to also support the composition of these services in semi-automatic mashups~\cite{Sheth2007}. Unfortunately, languages like SA-REST are not widely understood on the Web, for instance they are not processed by search engines. The importance of relying on mechanisms that can be well-understood by search engines for a realistic deployment of a service lookup service was discussed by Song et al.~\cite{Song2007}. 

Kopecky et al. propose to make descriptions of RESTful Web services directly embedded in HTML representations~\cite{Kopecky2008}. They propose the hRESTs microformat. Note that it is sometimes criticized by the Web community because most of the data the hRESTs microformat encapsulate can also be retrieved by crawling Web architectures that respect the REST constraints (and in particular the \important{connecteness} constraint). However, when deploying services directly on \sts{} (without \sgs{}) this format can be interesting as it avoids the great number of HTTP calls required by crawlers. Researchers such as Alarcon et al. discuss ways and languages to gather more valuable information when crawling RESTful services~\cite{Alarcon2010}.

Our approach is to propose the \stm{} model inspired by these description languages but we focus on a small set of properties that are sufficient to allow the findability of \sts{} and their integration into tools such as \pMashups{} editors. We further implement the model using a set of standard microformats as these are well supported and broadly understood by search engines. Yet, relying on existing search engines does not fully leverage \sts{} and does not really take into account their specific requirements such as mobility or their strong ties with physical locations. As a consequence a body of research is related to real-time search engines for the Internet and the \WoTLong{}.

As explained in the previous section, several platforms such as SenseWeb~\cite{Grosky2007,Luo2008}, Pachube or Sensor.Network~\cite{Gupta2010} were proposed. Building on top of Pachube, Kamilaris et al. propose \quote{bridging the Mobile Web and the \WoT{} in Urban Environments}~\cite{Kamilaris2010} and offer a mobile search engine using location information to retrieve nearby data services. Similarly, we explored using the concept of proximity to dynamically search for services offered by \sts{}~\cite{Gellersen2009}. However, these approaches are based on a centralized data lookup infrastructure and do not fully leverage the distributed nature of \sts{}.

In the world of business services, a crucial challenge for SOA developers and process designers is to find adequate services for solving a particular task~\cite{Crasso2008}. Discovering enterprise services often implies manually querying a number of registries, such as Universal Description and Discovery and Integration (UDDI) registries, and the results depend largely on the quality of the data within that registry. While such an approach is adequate for a rarely changing set of large-scale services, the same is insufficient for the requirements of the service offered by \sts{}. Registering a service with one or more UDDIs is rather complex (which is also why UDDIs are rarely used in practice), and does not comply with the usage minimization of the devices' limited resources. Furthermore extensive description information is necessary~\cite{Monson-Haefel2003}, while the \sts{} can only report basic information about themselves and the services they host. Trying to reduce the complexity of registration and discovery, different research directions have been followed in order to provide alternatives or extensions of the UDDI standard~\cite{Crasso2008,Song2007}. However, also these do not take into account the particular requirements of real-world services.  

R\"{o}mer et al. survey search engines for the real-world~\cite{Romer2010} and present their own engine in~\cite{Ostermaier2010} to search for real-world dynamic conditions such as \quote{finding the most quiet place in a city}. In their approach, probabilistic models are used to determine which actual sensors to contact for a particular query. Frank et al. study the use of a distributed query infrastructure composed of mobile nodes that can be used to search for the location of real-world objects~\cite{Frank2007,Frank2008}. A concept that inspired us to implement an application with RFID enabled mobile phones on top of the EPC network infrastructure as published in~\cite{Guinard2008} and described in \sectRef{epcFind}.

These engines and applications focus on leveraging the dynamic conditions of \sts{} which is a very promising approach to create real-world computational engines. Our target with the \findLayer{}, however, is slightly different. Indeed, we enable the search not for particular real-world situations but for services provided by \sts{} corresponding to user queries. This supports developers, tech-savvies and end-users in finding the services required to create composite applications such as \pMashups{}.

Haodong et al. took such an approach and propose Snoogle~\cite{HaodongWang2010}. This search engine can be used to find a particular mobile object or a list of objects that are likely to serve the requested service. It uses information retrieval techniques to maintain indexes of keywords corresponding to \sts{}. These indexes are managed by so called \important{Index Points} that are local to the \sts{} (e.g., one Index Point per room). On top of these local Index Points, a single mediator is maintaining an aggregate view of the whole network. A similar approach is taken by Yap et al. in MAX~\cite{Yap2005}. However, unlike in Snoogle where the \sts{} push information to the Index Point, in MAX metadata (keywords) are pulled from the infrastructure and the nodes upon queries. These approaches represent powerful real-world search engines but their integration to global networks such as the World Wide Web was not addressed.

Closer to our work, in a theoretical paper, Stirbu proposes leveraging the distributed infrastructure of \sts{} to achieve a discovery system on the Web~\cite{Stirbu2008}. The author considers the idea of devices registering themselves to a Registry Service through a simple HTTP \code{POST} call. However, the device has to post its full description to the Registry Service that publishes it in an Atom feed. This is a rather strong coupling between the device and the Registry. We avoid this by proposing an architecture inspired from Stirbu's proposal but rather consider the \sts{} as passive actors that just submit their root URIs either directly or through users. The actual semantic metadata extraction process is handled by the infrastructure. Just as search engines crawl pages of information without hard constraints on those pages, our LLDUs (with the help of a \transService{}) crawl the \sts{} to extract relevant metadata.
                                                                                                                                                                                                                                                                                                                                                                                                                                                                                                                                                                                                                                                                                                                                                                                                                                                                                                                                While we clearly do not pretend offering a comprehensive way to describe and search for all types of \sts{} on the Web, our approach leverages the lightweight infrastructure put in place at the \devLayer{} of the \WoTA{} and extends it with LLDUs that are used to retrieve metadata and offer localized service lookup queries. Furthermore, thanks to the combination of RESTful Web APIs respecting the REST constraints and different implementations of the \stm{} model (e.g., microformats or RDFa\cite{Aguilar2010}), we have a flexible and rather loosely-coupled way of crawling metadata without strong requirements on the \sts{} themselves.


\subsection{\shareLayer{}}
In recent years, the world experiences a renewed trend towards sharing physical resources in all kinds of domains~\cite{Botsman2010}. More specifically, in the Internet of Things domain, data sharing was identified as one of the key enablers~\cite{Vazquez2008,Blackstock2011}. As a consequence several research platforms such as SenseWeb~\cite{Grosky2007,Luo2008}, the Global Sensor Networks (GSN)~\cite{Aberer2007}, the SOCRADES Integration Infrastructure~\cite{DeSouza2008} or SensorBase~\cite{Chen2007} appeared. These early approaches inspired research in the field but did not use Web standards which requires additional bridges to re-use the data they hold on the Web.

The recent trend to adopt Web technologies and in particular RESTful architectures has influenced platforms such as Pachube or Sensor.Network~\cite{Gupta2010} which propose a solution by providing a central platform for people to share their sensor data. However, these approaches are based on a centralized data repository to which the data is pushed and do not allow authorized and authenticated direct interaction with \sts{} as we enable it with the \sacLong{} architecture. More importantly, unlike \sac{}, these platforms are based on their own access control lists which are hard to scale, maintain and manage.

Vasquez~\cite{Vazquez2008} introduced the notion collaboration between social networks and smart objects. Furthermore, several research projects have been focusing on leveraging the value of social graphs from existing social networks to share \sts{}. In~\cite{Blackstock2011} Blackstock et al. provide a survey of Social Web of Things projects. Several projects explore the use of Twitter as a publishing and sharing mechanism for \sts{}~\citeweb{things-tweet}. As an example, the S-Sensors project~\cite{Baqer2009} specifically looks at the use of Twitter as a messaging and sharing mechanism for Wireless Sensor Networks. The SenseShare~\cite{Schmid2007} project allows users to share sensor data with their friends. It also allows owners to apply different filters to the data before sharing it. 

While SenseShare was a source of inspiration for the \sac{} architecture, it presents some shortcomings. Similarly to Pachube or Sensor.Network, SenseShare acts as a data store between the sensors and the clients. SenseShare and S-Sensors further allow sharing the data coming from sensors but do not support direct interactions with the sensors. As an example, one can't enable switching on/off devices by close relatives. Similarly, a Web-enabled Hi-Fi system couldn't enable songs to be played remotely through a RESTful interface which access is managed by the sharing system.

Furthermore, SenseShare or S-Sensors require the use of Facebook, respectively Twitter. Such a tight coupling with a single external service whose contract (API and allowed accesses) is subject to change over time, is problematic. It is also restrictive as it prevents from using a more adapted social network for a specific use-case. As an example, LinkedIn might be more adapted for a B2C (Business to Consumer) or B2B (Business to Business) sharing of \sts{}. This led us to the interoperability requirement of \sac{}, which supports different social networks, and enables users to control which one to use for each \st{}. 

Recent work, published after the \sac{} project~\cite{Guinard2010-sharing}, also leverages recent standards such as OAuth, OpenID and OpenSocial to offer an interoperable solution to real-world data sharing. The SENSE-ATION~\cite{Shirazi2010} project enables sharing sensory data coming from mobile phones. Its focus, however, is to offer this information to the developers of applications hosted on social networks (e.g., OpenSocial Gadgets). Similarly, Parimpu~\cite{Pintus2011} offers a platform inside which applications based on sensor streams can be created and shared with Twitter contacts.

\subsection{\compoLayer{}}
The idea of physical devices that can be composed together and with their environment to create new applications has been long dreamed of by pioneers such as Mark Weiser~\cite{Weiser1991} and often refined since then for instance in the vision of heterogeneous homes~\cite{Aipperspach2009}.

Key in these vision is the notion of end-users being able to create these simple composite applications on their own. Implementations of this vision appeared thanks the evolution of opportunistic programming in which developers have access to tool-chains and programming languages helping them to iterate more rapidly, creating small prototypes with a very limited amount of code~\cite{Brandt2009,Hartmann2008}. This evolution was further fostered by the developments of Web technologies and languages considered as relatively easy to use and develop upon~\cite{Roelands2011}. 

Following these developments, the idea of enabling opportunistic applications by mixing the physical world (i.e., \sts{}) and the Web appeared~\cite{Kindberg2002}. Later, these opportunistic applications were influenced by the concepts of Web 2.0 mashups studied for instance in~\cite{Yee2008,Yu2008a}, where end-users are empowered to create simple composite applications on the Web. Wilde proposed the notion of \important{\pMashups{}} as \quote{[...] new applications using this unified view of the Web of today and tomorrow’s Web of Things}. In~\cite{Guinard2009} we proposed an experimentation and implementation of \pMashups{} and refined the notion together with Trifa and Wilde in~\cite{Guinard2010-WoT}.

Furthermore, in~\cite{Roelands2011}, Roelands et al. discuss the notion of \quote{Do-it-Yourself} in the space of the Internet of Things. They define the concept of \quote{Smart Composables Internet of Things} and explain how physical mashups can contribute to end-user re-usability of \sts{}.

Several researchers explored ways of easily combining physical objects and Web technologies to create ad-hoc applications. Vermeulen et al. proposed to let users link physical object to composite services on the Web using RFID tags~\cite{Vermeulen2007}. Similarly, Vasquez and Lopez-de-Inpina explored several simple applications where end-users could combine physical devices with services on the Web~\cite{Vazquez2008}. 

Rather than considering single use-case, we propose an architecture that supports several types of mashups. In this space, researchers identified the mobile phone as being a key platform for \pMashups{}~\cite{Maximilien2008}, thanks to its ubiquitous Web access. Brodt and Nicklas~\cite{Brodt2008} present an architecture for creating mashups on mobile phones using JavaScript and HTML as well as a mashup server where wrappers for each service are implemented. Mikkonen et al. propose a mashup framework running on embedded devices~\cite{Mikkonen2010}. The architecture of these solutions targets the use of mobile devices as \sts{}, in our approach we would like to support other types of \sts{} and enable their dynamic discovery.

Several projects explore more generic solutions where different smart objects can be composed to create new applications. Vermeulen et al. propose the use of Semantic Web Technologies to enable for semi-automated mashups between the physical and digital world~\cite{Vermeulen2007a}. This approach is promising but taking an automated approach slightly differs from the level of flexibility envisioned in \pMashups{} were people can \important{re-wire} the physical world~\cite{Hartmann2008} themselves, using simple compositions and rules.

Carboni and Zanarini propose the concept of Hyperpipes inspired from the Unix pipes~\cite{Carboni2007}. Hyperpipes are defined using a relatively simple Domain Specific Language based on the concept of sink and source objects that can be connected together. As an example, hyperpipes can be used to redirect the screen output of a laptop to a board. The authors further create a mobile application from which the \sts{} can be piped together.

We take a similar approach by proposing an extended language set (DSL) for \pMashups{} offering more possibilities and slightly more complex mashups. Furthermore, rather than proposing a single mashup editor we build the framework as a service featuring a RESTful API that can be used to create and run \pMashups{} and \pMashup{} editors.


\section{Summary}
In this chapter, we presented our \WoTA{} and its four layers. Rather than being a strictly layered architecture we suggest it should be taken as a set of architectural guidelines that help facilitate, brick by brick, the integration of \sts{} into Web applications.

In the \devLayer{}, we propose to push the Internet and the Web down to the \sts{} themselves. We explain how a Resource Oriented architectural approach can be used to model the services \sts{} have to offer and provide them through a uniform API for the real-world based on REST.
Furthermore, we discuss the need for \sts{} to push events rather than being constantly polled and propose a solution based on the upcoming developments of the Web such as HTML5 WebSockets.

For devices that cannot connect to the Internet and offer their services through a Web server, we propose the concept of \sgs{} which act as reverse proxies that can be dynamically extended to support new kinds of \sts{} and proprietary or lower-level protocols. We also provide an evaluation of the differences in terms of performance between an end-to-end HTTP approach and a synchronization-based \sg{} mediated approach. With this, we illustrate how a \sg{} helps scaling deployments and applications.

In the \findLayer{} we propose a set of metadata that covers the most important data required to enable searching for \sts{} on the Web and to automate processes such as simple UI rendering or the automatic generating of mashup building-blocks for \sts{}. We implement this model by using microformats combined with the crawling of RESTful APIs. We also extend the network of Web-enabled \sts{} and \sg{} with the concept of Local Lookup and Discovery Units that enable the registration and indexing of \sts{}. Furthermore, they allow mashup developers and users to formulate several types of local and contextual queries that help them finding the right services offered by \sts{}.

In the \shareLayer{} we emphasize on the importance of having an authentication and sharing mechanism for \sts{}. Instead of creating anonymous access control lists we leverage social networks and create a proxy called \sacLong{} that bridges social networks and the \WoT{}, implementing a Social \WoTLong{} in which owners of \sts{} can share their devices with friends, colleagues or relatives. This proxy can be deployed at several places in the network and manages both the access to things and the authentication on social networks through their Web connectors.

In the \compoLayer{} we adapt an existing mashup framework to Web-enabled \sts{} and illustrate how this process is made straightforward thanks to the use of Web standards. We further introduce the \pMashupsFw{} that was specifically designed to manage the life-cycle of \pMashups{}.

Finally, we consider a WS-* alternative architecture. We analyze the body of research comparing WS-* and RESTful architecture for \sts{} and complement these evaluations by a qualitative evaluation describing the experience of developers implementing a prototype using both technologies. We conclude that WS-* services have advantages for applications requiring complex service contracts or with high security requirements. However, when considering ease of use, ease of learning, Web integration and fostering public innovation, the RESTful approach seems more adapted.

In the next two chapters, the presented architecture is applied and evaluated in two concrete domains, Wireless Sensor Networks and automatic identification networks.
